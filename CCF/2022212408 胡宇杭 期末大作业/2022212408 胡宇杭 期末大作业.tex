\documentclass[UTF8, 12pt, a4paper, oneside]{ctexart}
\usepackage{listing}

\usepackage{amsmath}
\usepackage{amsfonts}
\usepackage{geometry}
\usepackage{ulem}
\usepackage[most]{tcolorbox}
\usepackage[hidelinks]{hyperref}
\usepackage{color}
\usepackage{framed}
\usepackage{mathtools,amssymb}
\usepackage{bm}

\lstset{
    basicstyle          =   \sffamily,          % 基本代码风格
    keywordstyle        =   \bfseries,          % 关键字风格
    commentstyle        =   \rmfamily\itshape,  % 注释的风格,斜体
    stringstyle         =   \ttfamily,  % 字符串风格
    flexiblecolumns,                % 别问为什么,加上这个
    numbers             =   left,   % 行号的位置在左边
    showspaces          =   false,  % 是否显示空格,显示了有点乱,所以不现实了
    numberstyle         =   \zihao{-5}\ttfamily,    % 行号的样式,小五号,tt等宽字体
    showstringspaces    =   false,
    captionpos          =   t,      % 这段代码的名字所呈现的位置,t指的是top上面
    frame               =   lrtb,   % 显示边框
}

\lstdefinestyle{C++}{
    language        =   C++, % 语言选Python
    basicstyle      =   \zihao{-5}\ttfamily,
    numberstyle     =   \zihao{-5}\ttfamily,
    keywordstyle    =   \color{blue},
    keywordstyle    =   [2] \color{teal},
    stringstyle     =   \color{magenta},
    commentstyle    =   \color{red}\ttfamily,
    breaklines      =   true,   % 自动换行,建议不要写太长的行
    columns         =   fixed,  % 如果不加这一句,字间距就不固定,很丑,必须加
    basewidth       =   0.5em,
}

\title{期末大作业 \ 实验报告}
\author{学号:2022212408 \ 姓名:胡宇杭}
\date{\today}

\begin{document}
\maketitle

\section*{提交结果}
\begin{figure*}[htbp]
    \centering
    \includegraphics*[width = 5.7in]{提交结果.png}
    \caption{提交结果}
\end{figure*}

\tableofcontents

\newpage
\section{202012-4 食材运输}
\begin{itemize}
    \item 时间限制:$1.0\texttt{ s}$
    \item 空间限制:$512.0\texttt{ MB}$
\end{itemize}
\subsection{题目}
\subsubsection{题目背景}
\par 在 $T$ 市有很多个酒店,这些酒店对于不同种类的食材有不同的需求情况,莱莱公司负责每天给这些酒店运输食材。由于酒店众多,如何规划运输路线成为了一个非常重要的问题。你作为莱莱公司的顾问,请帮他们解决这个棘手的问题。
\subsubsection{问题描述}
\par $T$ 市有 $N$ 个酒店,由 $N - 1$ 条无向边连接,构成一棵树,通过某一条边的时间取决于该边的权值。$T$ 市有 $K$ 种食材,每个酒店对食材的需求不同。现有 $K$ 辆车,每辆车负责配送一种食材,且不存在重复。
\par 每辆车都可以从不超过 $M$ $(1 \leq M \leq K)$个检查点中选择一个作为出发点,现需要确定如何设置检查点和车辆的出发点,使得所有酒店等待时间最大值的最小值。
\subsubsection{子任务}
\par $30\%$ 的测试数据满足 $N \leq 10^2$,$M = K$,$K \leq 10$,保证输入数据是一条链,且节点编号满足$u + 1 = v$。
\par $40\%$ 的测试数据满足 $N \leq 10^2$,$M = K$,$K \leq 10$,无特殊性质。
\par $30\%$ 的测试数据满足 $N \leq 10^2$,$M \leq K$,$K \leq 10$,无特殊性质。
\subsection{题解}
\subsubsection{解法一 (65分)}
\par 注意到 $70\%$ 的数据满足 $M = K$,等价于对每辆车都可以设置一个检查点,此时问题转化为求解出第 $i$ 种食材在以点$u$为出发点的运输距离$dis_{i_u}$,取其最小值(最短路径),再对整体取最大值,即:$$ans = max\{min(dis_{i_u}, u \in N), i \in K\}$$
\par 而求最短路径可以通过树状 $dp$ 实现。不难发现,除了离出发点最远的节点,其他路径都需要经过两次,因此可以通过 $dp$ 求出经过路径的总长度,同时 $DFS$ 出最长路径。即:$dis_{i_u} = dp[u] - max\_dis$。
\par 该解法理论得分70分。(\sout{剩下5分我也不知道哪扣的})
\subsubsection{C++ 代码实现}
\lstinputlisting[
    style = C++,
    title = {\bf 202012-4 食材运输 65分代码}
]{202012-4-65.cpp}

\subsubsection{解法二 (100分)}
\par 从状压 $dp$ 的角度考虑,设 $S_j$ 为二进制状态为 $j$ 的食材种类的集合。则 $dp(i, S_j)$ 代表选择 $i$ 个检查点,运送食材种类为 $S_j$ 时的最短路径。状态转移方程为:$$dp(i, S_j) = min\{max(dp(1, T), dp(i - 1, S_j - T)), T \subseteq S_j\}$$
\par 证明:假设已知$dp(1, T)$,$dp(i - 1, S_j - T)$,则 $dp(i, S_j)$ 即是在 $dp(i - 1, S_j - T)$ 的基础上再选一个节点去运送剩余食材种类 $T$。由于题目只要求我们求出等待时间最大值的最小值,所以我们只要取两者的最大值即为该状态下最大等待时间,再遍历 $S_j$,选出使最大等待时间最小的 $T$,即为最短路径。
\par 初始化:同解法一,用树状 $dp$ 计算出 $dp(1, S_j)$,即: $$dp(1, S_j) = min\{max(dis_{u_i}), i \in S_j\}, u \in N$$
\begin{itemize}
    \item 时间复杂度:$DFS$需要遍历每个节点,时间复杂度为 $O(N)$,状压 $dp$ 最外层遍历 $M$ 个检查点,内层循环遍历全部状态和其全部子集,时间复杂度为 $O(M*(2^K)^2)$,总时间复杂度为 $O(M*(2^K)^2)$;
    \item 空间复杂度:$O(M*2^K + NK + N)$。
\end{itemize}
\subsubsection{C++ 代码实现}
\lstinputlisting[
    style = C++,
    title = {\bf 202012-4 食材运输 100分代码}
]{202012-4-100.cpp}


\newpage
\section{202303-4 星际网络II}
\begin{itemize}
    \item 时间限制:$2.0\texttt{ s}$
    \item 空间限制:$1.0\texttt{ GB}$
\end{itemize}
\subsection{题目}
\subsubsection{问题描述}
\par 管理一系列由 $n$ 位二进制位组成($n \ mod \ 16 = 0$),用 $IPv6$ 方法表示的地址。记 $num(s)$ 为地址 $s$ 按高位在前、低位在后组成的 $n$ 位二进制数,称一段连续区间的一系列地址。现需要支持对地址进行以下操作。
\begin{itemize}
    \item $1 \ id \ l \ r$:用户 $id$ 申请地址在 $l ~ r$ 的一段连续地址,当且仅当该段地址未被分配或该段地址只被部分(不完全)分配给用户 $id$ 时能申请成功。
    \item $2 \ s$:检查地址 $s$ 被分配给了哪个用户。
    \item $3 \ l \ r$:检查 $l~r$ 范围内的所有地址是否全被分配给了某个用户。
\end{itemize}
\subsubsection{子任务}
\begin{figure*}[htbp]
    \centering
    \includegraphics*[width =4.5in]{202303-4 子任务.png}
    \caption{子任务}
\end{figure*}
\subsection{题解}
\subsubsection{解法一 (100分)}
\par 题目要求涉及到区间修改、单点查询、区间查询,可以考虑使用线段树进行维护。节点储存当前区间的左右端点 $l \ r$,已被占用的区间长度 $cnt$ 以及该区间 $id$ 的最大最小值。对于每次操作:
\begin{itemize}
    \item 操作$1$:若该段区间 $cnt = 0$,或 $cnt \ != r - l + 1$ 且 $id_{min} = id_{max} = id_{now}$,则该区间可修改。
    \item 操作$2$:二分搜索查找即可。
    \item 操作$3$:若该段区间 $cnt = r - l + 1$ 且 $id_{min} = id_{max}$,则全被分配给了某用户。
\end{itemize}
\par 同时,我们注意到:子任务中存在 $n = 512$ 的数据,而 $long \ long$ 的范围只到 $2^{64} - 1$,并且 $2^{512}$ 量级肯定会爆内存。但子任务中总操作次数 $q \leq 50000$,因此可以使用离散化处理(\sout{和 202112-4 一模一样,简直白给})。
\par 由于题目很贴心的使用了 $IPv6$ 的地址表示方法,所以可以直接比较地址字符串来确定相对大小,处理下进位即可。最后,需要注意要把 $l + 1$、$s + 1$ 一并离散化,否则会出现诸如 $[1,2]$, $[4,5]$在离散化后两区间相邻的问题。
\begin{itemize}
    \item 时间复杂度:设离散化后共有$n$个点,线段树单次查询,修改的时间复杂度均为$O(logn)$,共有 $q$ 次查询,总时间复杂度为 $O(q*logn)$;
    \item 空间复杂度:$O(4n)$。
\end{itemize}
\subsubsection{C++ 代码实现}
\lstinputlisting[
    style = C++,
    title = {\bf 202303-4 星际网络II}
]{202303-4.cpp}

\newpage
\section{Codeforces 1786F Wooden Spoon}
\begin{itemize}
    \item 时间限制:$4.0\texttt{ s}$
    \item 空间限制:$512.0\texttt{ MB}$
\end{itemize}
\subsection{题目}
\subsubsection{问题描述}
\par 有 $2^n$ 个人(编号$1 \thicksim  2^n$) 参加比赛,比赛形式类似一颗完全二叉树,并且每次比赛总是二者中编号较小的一方赢得胜利。同时,当参赛者满足以下条件时会获得安慰奖 $Wooden \ Spoon$。
\begin{itemize}
    \item 该参赛者 $A$ 输了他的第一场比赛;
    \item 打败上一位参赛者 $A$ 的参赛者 $B$ 输了他的第二场比赛;
    \item 打败上一位参赛者 $B$ 的参赛者 $C$ 输了他的第三场比赛;
    \item ……………………;
    \item 打败上一位参赛者的参赛者输了最终决赛。
\end{itemize}
\par 试确定对于每一位参赛者,其能获得 $Wooden \ spoon$ 时的比赛安排方式的种类($modulo \ 998244353$)。
\subsubsection{子任务}
\par 一共有20个测试点,对于第 $i$ 个测试点,有$n = i$。
\subsection{题解}
\subsubsection{解法一}
\par 根据题意,我们不难发现:
\begin{itemize}
    \item 从根节点到获得$Wooden \ Spoon$叶子节点(假定权值为$i$) 的路径满足节点权值单调递增的性质,即:$1 = x_1 < x_2 < … < x_n < i$。
    \item 某一子树的根节点的权值一定是该子树中权值最小的叶子节点。
\end{itemize}
\par 对于上述路径,考虑使用 $dp$ 求解。设 $dp(i, j)$ 表示路径 $x_1 -> x_i = j$时的方案数,即权值为 $j$ 的节点在倒数第 $i-1$ 场比赛输掉的方案数。则有: $$ dp(i, j) = 2 * 2^{n-i}! \ C_{2^n - 2^{n-i} - j}^{2^{n-i}-1} \ \sum_{k = 1}^{j - 1}dp(i-1, k)$$
\par 证明:
\begin{itemize}
    \item $2 * 2^{n-i}!$:由完全二叉树的性质,节点$j$ 高度为 $i$,则其某一子树的叶子节点共有 $2^{n-i}$ 个;由性质二,无论这些叶子节点如何排列,最终胜出的一定为 $j$,所以对$j$所在子树的节点求全排列数,即$2^{n-i}!$;同时,$j$节点可能出现在左右子树,有两种可能,因此最终结果为$2*2^{n-i}!$。
    \item $C_{2^n - 2^{n-i} - j}^{2^{n-i}-1}$:目前已算出 $j$ 所在子树的 $2^{n-1}$个叶子节点的排列方式,还需确定 $j$ 的对手的子树;由性质二:以$j$ 为根节点的子树的叶子节点 $k$ 权值一定满足 $j < k \leq 2^n$,又因为 $j$ 赢得该场比赛,所以其对手节点权值肯定大于 $j$,此时只需要选择剩下 $2_{n-i} - 1$ 个叶子节点,同时,注意不能选择 $j$ 所在子树的 $2_{n-i}个$ 叶子节点。
    \item $\sum_{k = 1}^{j - 1}dp(i-1, k)$:选择一个权值小于 $j$ 的节点作为 $j$ 的对手,保证 $j$ 失败,其方案数共有 $\sum_{k = 1}^{j - 1}dp(i-1, k)$ 种。
\end{itemize}
\par 最终答案为:$\sum_{j = 1}^{i-1}dp(n, j)$,这是因为 $dp(n, j)$ 代表沿着上述路径,但只赢了第一场比赛的方案数,而我们要求的是沿上述路径第一场也输了的方案数。此时,对于编号为 $j$ 的参赛者,他可以选择与玩家 $i \ (i < j)$ 比赛来保证第一场比赛失败。
\par $PS$:由于题目要求对结果取模,所以可以用快速模指数运算和费马小定理求逆元(\sout{死去的离散又开始攻击我})。
\begin{itemize}
    \item 时间复杂度:$O(n*2^n)$;
    \item 空间复杂度:$O((2n + 2)*2^n)$。
\end{itemize}
\subsubsection{C++ 代码实现}
\lstinputlisting[
    style = C++,
    title = {\bf Codeforces 1786F Wooden Spoon}
]{1786-F.cpp}
\end{document}