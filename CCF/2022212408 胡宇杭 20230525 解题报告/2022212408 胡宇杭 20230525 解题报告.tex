\documentclass[UTF8, 12pt, a4paper, oneside]{ctexart}
\usepackage{listing}

\usepackage{amsmath}
\usepackage{amsfonts}
\usepackage{geometry}
\usepackage{ulem}
\usepackage[most]{tcolorbox}
\usepackage[hidelinks]{hyperref}
\usepackage{color}
\usepackage{framed}
\usepackage{mathtools,amssymb}
\usepackage{bm}

\lstset{
    basicstyle          =   \sffamily,          % 基本代码风格
    keywordstyle        =   \bfseries,          % 关键字风格
    commentstyle        =   \rmfamily\itshape,  % 注释的风格,斜体
    stringstyle         =   \ttfamily,  % 字符串风格
    flexiblecolumns,                % 别问为什么,加上这个
    numbers             =   left,   % 行号的位置在左边
    showspaces          =   false,  % 是否显示空格,显示了有点乱,所以不现实了
    numberstyle         =   \zihao{-5}\ttfamily,    % 行号的样式,小五号,tt等宽字体
    showstringspaces    =   false,
    captionpos          =   t,      % 这段代码的名字所呈现的位置,t指的是top上面
    frame               =   lrtb,   % 显示边框
}

\lstdefinestyle{C++}{
    language        =   C++, % 语言选Python
    basicstyle      =   \zihao{-5}\ttfamily,
    numberstyle     =   \zihao{-5}\ttfamily,
    keywordstyle    =   \color{blue},
    keywordstyle    =   [2] \color{teal},
    stringstyle     =   \color{magenta},
    commentstyle    =   \color{red}\ttfamily,
    breaklines      =   true,   % 自动换行,建议不要写太长的行
    columns         =   fixed,  % 如果不加这一句,字间距就不固定,很丑,必须加
    basewidth       =   0.5em,
}

\title{20230525 第四次作业 实验报告}
\author{学号:2022212408 姓名:胡宇杭}
\date{\today}

\begin{document}
\maketitle

\section*{提交结果}
\begin{figure*}[htbp]
    \centering
    \includegraphics*[width = 6in]{submit result.png}
    \caption{提交结果}
\end{figure*}

\section{HDU-2602 Bone Collector}
\begin{itemize}
    \item 时间限制:$1000\texttt{ ms}$
    \item 空间限制:$32768\texttt{ KB}$
\end{itemize}
\subsection{题目}
\subsubsection{问题描述}
\par Bone Collector 带着一容量为 $V$ 的背包,一共有 $N$ 块骨头。对每一次测试,已知每块骨头的体积和价值,问能装下骨头的最大价值?
\subsubsection{子任务}
\par 对于所有测试数据,满足 $N \leq 1000$,$V \leq 1000$。
\subsection{题解}
\subsubsection{解法一 (100分)}
\par 根据题意,考虑使用 $dp$ 求解(\sout{都学期末了连01背包都不知道的可以洗洗睡了})。状态转移方程如下:$$f(i, j) = max(f(i-1, j), f(i-1, j-w_i) + v_i)$$
\begin{itemize}
    \item 时间复杂度:$O(nm)$
    \item 空间复杂度:$O(n)$
\end{itemize}
\subsubsection{C++ 代码实现}
\lstinputlisting[
    style = C++,
    title = {\bf HDU-2602 Bone Collector}
]{test01.cpp}

\section{HDU-1712 ACboy needs your help}
\begin{itemize}
    \item 时间限制:$1000\texttt{ ms}$
    \item 空间限制:$32768\texttt{ KB}$
\end{itemize}
\subsection{题目}
\subsubsection{问题描述}
\par ACboy 这学期可以选择 $N$ 门课程,\sout{但 ACboy 是个摆子},所以这学期他只学 $M$ 天。对于课程 $i$ $(1 \leq i \leq N)$,ACboy 学习 $j$ $(1 \leq j \leq M)$ 天可以获得学分 $A_{ij}$,问 ACboy 怎么安排可以获得最大学分?
\subsubsection{子任务}
\par 对于所有测试数据,满足 $N \leq 100$,$M \leq 100$。
\subsection{题解}
\subsubsection{解法一 (100分)}
\par 与题一不同,每个课程的价值不在固定,考虑使用分组背包解决。具体来说即是在01背包的基础上再加一层循环遍历第 $i$ 门课程学习 $j$ 天的最优解。状态转移方程如下:$$dp(i, j) = max(dp(i-1, j), dp(i-1, j-k) + a_{ik})$$
\par 其中 $i$ 代表当前选择的课程,$j$ 代表当前剩余可学习天数,$k$ 代表当前课程的学习天数。
\begin{itemize}
    \item 时间复杂度:$O(nm^2)$
    \item 空间复杂度:$O(nm)$
\end{itemize}
\subsubsection{C++ 代码实现}
\lstinputlisting[
    style = C++,
    title = {\bf HDU-1712 ACboy needs your help}
]{test02.cpp}

\section{洛谷-P4124 手机号码}
\subsection{题目}
\begin{itemize}
    \item 时间限制:$1.0\texttt{ s}$
    \item 空间限制:$250\texttt{ MB}$
\end{itemize}
\subsubsection{问题描述}
\par 现需要统计号码段中满足某些特征的手机号的数量,特征如下:
\begin{itemize}
    \item 号码中要出现至少 $3$ 个相邻的相同数字;
    \item 号码中不能同时出现 $4$ 和 $8$。
\end{itemize}
\par 手机号码一定是11位数,前不含前导 $0$,给定 $L$,$R$,统计 $[L, R]$ 区间内所有满足条件的号码数量。
\subsubsection{子任务}
\par 对于所有测试数据,满足 $10^10 \leq L \leq R \leq 10^11$。
\subsection{题解}
\subsubsection{解法一 (100分)}
\par 数位dp,\sout{暴力枚举!}。假设 $dp(pos, pre_1, pre_2, 0/1, 0/1, 0/1)$, 其中,$pos$ 代表当前位数,$pre_1$、$pre_2$ 分别代表 $pos+1$、$pos+2$位的数,三个 $0/1$ 分别代表是否出现过 $3$ 个连续且相同的数、是否出现过 $4$、是否出现过 $8$,按照题目条件确定状态如何转移即可。由于 $dp$ 代码实现晦涩难懂(\sout{懒}),这里采用 $dfs$ 记忆化搜索的形式。
\subsubsection{C++ 代码实现}
\lstinputlisting[
    style = C++,
    title = {洛谷-P4124 手机号码}
]{test03.cpp}

\section{洛谷-P1352 没有上司的舞会}
\subsection{题目}
\begin{itemize}
    \item 时间限制:$1000\texttt{ ms}$
    \item 空间限制:$134144\texttt{ KB}$
\end{itemize}
\subsubsection{问题描述}
\par 某大学有 $n$ 个职员,他们之间有从属关系,除校长以外每个职员都有直接上司。现有一周年庆宴会,宴会邀请编号为 $i$ 的职员会增加快乐指数 $r_i$,但是一个职员和他的直接上司不会同时来参加宴会。问邀请哪些职员可以使宴会的快乐指数最大?
\subsubsection{子任务}
\par 对于所有测试数据,满足 $1 \leq n \leq 6*10^3$,$-128 \leq r_i \leq 127$,$1 \leq l, k \leq n$,且给出的关系一定是一棵树。
\subsection{题解}
\subsubsection{解法一 (100分)}
\par 该题可以抽象成一棵以校长为根的树,对于这种情况需使用树形 $dp$ 求解。假设 $dp(i, 0/1)$ 表示以 $i$ 为根节点的子树在不邀请 $/$ 邀请 $i$ 时可达到的最大快乐指数,则状态转移方程为:$$dp(i, 1) = \sum_{j = son[0]}^{sizeof(son)} dp(j, 0)$$ $$dp(i, 0) = \sum_{j = son[0]}^{sizeof(son)} max(dp(j, 0), dp(j, 1))$$
\par 同时,定义 $bool$ 数组 $vis$ 记录节点 $i$ 是否有父亲从而寻找根节点,从根节点开始 $dfs$ 即可。
\subsubsection{C++ 代码实现}
\lstinputlisting[
    style = C++,
    title = {洛谷-P1352 没有上司的舞会}
]{test04.cpp}
\end{document}