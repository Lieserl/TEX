\section{算法设计题}
    \begin{enumerate}
        \item 假设以顺序存储结构实现一个双向栈,即在一维数组的存储空间中存在着两个栈,它们的栈底分别设在数组的两个端点。试编写实现这个双向栈 \textbf{tws} 的三个操作:
            \begin{choices}
                \item 初始化栈 \textbf{InitStack(\&tws)};
                    \lstinputlisting[style = C++]{algorithm_1_1.cpp}
                \item 入栈 \textbf{push(\&tws, i, e)};
                    \lstinputlisting[style = C++]{algorithm_1_2.cpp}
                \item 出栈 \textbf{pop(\&tws, i, \&e)},其中 \textbf{i} 为 \textbf{0} 或 \textbf{1},分别指示设在数组两端的两个栈
                    \lstinputlisting[style = C++]{algorithm_1_3.cpp}
            \end{choices}
        \item 假设以带头结点的循环链表表示队列,并且只设一个指针指向队尾元素结点(注意不设头指针),试编写相应的队列初始化、入队列和出队列的算法
            \begin{choices}
                \item 循环链队初始化
                    \begin{mquote}
                        \par 初始时队列里没有元素,所以尾指针直接指向头结点
                    \end{mquote}
                    \lstinputlisting[style = C++]{algorithm_2_1.cpp}
                \item 入队
                    \begin{mquote}
                        \par 入队时将新的元素的 \textbf{next} 指向头结点,然后将尾指针(原队尾元素)的 \textbf{next} 指向新元素,最后将尾指针指向新元素
                    \end{mquote}
                    \lstinputlisting[style = C++]{algorithm_2_2.cpp}
                \item 出队
                    \begin{mquote}
                        \par 出队时先判断队列是否为空。如果为空直接返回;如果不为空,则将头结点后的结点(队头元素)删除。同时如果队列里只有一个结点,删除后队列为空,此时应将尾指针指向头指针
                    \end{mquote}
                    \lstinputlisting[style = C++]{algorithm_2_3.cpp}
            \end{choices}
        \item 回文是前后两个方向拼写完全相同的字符串。很显然,空字符串是回文,任何一个只有1个字符的字符串是回文。编写一个函数 \textbf{testPalindrome},判断一个字符串是否是回文
            \begin{mquote}
                \par 因为这两节课学的栈,所以这里用栈实现回文串的判断(\sout{感觉不如双指针})
                \par 具体实现方法如下:
                \par 将字符串的前半段压入栈中,然后依次判断字符串的后半段和栈顶元素是否一致。如果不一致,则说明该字符串不是回文串;如果一致,则出栈,继续下一次判断
                \par 如果字符串的长度为奇数,不难发现正中间元素不会影响其是否为一个回文串,因此可以省略
            \end{mquote}
            \lstinputlisting[style = C++]{algorithm_3.cpp}
    \end{enumerate}