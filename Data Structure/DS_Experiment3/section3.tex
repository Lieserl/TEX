\section{实验步骤}
    \textbf{操作步骤+运行截图}
    \subsection{代码解释}
        \par 由于本次实验要求实现的功能繁多,因此采用分文件编写的方法,下面对每一个模块进行解释说明。
        \subsubsection{LINKSTRING}  
            \par \textbf{LinkString} 用于存储每一行的字符串,由于代码文本编辑器需要频繁的进行插入和删除的功能,同时为了节省空间,这里采用不带头结点的链串进行实现。
            \begin{figure*}[htbp]
                \includegraphics*[width = 12cm]{ls_1.png}
            \end{figure*}
            \newpage
            \par 下面对每一个函数进行说明:
            \begin{enumerate}
                \item \textbf{LinkString* init\_linkstring(void)}
                    \par 在初始化时,我们在堆区分配一块内存,并将头结点设为 \textbf{NULL} (因为是无头结点的链串),并初始化 \textbf{length} 为 \textbf{0}。
                    \par 由于所有操作都是在常数时间内完成的,因此时间复杂度为 \textbf{O(1)}
                    \begin{figure*}[htbp]
                        \includegraphics*[width = 12cm]{ls_2.png}
                    \end{figure*}
                \item \textbf{void destory\_linkstring(LinkString* ls)}
                    \par 该函数实现了销毁链串的功能,我们一次遍历每一个结点,将其释放掉在移动至下一个结点,最后将链串本身释放掉。
                    \par 函数需要遍历一次链串,因此时间复杂度为 \textbf{O(n)}
                    \begin{figure*}[htbp]
                        \includegraphics*[width = 8cm]{ls_3.png}
                    \end{figure*}
                \item \textbf{bool insert\_substring(LinkString* ls, int pos, const char* substr)}
                    \par 该函数实现了在特定位置插入子串的功能,同时做了特判,在输入的 \textbf{pos} 大于链串长度时,默认从串尾进行插入操作。在函数中,我们首先迭代至起始位置,同时由于链串不带头结点,因此需要利用 \textbf{prev} 记录当前结点的前驱结点。在插入操作时,我们需要查看前驱结点 \textbf{prev} 是否为空指针。如果为空指针,说明我们需要将结点插入至队头位置。
                    \par 函数首先迭代至指定插入的结点,这部分的时间复杂度为 \textbf{O(n)},然后进行子串的插入操作,该过程需要遍历子串,故时间复杂度为 \textbf{O(m)}。因此,总的时间复杂度为 \textbf{O(n + m)}。
                    \begin{figure*}[htbp]
                        \includegraphics*[width = 10cm]{ls_4.png}
                    \end{figure*}
                \newpage
                \item \textbf{bool delete\_substring(LinkString* ls, int pos, int len)}
                    \par 该函数实现了在特定位置删除指定长度子串的功能,与插入类似,我们对 \textbf{len} 越界的情况做了特殊处理,并且利用 \textbf{prev} 储存前驱结点,防止因当前结点为队头而错误地将队头指针变为野指针。
                    \par 函数的基本实现与插入函数类似,都需要迭代至需要删除的位置并删除指定长度的子串,因此时间复杂度也为 \textbf{O(n + m)}
                    \begin{figure*}[htbp]
                        \includegraphics*[width = 10cm]{ls_5.png}
                    \end{figure*}
                \newpage
                \item \textbf{char* get\_linkstring(LinkString* ls)}
                    \par 该函数实现了遍历整个链串,并将其中元素提取出来,最后返回一个 \textbf{C} 的字符串的功能,同时使用了 \textbf{malloc} 开辟内存,防止链串长度超出 \textbf{res} 的最大长度导致截断。
                    \par 函数需要遍历整个链串,因此时间复杂度为 \textbf{O(n)}。
                    \begin{figure*}[htbp]
                        \includegraphics*[width = 9cm]{ls_6.png}
                    \end{figure*}
                \item \textbf{int find\_substring(LinkString* ls, const char* pattern)}
                    \par 该函数实现了使用 \textbf{KMP} 算法(\textbf{next} 数组初始化 \textbf{0} 的版本)完成了模式串与主串匹配的过程。
                    \par 我们知道 \textbf{KMP} 算法只需遍历一次主串就能实现查找功能,其中,\textbf{next} 数组的求解还需要遍历一次模式串,因此总的时间复杂度为 \textbf{O(n + m)}。
                    \begin{figure*}[htbp]
                        \includegraphics*[width = 9cm]{ls_7.png}
                    \end{figure*}
            \end{enumerate}
        \subsubsection{LINKLIST}
            \par 将\textbf{LinkString} 封装进 \textbf{LinkList},这样我们就实现了最基本的文本存储访问功能:通过 \textbf{LinkList} 访问指定的行,然后通过 \textbf{LinkString} 访问指定的列。
            \begin{figure*}[htbp]
                \includegraphics*[width = 12cm]{ll_1.png}
            \end{figure*}
            \par 下面对每一个函数进行说明:
            \begin{enumerate}
                \item \textbf{LinkList* init\_linklist(void)}
                    \par 该函数的实现逻辑与链串的初始化一致,再次不再赘述。
                    \par 时间复杂度 \textbf{O(1)}
                    \begin{figure*}[htbp]
                        \includegraphics*[width = 11cm]{ll_2.png}
                    \end{figure*}
                \item \textbf{LNode* get\_row(LinkList* ll, int row\_pos)}
                    \par 该函数用于通过指定的 \textbf{pos} 找到相应的结点。
                    \par 函数需要迭代至指定位置,因此时间复杂度为 \textbf{O(n)}
                    \begin{figure*}[htbp]
                        \includegraphics*[width = 11cm]{ll_4.png}
                    \end{figure*}
                \item \textbf{void destory\_linklist(LinkList* ll)}
                    \par 与 \textbf{LinkString} 类似,该函数用于销毁 \textbf{LinkList},在此不再赘述。
                    \par 时间复杂度 \textbf{O(n)}
                    \begin{figure*}[htbp]
                        \includegraphics*[width = 12cm]{ll_3.png}
                    \end{figure*}
                \item \textbf{bool insert\_linklist(LinkList* ll, int row\_pos, int col\_pos, const char* s)}
                    \par 该函数实现了在指定位置插入行的功能,同时在函数中我们调用了之前的 \textbf{insert\_substring()} 函数完成插入指定字符串的功能,防止了代码冗余的现象。
                    \par 函数需要迭代至指定位置,这部分的时间复杂度为 \textbf{O(n)},根据之前的分析,在链串中插入字符串的时间复杂度为 \textbf{O(n + m)(n = 0)} $=$ \textbf{O(m)}。因此总的时间复杂度为 \textbf{O(n + m)}。
                    \begin{figure*}[htbp]
                        \includegraphics*[width = 12cm]{ll_5.png}
                    \end{figure*}
                \newpage
                \item \textbf{bool delete\_linklist(LinkList* ll, int pos, int len)}
                    \par 该函数与插入类似,都调用了 \textbf{LinkString} 中的函数,在此不再赘述。
                    \par 时间复杂度 \textbf{O(n + m)}
                    \begin{figure*}[htbp]
                        \includegraphics*[width = 10cm]{ll_6.png}
                    \end{figure*}
            \end{enumerate}
       \subsubsection{OPRTSTACK}
            \par \textbf{OprtStack} 用于实现附加要求中的运算符匹配算法,由于之前没写过数组版本的栈,所以在这就不再写链栈了(\sout{绝对不是因为懒})。
            \begin{figure*}[htbp]
                \includegraphics*[width = 8cm]{os_1.png}
            \end{figure*}
            \begin{enumerate}
                \item \textbf{OpStack* init\_stack(void)}
                    \par 初始化时为栈开辟指定的内存空间,由于使用动态数组实现,因此初始化 \textbf{top} $=$ \textbf{-1} 代表一开始栈是空的。
                    \begin{figure*}[htbp]
                        \includegraphics*[width = 9cm]{os_2.png}
                    \end{figure*}
                \item \textbf{bool extend\_stack(OpStack* s)}
                    \par 该函数用于实现在栈满时对栈容量的扩展,保证不会因为栈的容量问题导致判断运算符失败。为了防止 \textbf{realloc} 函数开辟空间失败返回空指针导致原空间无法访问造成内存泄漏,这里创建临时变量接受新的内存地址。
                    \par 由于 \textbf{realloc} 函数并不是在原有内存空间上进行扩展,而是开辟一个新的空间,并将原有数据拷贝过去,因此时间复杂度为 \textbf{O(n)}
                    \begin{figure*}[htbp]
                        \includegraphics*[width = 12cm]{os_3.png}
                    \end{figure*}
                \newpage
                \item 其他函数的实现比较简单,这里不再赘述。
            \end{enumerate}
        \subsubsection{TEXTEDITOR}
            \par 在完成了 \textbf{LinkList} 的编写后,我们将其封装进 \textbf{TextEditor} 中,方便调用。
            \begin{figure*}[htbp]
                \includegraphics*[width = 12cm]{te_1.png}
            \end{figure*}
            \par 下面对每一个函数进行说明:
            \begin{enumerate}
                \item \textbf{TextEditor* init\_editor} 与 \textbf{void destory\_editor(TextEditor* editor, int oprt)}
                    \par 在 \textbf{init\_editor} 中,我们为其在堆区开辟了一片空间,并使用之前编写的 \textbf{init\_linklist} 初始化封装的链表。时间复杂度 \textbf{O(1)}
                    \par 在 \textbf{destory\_editor} 中,我们调用之前编写的 \textbf{destory\_linklist} 函数销毁链表,但由于在某些操作中,我们可能并不希望将 \textbf{editor} 也释放掉,所以增加了变量 \textbf{oprt},当 \textbf{oprt} $=$ \textbf{0} 时,才会执行 \textbf{free(editor)} 操作,彻底销毁整个 \textbf{TextEditor}。时间复杂度 \textbf{O(nm)}
                    \begin{figure*}[htbp]
                        \includegraphics*[width = 12cm]{te_2.png}
                    \end{figure*}
                \item \textbf{void show\_menu(void)}
                    \par 该函数用于打印可执行操作的选项。
                \item \textbf{bool read\_file(TextEditor** editor, const char* file\_name)}
                    \par 该函数用于实现读取文件的功能。由于我们可能在程序运行时从不同的文件中读入数据,因此每次读入前都需要将当前的数据释放掉,这里使用了之前编写的 \textbf{destory\_editor} 函数,同时我们希望能继续使用 \textbf{editor} 存储新的数据,因此 \textbf{oprt} $=$ \textbf{1}。
                    \par 需要注意的是,由于此处涉及到对指针分配内存,因此函数的参数应该是一个二重指针。
                    \par 由于需要将整个文本读入,因此时间复杂度为 \textbf{O(nm)}
                    \begin{figure*}[htbp]
                        \includegraphics*[width = 18cm]{te_5.png}
                    \end{figure*}
                \newpage
                \item \textbf{bool save\_file(TextEditor* editor, const char* file\_name)}
                    \par 该函数用于实现将数据写入文件的功能,我们遍历链表的每一行,同时调用 \textbf{get\_linkstring} 函数,将每一行存储的链串转换成字符串并存储到文件中。
                    \par 在函数中我们需要遍历链表的每一个结点,所以外层循环的迭代次数为 \textbf{n},根据之前的分析,每次调用 \textbf{get\_linkstring} 函数的时间复杂度为 \textbf{O(m)},因此总的时间复杂度为 \textbf{O(nm)}
                    \begin{figure*}[htbp]
                        \includegraphics*[width = 12cm]{te_6.png}
                    \end{figure*}
                \newpage
                \item \textbf{void print\_text(TextEditor* editor, int start, int end)}
                    \par 该函数用于打印指定范围的行,在打印时,会显示当前行数,方面对其进行操作。与之前类似,当输入的数据大于行数是默认打印至末尾。我们先使用 \textbf{get\_row} 函数获取起始结点,依次遍历剩下的结点,每次调用 \textbf{get\_linkstring} 函数获得字符串并打印。
                    \par 根据之前的分析,\textbf{get\_row} 函数和外层循环的时间复杂度为 \textbf{O(n)},调用 \textbf{get\_linkstring} 函数的时间复杂度为 \textbf{O(m)},总的时间复杂度为 \textbf{O(nm)}
                    \begin{figure*}[htbp]
                        \includegraphics*[width = 13cm]{te_4.png}
                    \end{figure*}
                \item \textbf{bool check\_oprt(TextEditor* editor)}
                    \par 该函数用于实现检测运算符是否匹配。具体实现原理是通过利用栈的后进先出性质,按行优先的顺序访问整个文本,将所有的 \textbf{\{, (} 运算符压入栈中,并查看之后出现的 \textbf{\}, )} 运算符是否与栈顶元素匹配。在实现基本功能的同时,通过一定的算法将运算符是否在头文件中、\textbf{comment} 中、 \textbf{`'}中的情况过滤。
                    \par 由于需要遍历一遍文本,因此时间复杂度为 \textbf{O(nm)}
                    \begin{figure*}[htbp]
                        \includegraphics*[width = 12cm]{te_7.png}
                    \end{figure*}
            \end{enumerate}
    \newpage
    \subsection{运行截图}
        \begin{enumerate}
            \item \textbf{运行程序}
                \par 运行程序,会先给出可用操作,并提示操作。
                \begin{figure*}[htbp]
                    \centering
                    \includegraphics*[width = 10cm]{op_1.png}
                \end{figure*}
            \item \textbf{操作一:读入文件}
                \par 输入 \textbf{1},程序提示输入文件名,如果文件不存在,则会提示失败;相反,会提示成功
                \begin{figure*}[htbp]
                    \centering
                    \includegraphics*[width = 7cm]{op_2.png}
                    \includegraphics*[width = 7.5cm]{op_3.png}
                \end{figure*}
            \item \textbf{操作二:输出指定范围的行}
                \par 输入 \textbf{2},程序提示输入范围,在越界时会默认输出至末尾
                \begin{figure*}[htbp]
                    \centering
                    \includegraphics*[width = 7cm]{op_4.png}
                    \includegraphics*[width = 7cm]{op_5.png}
                \end{figure*}
            \item \textbf{操作三:行插入}
                \par 输入 \textbf{3},程序提示输入指定行,在越界时默认在末尾插入
                \begin{figure*}[htbp]
                    \centering
                    \includegraphics*[width = 10cm]{op_6.png}
                \end{figure*}
                \newpage
                \par 查看插入后的效果
                \begin{figure*}[htbp]
                    \centering
                    \includegraphics*[width = 10cm]{op_7.png}
                \end{figure*}
            \item \textbf{操作四:行删除}
                \par 输入 \textbf{4},程序提示输入指定行和长度,在越界时默认从起始位置开始全部删除
                \begin{figure*}[htbp]
                    \centering
                    \includegraphics*[width = 12cm]{op_8.png}
                \end{figure*}
                \newpage
                \par 查看删除后的效果
                \begin{figure*}[htbp]
                    \centering
                    \includegraphics*[width = 12cm]{op_9.png}
                \end{figure*}
            \item \textbf{操作五:行内文本插入}
                \par 输入 \textbf{5},程序提示输入指定行和列,在越界时默认在末尾插入
                \begin{figure*}[htbp]
                    \centering
                    \includegraphics*[width = 12cm]{op_10.png}
                \end{figure*}
                \par 查看插入后的效果
                \begin{figure*}[htbp]
                    \centering
                    \includegraphics*[width = 11cm]{op_11.png}
                \end{figure*}
            \newpage
            \item \textbf{操作六:行内文本删除}
                \par 输入 \textbf{6},程序提示输入指定行和列,以及长度,在越界时默认从起始位置开始全部删除
                \begin{figure*}[htbp]
                    \centering
                    \includegraphics*[width = 12cm]{op_12.png}
                \end{figure*}
                \par 查看插入后的效果
                \begin{figure*}[htbp]
                    \centering
                    \includegraphics*[width = 10cm]{op_13.png}
                \end{figure*}
            \item \textbf{操作七:文本查找}
                \par 输入 \textbf{7},程序提示输入指定的行和想查找的字符串,如果成功,则返回字符串的起始位置(下标从 \textbf{0} 开始)
                \begin{figure*}[htbp]
                    \centering
                    \includegraphics*[width = 11cm]{op_14.png}
                \end{figure*}
            \newpage
            \item \textbf{操作八:检查运算符匹配}
                \par 输入 \textbf{8},程序自动检查运算符是否匹配并输出
                \begin{figure*}[htbp]
                    \centering
                    \includegraphics*[width = 7cm]{op_15.png}
                    \includegraphics*[width = 7cm]{op_17.png}
                \end{figure*}
                \begin{figure*}[htbp]
                    \centering
                    \includegraphics*[width = 6cm]{op_13.png}
                    \includegraphics*[width = 8cm]{op_16.png}
                \end{figure*}
                \par 包含所有特殊情况的例子如下
                \begin{figure*}[htbp]
                    \centering
                    \includegraphics*[width = 8cm]{op_19_1.png}
                \end{figure*}
            \item \textbf{操作九:保存文件}
                \par 输入 \textbf{9},程序保存文件,结束运行
                \begin{figure*}[htbp]
                    \centering
                    \includegraphics*[width = 12cm]{op_18.png}
                \end{figure*}
                \par 重新运行程序,将之前保存的文件读入并打印
                \begin{figure*}[htbp]
                    \centering
                    \includegraphics*[width = 10cm]{op_19.png}
                \end{figure*}
        \end{enumerate}
         