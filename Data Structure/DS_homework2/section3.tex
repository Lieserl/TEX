\section{算法设计题}
\begin{enumerate}
    \item 从一个给定的顺序表 \textbf{L} 中删除值在 \textbf{[x, y]} $(x \leq y)$ 之间的所有元素,要求以较高的效率来实现,并分析你的算法时间复杂度。(提示:移动位置一步到位)
        \begin{mquote}
            \begin{enumerate}
                \item 定义两个指针 \textbf{\textit{p、q}},\textbf{\textit{p}}指针用来遍历原顺序表,\textbf{\textit{q}} 指针指向新顺序表的末尾,\textbf{\textit{p}} 的每一次迭代中,我们检查 \textbf{elem[p]} 的值是否合法,如果合法,就将其加入新顺序表,并令 \textbf{\textit{q}} 后移一位。操作完成后,新顺序表中存放的即是结果,同时,由于 \textbf{\textit{q}} 指针一值指向表尾后一位,可以用来更新表长
                \item 注意到每一次操作后,两指针的位置关系有 \textbf{\textit{q}} $\leq$ \textbf{\textit{p}},所以我们可以直接在原顺序表上进行操作,同时保证不会意外将合法元素移除。
            \end{enumerate}
            \par 由于只需要一次遍历即可求解出答案,因此时间复杂度为:\textbf{\textit{O(n)}}    
        \end{mquote}
        \lstinputlisting[
            style = C++,
        ]{algorithm_1.cpp}
    \item 设计一个空间复杂度为 \textbf{\textit{0(1)}} 的算法 \textbf{shift(SqList L, int k)} 将顺序表 \textbf{L} 中的元素整体循环左移 \textbf{k} 位,要求以较高的效率来实现,并分析在如下示例的 \textbf{10} 个数据,循环左移 \textbf{4} 位时这 \textbf{10} 个数据总共移动的次数。
        \par 示例数据:
        \par 顺序表 \textbf{L} 初始数据为:\textbf{1, 2, 3, 4, 5, 6, 7, 8, 9, 10}
        \par 整体循环左移 \textbf{4} 位后为:\textbf{5, 6, 7, 8, 9, 10, 1, 2, 3, 4}
        \begin{mquote}
            \par 当我们将最终得到的序列反转后,不难发现其为 \textbf{(1, k)} 和 \textbf{(k + 1, length)} 区间各反转一次的结果,以下给出证明:
            \par 设元素 \textbf{\textit{e}} 的位置为 \textbf{\textit{x}} $(1 \leq x \leq length)$, 由题意可知,最终其位置应为 \textbf{(x - k) \% length}
            \par 可以分为  \textbf{\textit{x}} $\leq$ \textbf{\textit{k}} 和 \textbf{\textit{x}} $>$ \textbf{\textit{k}} 两种情况讨论
            \begin{itemize}
                \item \textbf{\textit{x}} $\leq$ \textbf{\textit{k}} 时,第一次反转后其位置为 $k - x + 1$,整体反转后其位置为 $length - (k - x + 1) + 1 = length - k + x = (x - k) \ \% \ length$;
                \item \textbf{\textit{x}} $>$ \textbf{\textit{k}} 时,第一次反转后其位置为 $length - x + 1$,整体反转后其位置为 $length - (length - x + 1) + 1 = x - k = (x - k) \ \% \ length$
            \end{itemize}
        \end{mquote}
        \lstinputlisting[
            style = C++,
        ]{algorithm_2.cpp}
        \begin{mquote}
            \par 下面分析图示数据所需移动次数
            \par 我们首先反转了前 \textbf{4} 个元素,每个元素移动 \textbf{1} 次,共移动 \textbf{4} 次;再反转后 \textbf{6} 个元素,共移动 \textbf{6} 次;最后我们反转整个序列,共移动 \textbf{10} 次
            \par 综上,我们一共移动了 \textbf{20} 次
        \end{mquote}
    \item 设计一个算法删除一个单链表倒数第 \textbf{k} 个结点
        \begin{mquote}
            \par 定义两个指针 \textbf{\textit{p}} 和 \textbf{\textit{q}},首先先让 \textbf{\textit{p}} 指针走到第 \textbf{\textit{k}} 个结点,然后再让 \textbf{\textit{p}} 指针和 \textbf{\textit{q}} 指针一起走到 \textbf{\textit{p}} 指针到达表尾时,此时 \textbf{\textit{q}} 指针的位置刚好在倒数第 $k + 1$ 个结点的位置
        \end{mquote}
        \lstinputlisting[
            style = C++,
        ]{algorithm_3.cpp}
    \item 设计一个算法检测一个单链表 \textbf{L} 上是否因为某种错误操作而产生了环
        \begin{mquote}
            \par 判断成环的一个经典算法是 \textbf{\textit{Floyd判圈算法}},定义一个快指针和一个慢指针,慢指针每次移动一位,快指针每次移动两位,如果链表成环,快指针一定会在某一时刻与慢指针指向同一结点
        \end{mquote}
        \lstinputlisting{algorithm_4.cpp}
\end{enumerate}