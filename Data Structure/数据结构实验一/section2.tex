\section{实验内容}
    \subsection{实验内容一:结构体及文件基本操作}
        \textbf{要求:}
        \begin{itemize}
            \item \textbf{基本要求:}定义一个结构数组,用于保存学生信息,数据项包括:学号、姓名、年龄;
            \item \textbf{功能项1:}从键盘输入学生信息存入结构数组,可按照学号依次输入从本人开始的 \textbf{\textit{5}} 名同学信息;
            \item \textbf{功能项2:}输入命令可将结构数组中的数据将保存在 \textbf{\textit{input.dat}} 文件中(在文件中以二进制或文本存储信息均可);
            \item \textbf{功能项3:} 输入命令可将信息从 \textbf{\textit{input.dat}} 文件中读入到结构数字,然后反向顺序输出到 \textbf{\textit{output.dat}} 文件中(在文件中以二进制或文本存储信息均可);
            \item \textbf{附加要求:}设置条件断点,在执行到处理第 \textbf{\textit{5}} 位学生信息时中断。
        \end{itemize}
    \subsection{实验内容二:算法执行效率测量与分析}
        \textbf{要求:}
        \begin{itemize}
            \item 编写程序分别调用下述两个函数 \textbf{\textit{copyij}} 和 \textbf{\textit{copyji}};
            \item 统计两个子程序的运行时间(绝对时间);
            \item 比较两个子程序运行绝对时间上的差异,试分析成因;
            \item 给出两个算法的时间复杂度,说明差异;
            \item \textbf{附加要求:}编译调试时采用优化和非优化方式进行分别进行编译测试,观察启用编译优化后是否能带来性能变化。
        \end{itemize}
        \begin{figure*}[htbp]
            \centering
            \includegraphics*[width = 14cm]{program1.jpg}
        \end{figure*}