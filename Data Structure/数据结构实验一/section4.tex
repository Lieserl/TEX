\section{实验分析和总结}
    \subsection{实验时的工作思路、设想、效果等综合分析}
        \begin{enumerate}
            \item 
                \par 实验的基本要求为定义一个结构数组用来保存学生信息,可以使用 \textbf{\textit{struct}} 自定义数据类型实现。
                \par 实验的三个功能项要求我们能够实现文件读写和保留历史数据,为了防止新写入的信息覆盖之前写入的数据,我运用课上学习的知识,将其封装进线性表,并用 \textbf{\textit{index}} 记录表长,这样在写入新数据时可以直接从下标为表长的地方开始写入,避免数据覆盖的风险。
                \par 在实现三个要求函数之前,我先实现了 \textbf{\textit{init()}} 函数,用于初始化线性表并判断是否有历史数据。如果有,则读入数据并存储在线性表中;如果没有,则初始化表长为 \textbf{\textit{0}}。
                \par 第一个功能项的实现,由于学号递增,只需输入第一位学生的学号,可以通过 \textbf{\textit{index}} 向线性表中添加元素。
                \par 第二个功能项的实现,以二进制方式写入文件,如果文件无法打开,则抛出异常,否则将表长和学生结构体数组存入 \textbf{\textit{input.dat}} 中。至于为什么先写入表长,在前面已有详细解释,在此不再赘述。
                \par 第三个功能项的实现,由于我们需要将一个文件里的数据逆序写入到另一个文件中,所以需要创建一个临时的线性表接受数据,再将其逆序写入新的文件中。
            \item 
                \par 在定义两个数组时,由于栈区内存限制,这里采用定义全局变量的方式实现。
                \par 由于需要比较两个函数的运行时间差异,我想到可以使用 \textbf{time.h} 中的 \textbf{\textit{clock()}} 函数,分别记录下两个函数开始和结束的时间点,再相减得出两个函数的运行时间。
        \end{enumerate}
    \subsection{总结心得}
        \par 在实现实验一的各项要求时,我发现虽然得益于上学期在算法方面的学习,在大体框架和实现方法上并没有遇到太多阻碍,但算法上不常接触的文件读写,内存分配方面的知识却基本忘光了,通过这次实验,我系统地复习了文件读写,内存分配的基本操作,最终完成了实验。
        \par 虽然之前就了解过行优先遍历会比列优先遍历的效率更高,但却没有深入了解其原理,通过这次实验,我大体上了解了缓存等相关知识,同时也体验到了优化编译对程序运行效率带来的提升。