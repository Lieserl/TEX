\section{双端口存储器实验}
    \subsection{实验目的及任务}
        \begin{itemize}
            \item 实验目的:
            \begin{enumerate}
                \item 进一步熟悉 \textit{TEC-Plus} 模型计算机的数据通路
                \item 熟练掌握数据通路中各种控制信号的作用和用法
                \item 掌握数据通路中数据流动的路径
            \end{enumerate}
            \item 实验任务:
            \begin{enumerate}
                \item 向通用寄存器堆内的R3-R0写入数据
                \item 将寄存器 R0-R3 中的数据写入双端口 \textit{RAM} 的 20H、21H、22H、23H 存储单元
                \item 从存储器 20H、21H、22H、23H 存储单元中读出数据,并存入寄存器 R3-R0
                \item 显示寄存器 R3-R0 的值,检查数据传送是否正确
            \end{enumerate}
        \end{itemize}
    
    \subsection{实验电路分析}
        \begin{figure}[htbp]
            \centering
            \includegraphics*[width=12cm]{3_cu.png}
        \end{figure}

        \par 一次完整的运算步骤的电路分析如下:
        \begin{enumerate}
            \item 通过数据开关 \textit{SD}7-\textit{SD}0 将两位 $16$ 进制数输入到 \textit{SWD} 中,当 $SBUS = 1$ 时,数据被送入 \textit{DBUS} 总线上
            \item 通过 2-4 译码器片选输入 $RD0$、$RD1$ 进行寄存器的选择
            \item 令 $DRW = 1$,此时被选中的寄存器可以写入
            \item 按下 $QD$ 输入时钟信号,在 $T3$ 时钟上升沿 \textit{DBUS} 中的数据被写入寄存器中
            \item 关闭所有开关,令 $SBUS = 0, LAR = 1$,此时地址寄存器可以写入
            \item \textit{QD},在 $T3$ 时钟上升沿 \textit{DBUS} 上的数据被送入地址寄存器 \textit{AR} 中
            \item 此时 \textit{AR} 中的数据被送至 A7L-A0L 端口,选中对应的内存地址,令 $LAR=0$,关闭地址寄存器写入,令 $MEMW=1$,开启双端口 RAM 的写入功能
            \item 令$S3-S0 = F, M = 1, CIN = 0, SBUS = 0, ABUS = 1, MBUS = 0$,通过 RD1 和 RD0 选中寄存器,此时 ALU 将(A)端口的数据直接输出至 DBUS 上
            \item \textit{QD},在 $T2$ 时钟上升沿 \textit{DBUS} 上的数据被送入双端口 \textit{RAM} 中由 \textit{AR} 寄存器指定的地址中
            \item 令 $MEMW=0, SBUS=0, MBUS=1, ABUS=0$,此时由 $AR$ 选中的内存地址中的数据被送到 \textit{DBUS} 上
            \item 重复步骤 2-4 将内存中的数据写入寄存器中
            \item 重复步骤 8 将内存中的数据读出至 DBUS 上
        \end{enumerate}

        \subsection{思考题解答}
        \begin{problem}
            同步从 RAM 的右端口读出数据应该如何操作,信号如何设置,PC7-PC0、INS7- INS0 显示情况如何?
        \end{problem}
        \begin{solution}
            \begin{enumerate}
                \item SBUS=1,LPC=1,数据开关为 20H,QD,将 20H 存入 PC,此时 PC=20H,INS=75H
                \item PCINC=1,QD,PC自增,此时 PC=21H,INS=28H
                \item 重复步骤 2,此时 PC=22H,INS=89H
                \item 重复步骤 2,此时 PC=23H,INS=32H
            \end{enumerate}
        \end{solution}
    
    
    \subsection{实验过程及结果}
    \begin{table}[htbp]
        \centering
        \scalebox{0.5}{
        \begin{tabular}{|c|>{\centering\arraybackslash}p{4cm}|c|>{\centering\arraybackslash}p{6cm}|>{\centering\arraybackslash}p{3cm}|c|}
            \hline
            \multicolumn{6}{|c|}{\textbf{实验过程记录表}} \\ \hline
            \textbf{序号} & \textbf{操作} & \textbf{数据开关} & \textbf{操作目的} & \textbf{实验现象} & \textbf{备注} \\ \hline
            \textbf{1} & CLR & & 复位 & & \\ \hline
            \textbf{2} & DP=1 & & 设置操作模式 & & \\ \hline
            \textbf{3} & SBUS=1, RD=00, DRW=1 & 75H & R0 存数 & A=10H D=75H & \\ \hline
            \textbf{4} & SBUS=1, RD=01, DRW=1 & 28H & R1 存数 & A=28H D=28H & \\ \hline
            \textbf{5} & SBUS=1, RD=10, DRW=1 & 89H & R2 存数 & A=89H D=89H & \\ \hline
            \textbf{6} & SBUS=1, RD=11, DRW=1 & 32H & R3 存数 & A=32H D=32H & \\ \hline
            \textbf{6} & SBUS = 1 LAR = 1 QD & 20H & AR 写入地址 & AR=20H D=20H & \\ \hline
            \textbf{7} & S3-S0=F, M=1, CIN=0, SBUS=0, ABUS=1, MEMW=1, ARINC=1, QD=00, QD & & R0 写入地址 20H,AR自增 & D=75H A=75H AR=21H & \\ \hline
            \textbf{8} & S3-S0=F, M=1, CIN=0, SBUS=0, ABUS=1, MEMW=1, ARINC=1, QD=01, QD & & R1 写入地址 21H,AR自增 & D=28H A=28H AR=22H & \\ \hline
            \textbf{9} & S3-S0=F, M=1, CIN=0, SBUS=0, ABUS=1, MEMW=1, ARINC=1, QD=10, QD & & R2 写入地址 22H,AR自增 & D=89H A=89H AR=23H & \\ \hline
            \textbf{10} & S3-S0=F, M=1, CIN=0, SBUS=0, ABUS=1, MEMW=1, ARINC=1, QD=11, QD & & R0 写入地址 23H,AR自增 & D=32H A=32H AR=24H & \\ \hline
            \textbf{11} & 全关,LAR=1, SBUS=1 & 20H & AR 写入地址 & AR=20H D=20H & \\ \hline
            \textbf{12} & RD=11, MBUS=1, DRW=1 SBUS=0, LAR=0, ARINC=1 & & 将地址 20H 数据写入 R3,AR自增 & D=75H A=75H AR=21H & \\ \hline
            \textbf{13} & RD=10, MBUS=1, DRW=1 SBUS=0, LAR=0, ARINC=1 & & 将地址 21H 数据写入 R2,AR自增 & D=28H A=28H AR=22H & \\ \hline
            \textbf{14} & RD=01, MBUS=1, DRW=1 SBUS=0, LAR=0, ARINC=1 & & 将地址 22H 数据写入 R1,AR自增 & D=89H A=89H AR=23H & \\ \hline
            \textbf{15} & RD=00, MBUS=1, DRW=1 SBUS=0, LAR=0, ARINC=1 & & 将地址 23H 数据写入 R0,AR自增 & D=32H A=32H AR=24H & \\ \hline
            \textbf{16} & RD=00 & & 读取 R0 中的数据 & A=32H & \\ \hline
            \textbf{17} & RD=01 & & 读取 R1 中的数据 & A=89H & \\ \hline
            \textbf{18} & RD=10 & & 读取 R2 中的数据 & A=28H & \\ \hline
            \textbf{19} & RD=11 & & 读取 R3 中的数据 & A=75H & \\ \hline
        \end{tabular}
        } 
    \end{table}

    \subsection{实验收获及体会}
        \par 熟悉了 \textit{TEC-Plus} 模型计算机的数据通路,掌握了数据通路中各种控制信号的作用和用法,掌握了数据通路中数据流动的路径

    