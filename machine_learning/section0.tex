\section{实验任务背景介绍}
    \fontsize{10.5pt}{12.6pt}\selectfont
    \songti
    \par 基数估计是数据库中一个重要的模块。对于输入的查询,基数估计模块将快速估计其满足不同执行顺序条件下的记录行数,其中记录行数也称为基数。通过对不同执行顺序的基数估计,可以选择出最优的执行顺序,也称为执行计划。基数估计作为数据库关系系统查询优化器的基础和核心,对于查询性能的优化和整体数据库系统的效率至关重要,近年来伴随人工智能技术的发展,其在数据处理,提取数据之间的关系等方面展示出了更优越的性能。
    \par 传统的基数估计技术一般采用像直方图等数据结构的统计方法来拟合数据表上的数据分布\cite{ref1}。在当前大数据时代下,面对不断膨胀的数据信息、复杂多样的应用场景、异构的硬件架构和参差不齐的用户使用水平,传统方法可能很难适应这些新的场景和变化。而基于机器学习的数据基数估技术因其较强的学习能力,逐渐在数据库领域展现出了潜力和应用前景\cite{ref2},具有很重要的现实意义。
    \par 本文的主要贡献包括以下 $4$ 个方面:
    \par 1) 针对基于机器学习的数据库基数估计任务,我们应用了 \textbf{MLP} 多层感知机模型。通过对输入的查询特征进行学习,模型能够有效预测查询结果的基数,为数据库查询优化提供一种解决思路;
    \par 2) 我们通过将查询条件抽象为统一的特征表示,实现了对查询条件的灵活建模,使模型能够处理多种类型的查询,提高了模型的泛化性;
    \par 3) 引入了三种启发式方法 \textbf{attribute value independence, AVI}、\textbf{exponential backOff, EBO}、\textbf{minimum selectivity, Min-Sel} 分别计算得到总的选择率,使用这些指标来辅助模型从而得到更准确的选择率,从而提高模型在不充分数据下的学习能力。
    \par 4) 通过对 \textbf{IMDB} 数据集的实验验证,我们的 \textbf{MLP} 模型在基数估计任务上取得了显著的性能优越性,有效提高了数据库基数估计任务的准确度。

