\section{实验内容和实验环境描述}
    \subsection{实验任务内容和目的}
        \begin{itemize}
            \item 利用所学数据链路层原理,自己设计一个滑动窗口协议,在仿真环境下编程实现有噪音信道环境下两站点之间无差错双工通信
            \item 实现有噪音信道环境下的无差错传输,充分利用传输信道的带宽
            \item 实现搭载 ACK 技术的 GBN 和 选择重传协议
            \item 实现 ACK 计时器、捎带确认、NAK 等补充功能
            \item 根据信道实际情况合理地为协议配置工作参数,包括滑动窗口的大小和重传定时器时限以及ACK搭载定时器的时限
            \item 进一步巩固和深刻理解数据链路层误码检测的 CRC 校验技术,以及滑动窗口的工作机理。
        \end{itemize}

    \subsection{实验环境}
        \par 本次实验在 WINDOWS 11 下进行,使用 CLion 作为 IDE,gcc 作为编译工具,Ninja 作为生成器
        \begin{itemize}
            \item 系统版本:WINDOWS 11
            \item 编译器:gcc 8.1.0
            \item 生成器:Ninja
            \item IDE:CLion
        \end{itemize}

