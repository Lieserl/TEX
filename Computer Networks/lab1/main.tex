\documentclass[UTF8, 12pt, a4paper, oneside]{ctexart}

\usepackage{amsmath}
\usepackage{amsfonts}
\usepackage{geometry}
\usepackage{ulem}
\usepackage[most]{tcolorbox}
\usepackage[hidelinks]{hyperref}
\usepackage{color}
\usepackage{xcolor}
\usepackage{graphicx}
\usepackage{framed}
\usepackage{mathtools,amssymb}
\usepackage{bm}
\usepackage{multirow}
\usepackage{array}

\usepackage{listings}

\definecolor{backcolour}{rgb}{0.18,0.18,0.18}
\definecolor{codegray}{rgb}{0.93,0.93,0.93}
\definecolor{codegreen}{rgb}{0.34,0.76,0.34}
\definecolor{codeblue}{rgb}{0.27,0.35,0.69}
\definecolor{codepurple}{rgb}{0.69,0.25,0.82}
\definecolor{codeorange}{rgb}{0.8,0.6,0.2}
\definecolor{codered}{rgb}{0.8,0.25,0.33}

\lstset{
    basicstyle          =   \sffamily,          % 基本代码风格
    keywordstyle        =   \bfseries,          % 关键字风格
    commentstyle        =   \rmfamily\itshape,  % 注释的风格,斜体
    stringstyle         =   \ttfamily,  % 字符串风格
    flexiblecolumns,                % 别问为什么,加上这个
    numbers             =   left,   % 行号的位置在左边
    showspaces          =   false,  % 是否显示空格,显示了有点乱,所以不现实了
    numberstyle         =   \zihao{-5}\ttfamily,    % 行号的样式,小五号,tt等宽字体
    showstringspaces    =   false,
    captionpos          =   t,      % 这段代码的名字所呈现的位置,t指的是top上面
    aboveskip           =   1em,
    belowskip           =   1em,
    frame               =   lrtb,   % 显示边框
}

\lstdefinestyle{C++}{
    language        =   C++, % 语言选Python
    basicstyle      =   \zihao{-5}\ttfamily,
    numberstyle     =   \zihao{-5}\ttfamily,
    keywordstyle    =   \color{blue},
    keywordstyle    =   [2] \color{teal},
    stringstyle     =   \color{magenta},
    commentstyle    =   \color{red}\ttfamily,
    breaklines      =   true,   % 自动换行,建议不要写太长的行
    columns         =   fixed,  % 如果不加这一句,字间距就不固定,很丑,必须加
    basewidth       =   0.5em,
}

\lstdefinestyle{C}{
    language        =   C, % 语言选Python
    basicstyle      =   \zihao{-5}\ttfamily,
    numberstyle     =   \zihao{-5}\ttfamily,
    keywordstyle    =   \color{codepurple},
    keywordstyle    =   [2] \color{codegray},
    stringstyle     =   \color{magenta},
}

\usepackage{paralist}
\let\itemize\compactitem
\let\enditemize\endcompactitem
\let\enumerate\compactenum
\let\endenumerate\endcompactenum
\let\description\compactdesc
\let\enddescription\endcompactdesc

\definecolor{shadecolor}{RGB}{241, 241, 255}
\geometry{left=2.54cm,right=2.54cm,top=3.18cm,bottom=3.18cm}
\linespread{1.5}

\newcounter{problemname}[subsubsection]
%\newenvironment{problem}{\begin{shaded}\stepcounter{problemname}\par\noindent\textbf{题目\arabic{problemname}. }}{\end{shaded}\par}
\newenvironment{problem}{\begin{shaded}\stepcounter{problemname}\par\noindent\textbf{题目. }}{\end{shaded}\par}

%\newenvironment{problem}{\begin{shaded}\par\noindent\textbf{题目. }}{\end{shaded}\par}
\newenvironment{solution}{\par\noindent\textbf{解答. }}{\par}
\newenvironment{note}{\par\noindent\textbf{题目\arabic{problemname}的注记. }}{\par}

\begin{document}
    \sloppy
    \thispagestyle{empty}
    \begin{figure}[t]
        \centering
        \includegraphics[width=13cm]{logo1.jpg}
    \end{figure}

    \begin{center}
        \Huge\textbf{实验报告}
    \end{center}
    \vspace*{1em}
    \begin{figure}[htbp]
        \centering
        \includegraphics[width=3cm]{logo2.jpg}
    \end{figure}
        \begin{center}
            \huge\textbf{课程名称} \underline{\textbf{计算机组成原理}} \\
            \huge\textbf{实验名称} \underline{\textbf{运算器、双端口储存器、数据通路联合实验}} 
        \end{center}
    \vspace*{5em}
    \begin{table}[htbp]
        \centering
        \large
        \begin{tabular}{ll}
        \textbf{班级:} & \underline{2022211305} \\
        \textbf{学号:} & \underline{2022212408} \\
        \textbf{姓名:} & \underline{胡宇杭} \\
        \textbf{学院:} & \underline{计算机学院(国家示范性软件学院)} \\
        \textbf{时间:} & \underline{\today} \\
        \end{tabular}
    \end{table}
    \newpage
    \tableofcontents
    \newpage

    \section{实验内容和实验环境描述}
    \subsection{实验任务内容和目的}
        \begin{itemize}
            \item 利用所学数据链路层原理,自己设计一个滑动窗口协议,在仿真环境下编程实现有噪音信道环境下两站点之间无差错双工通信
            \item 实现有噪音信道环境下的无差错传输,充分利用传输信道的带宽
            \item 实现搭载 ACK 技术的 GBN 和 选择重传协议
            \item 实现 ACK 计时器、捎带确认、NAK 等补充功能
            \item 根据信道实际情况合理地为协议配置工作参数,包括滑动窗口的大小和重传定时器时限以及ACK搭载定时器的时限
            \item 进一步巩固和深刻理解数据链路层误码检测的 CRC 校验技术,以及滑动窗口的工作机理。
        \end{itemize}

    \subsection{实验环境}
        \par 本次实验在 WINDOWS 11 下进行,使用 CLion 作为 IDE,gcc 作为编译工具,Ninja 作为生成器
        \begin{itemize}
            \item 系统版本:WINDOWS 11
            \item 编译器:gcc 8.1.0
            \item 生成器:Ninja
            \item IDE:CLion
        \end{itemize}


    \section{相关技术}
    \fontsize{10.5pt}{12.6pt}\selectfont
    \songti
    \par 基数估计定义:给定 \textbf{SQL} 查询语句,数据库 $D$,其表示在数据库 $D$ 中执行查询 $q$ 返回的结果行数,记为 $C(q|D)=|D'|$,其中 $D'$表示查询结果。
    \par 数据库优化查询系统中,基数估计发挥了重要作用。基数估计技术根据是否使用机器学习算法分为了传统的基数估计技术和基于机器学习的基数估计技术两类。传统的基数估计一般采用统计的方法,其核心是使用某种数据结构(例如直方图,数据画像)来拟合表上的数据分布。基于直方图的算法\cite{ref3}根据边界 $[b_1, b_2, \dots, b_n]$ 将列数据划分成若干份,并且统计如下信息,一个是该属性位于 $b_{i-1}$ 和 $b_i$ 之间的元素行数,另一个是位于此范围不同的元素行数,这种直方图间隔也被称作是分桶。其中J.Xu\cite{ref4} 等人利用直方图的方式对表中的数据分布进行查询,此方法会导致基数估计偏低,任意数据分布可以通过哈希函数得到一个均匀分布。
    \par 基于数据画像的统计方法,它的核心思想是使用位图来记录元素的出现情况,从而在降低计算成本的同时提供对不同元素数量的估计。这种方法适用于大规模数据集,其中直接计算精确的基数可能变得非常昂贵。与传统的基数估计算法相比,基于数据画像的方法通常更节省内存,因为它们不需要存储实际的元素,而是使用位图记录元素的出现情况。然而这种方法是一种概率性估计,结果可能受到哈希冲突等因素的影响\cite{ref2}。
    \par 基于统计的基数估计技术适用于拟合单列的数据分布,而在处理任意多列组合上数据之间的复杂关系,其能力较弱\cite{ref1}。
    \par 还有一类基于线性映射的基数估计方法,包括线性计数\cite{ref6}和布隆过滤\cite{ref7}.基本思想是使用线性哈希函数将数据均匀映射到位图上,根据位图上每个位置被访问到的次数,利用极大似然对基数进行估计。后者为了在有限的空间中减少哈希结果碰撞,使用了多个独立的哈希函数,每个元素可以被映射到固定数量的位置上,布隆过滤的方法已经被广泛使用\cite{ref2}。
    \par 采样是一种能够替代基于统计法的基数估计方法.它不依赖于特定假设,能够发现一些偶然的关联从而获得更加准确的估计\cite{ref2}。基于采样的基数估计技术在大规模数据表的复杂查询场景中需要消耗大量空间存储采样的元组,同时有效采样元组会随着多表复杂的连接而减少,损失估计的性能。
    \par 综上所述,传统的基数估计技术计算通常需要存储直方图,位图,采样的元组等信息,这会占用较大的存储空间,并且可能难以适应数据的动态变化。而基于机器学习的基数估计技术则是利用机器学习或深度学习的方法来学习数据的分布和查询的特征,从而预测基数,需要存储学习映射函数 $f(\cdot)$ 的模型\cite{ref1}。相较于传统基数估计基数,其模型占用空间小,这种技术的优点是可以更好地拟合数据的复杂分布和查询的复杂关系,从而提高基数估计的准确度。本文采取的基于机器学习的数据库基数估计,成功引入了多层感知机模型,可进一步提升基数估计精度,减少空间占用,加强拟合复杂数据关系能力。
    \vfill
    \section{实验内容}
    \subsection{实验内容一:结构体及文件基本操作}
        \textbf{要求:}
        \begin{itemize}
            \item \textbf{基本要求:}定义一个结构数组,用于保存学生信息,数据项包括:学号、姓名、年龄;
            \item \textbf{功能项1:}从键盘输入学生信息存入结构数组,可按照学号依次输入从本人开始的 \textbf{\textit{5}} 名同学信息;
            \item \textbf{功能项2:}输入命令可将结构数组中的数据将保存在 \textbf{\textit{input.dat}} 文件中(在文件中以二进制或文本存储信息均可);
            \item \textbf{功能项3:} 输入命令可将信息从 \textbf{\textit{input.dat}} 文件中读入到结构数字,然后反向顺序输出到 \textbf{\textit{output.dat}} 文件中(在文件中以二进制或文本存储信息均可);
            \item \textbf{附加要求:}设置条件断点,在执行到处理第 \textbf{\textit{5}} 位学生信息时中断。
        \end{itemize}
    \subsection{实验内容二:算法执行效率测量与分析}
        \textbf{要求:}
        \begin{itemize}
            \item 编写程序分别调用下述两个函数 \textbf{\textit{copyij}} 和 \textbf{\textit{copyji}};
            \item 统计两个子程序的运行时间(绝对时间);
            \item 比较两个子程序运行绝对时间上的差异,试分析成因;
            \item 给出两个算法的时间复杂度,说明差异;
            \item \textbf{附加要求:}编译调试时采用优化和非优化方式进行分别进行编译测试,观察启用编译优化后是否能带来性能变化。
        \end{itemize}
        \begin{figure*}[htbp]
            \centering
            \includegraphics*[width = 14cm]{program1.jpg}
        \end{figure*}

\end{document}