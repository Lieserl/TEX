\section{总结体会}
    \begin{itemize}
        \item 体会:通过这次实验,我详细了解了 \textbf{CI} 和 \textbf{ROP} 两种针对缓冲区溢出的攻击手段;对堆栈和程序的运行过程有更深的理解;同时知道了函数调用的机制和栈帧的结构;也明白明白了如何编写更加安全的程序。
        \item 建议:不可否认这次的实验同样十分有趣,但最后 \textbf{Phase 5} 时在那盯着眼找一堆指令确实有些枯燥乏味,希望可以就此进行改进。
        \item 写在最后:速速把下次实验搬上桌来。
    \end{itemize} 

\section{Phase 5 小工具}
    \begin{itemize}
        \item movq \%rsp, \%rax
        \par 48 89 e0 对应 movq \%rsp, \%rax
        \par 90 对应 nop
        \par c3 对应 retq
        \begin{figure*}[htbp]
            \centering
            \includegraphics*[width = 10cm]{rsp_rax.png}
        \end{figure*}
        \item movq \%rax, \%rdi
        \par 48 89 c7 对应 movq \%rax, \%rdi
        \par 90 对应 nop
        \par c3 对应 retq
        \begin{figure*}[htbp]
            \centering
            \includegraphics*[width = 10cm]{rax_rdi.png}
        \end{figure*}
        \item popq \%rax (0x48)
        \par 58 对应 popq \%rax
        \par 90 对应 nop
        \par c3 对应 retq
        \begin{figure*}[htbp]
            \centering
            \includegraphics*[width = 10cm]{pop_rax.png}
        \end{figure*}
        \item movl \%eax, \%ecx
        \par 89 c1 对应 movl \%eax, \%ecx
        \par 90 90 对应 nop nop
        \par c3 对应 retq
        \begin{figure*}[htbp]
            \centering
            \includegraphics*[width = 10cm]{eax_ecx.png}
        \end{figure*}
        \item movl \%ecx, \%edx
        \par 89 ca 对应 movl \%ecx, \%edx
        \par 08 db 对应 orb \%bl
        \par c3 对应 retq
        \begin{figure*}[htbp]
            \centering
            \includegraphics*[width = 10cm]{ecx_edx.png}
        \end{figure*}
        \item movl \%edx, \%esi
        \par 89 d6 对应 movl \%edx, \%esi
        \par 08 d2 对应 orb  \%dl
        \par c3 对应 retq
        \begin{figure*}[htbp]
            \centering
            \includegraphics*[width = 10cm]{edx_esi.png}
        \end{figure*}
        \item movq \%rax, \%rdi
        \par 和之前重复,不再赘述
    \end{itemize}