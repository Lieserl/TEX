\section{实验内容}
    \begin{itemize}
        \item 登录 \textbf{bupt1} 服务器,在 \textbf{home} 目录下可以找到一个 \textbf{targetn.tar} 文件,解压后得到如下文件:
        \begin{table}[htbp]
            \centering
            \begin{tabular}{c|c|c}
                README.txt & ctargat & rtarget \\
                cookie.txt & farm.c & hex2raw \\
            \end{tabular}
        \end{table}
        \item \textbf{ctarget} 和 \textbf{rtarget} 运行时从标准输入读入字符串,这两个程序都存在缓冲区溢出漏洞。通过代码注入的方法实现对 \textbf{ctarget} 程序的攻击,共有 \textbf{3} 关,输入一个特定字符串,可成功调用 \textbf{touch1},或 \textbf{touch2},或 \textbf{touch3} 就通关,并向计分服务器提交得分信息;通过 \textbf{ROP} 方法实现对 \textbf{rtarget} 程序的攻击,共有 \textbf{2} 关,在指定区域找到所需要的小工具,进行拼接完成指定功能,再输入一个特定字符串,实现成功调用 \textbf{touch2} 或 \textbf{touch3} 就通关,并向计分服务器提交得分信息;否则失败,但不扣分。因此,本实验需要通过反汇编和逆向工程对 \textbf{ctraget} 和 \textbf{rtarget} 执行文件进行分析,找到保存返回地址在堆栈中的位置以及所需要的小工具机器码。实验 \textbf{2} 的具体内容见 \textbf{实验2说明},尤其需要认真阅读各阶段的\textbf{Some Advice}提示。
        \item 本实验包含了 \textbf{5} 个阶段(或关卡),难度逐级递增。各阶段分数如下所示:
        \begin{figure*}[htbp]
            \centering
            \includegraphics*[width = 10cm]{pointTable.png}
        \end{figure*}
    \end{itemize}