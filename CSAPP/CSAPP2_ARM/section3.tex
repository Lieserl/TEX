\section{实验步骤及实验分析}
    \subsection{准备工作}
        \begin{enumerate}
            \item 登陆 \textbf{kunpeng3} 服务器,在自己的目录下找到 \textbf{bomb132.tar} 文件,使用 \textbf{tar -xvf bomb132.tar} 命令解压缩该文件。
                \begin{figure*}[htbp]
                    \centering
                    \includegraphics*[width = 10cm]{s0_0.png}
                \end{figure*}
            \item 使用 \textbf{cd bomb132} 命令移动到 \textbf{bomb132} 目录下并使用 \textbf{ls} 命令查看当前目录下的文件。可以发现,\textbf{bomb} 即是我们需要逆向的可执行文件,\textbf{bomb.c} 是提供的部分 \textbf{C} 代码。
                \begin{figure*}[htbp]
                    \centering
                    \includegraphics*[width = 10cm]{s0_1.png}
                \end{figure*}
            \item 使用 \textbf{gdb bomb} 命令进入调试阶段。至此,准备工作完毕,开始正式拆弹。
                \begin{figure*}[htbp]
                    \centering
                    \includegraphics*[width = 12cm]{s0_2.png}
                \end{figure*}
        \end{enumerate}

    \subsection{第一炸弹:字符串炸弹}
        \par 写在前面:由于调试过程中需要用到大量 \textbf{disassemble}、\textbf{stepi}、\textbf{nexti} 指令,全部截图显得十分冗余,故只在 \textbf{第一炸弹} 展示完整截图,其他部分相关命令只进行文字描述。其次,有时候会忘记截图,想起来的时候代码已经执行过几行了,所以看到进入函数后第一个 \textbf{disassemble} 命令指向的位置不是第一句,请不要感到意外。        \begin{enumerate}
            \item 进入调试界面,我们首先使用 \textbf{list} 命令查看第一颗炸弹的位置,在 \textbf{74} 行发现了第一颗炸弹。于是我们使用 \textbf{break 74} 命令在该行设置一个断点。
                \begin{figure*}[htbp]
                    \centering
                    \includegraphics*[width = 12cm]{s1_0.png}
                \end{figure*}
            \item 使用 \textbf{run} 命令运行程序,提示输入第一颗炸弹。由于我们此时并不知道密码是多少,因此随便填写。填写完成后,程序运行到断点位置暂停。
                \begin{figure*}[htbp]
                    \centering
                    \includegraphics*[width = 12cm]{s1_1.png}
                \end{figure*}
            \item 接下来我们使用 \textbf{stepi} 命令单步执行,进入 \textbf{phase\_1(input)} 函数内部进行拆弹。
                \begin{figure*}[htbp]
                    \hspace*{1.5cm}
                    \includegraphics*[width = 12cm]{s1_3.png}
                \end{figure*}
            \newpage
            \item 使用 \textbf{disassemble} 命令查看第一颗炸弹的汇编代码。接下来对汇编代码进行分析:
                \par 在 \textbf{<+8>} 位置,将 \textbf{\$0x402000} 移动到了 \textbf{x1} 寄存器中,并加上了 \textbf{0x680},猜测该寄存器存放了我们输入的字符串作为参数传入 \textbf{strings\_not\_equal} 函数中,然后在 \textbf{<+20>} 位置判断函数返回值是否为 \textbf{0}。如果是,则继续执行;否则转跳到 \textbf{<+32>} 位置,引爆炸弹,\textbf{BOMB!!!}
                \begin{figure*}[htbp]
                    \centering
                    \includegraphics*[width = 12cm]{s1_4.png}
                \end{figure*}
            \item 为了验证我们的猜想,我们先使用 \textbf{stepi} 命令移动到 \textbf{<+16>} 位置,然后使用 \par \textbf{x /s} 命令查看 \textbf{x1} 寄存器中究竟存放着什么,结果发现竟然就是我们梦寐以求的密码,第一颗炸弹成功解除!(\sout{更详细的方法和解释在基础版炸弹写过了,这里懒得写了})
                \begin{figure*}[htbp]
                    \centering
                    \includegraphics*[width = 12cm]{s1_5.png}
                \end{figure*}
            \item 重新运行程序,输入密码,第一炸弹被拆除。
                \begin{figure*}[htbp]
                    \centering
                    \includegraphics*[width = 12cm]{s1_6.png}
                \end{figure*}
        \end{enumerate}
    \newpage
    \subsection{第二炸弹:枯萎穿心攻击}
        \begin{enumerate}
            \item 我们在第二炸弹位置设置断点,进入函数内部,使用 \textbf{disassemble} 命令查看汇编代码。
                \begin{figure*}[htbp]
                    \centering
                    \includegraphics*[width = 12cm]{s2_0.png}
                \end{figure*}
            \item 由函数 \textbf{read\_six\_numbers} 可以推测这关的密码是 \textbf{6} 个符合一定顺序的数字,为了验证假设,我们进入函数内部查看其汇编代码。
            \item 在函数内部我们发现了提示中的关键函数 \textbf{sscanf},在函数之后程序将 \textbf{w0} 寄存器的内容(\textbf{sscanf} 的返回值)与 \textbf{5} 做比较,并在其小于等于 \textbf{5} 时引爆炸弹。
                \begin{figure*}[htbp]
                    \centering
                    \includegraphics*[width = 12cm]{s2_1.png}
                \end{figure*}
            \newpage
            \item 我们移动至 \textbf{<+40>} 位置,使用 \textbf{x /s } 命令查看 \textbf{sscanf} 的第一个参数值 \textbf{x1},发现函数确实是读入 \textbf{6} 个数字,并且中间用空格隔开。
                \begin{figure*}[htbp]
                    \hspace*{2cm}
                    \includegraphics*[width = 8cm]{s2_2.png}
                \end{figure*}
            \item 现在让我们来分析一下 \textbf{phase\_2} 中剩下的内容:
                \begin{itemize}
                    \item 读入完成后程序将 \textbf{w0 + 40}(即 \textbf{w0 + 0x28},\textbf{x1} 寄存器指向的地址) 指向地址的值与 \textbf{0x1} 做比较,若不相等则引爆炸弹。
                        \begin{figure*}[htbp]
                            \hspace*{1.5cm}
                            \includegraphics*[width = 11cm]{s2_3.png} 
                        \end{figure*}
                    \item 接下来程序直接跳转至 \textbf{<+64>} 位置,分别把 \textbf{[x19]} 和 \textbf{[x19, \#4]} 地址的值移动到 \textbf{w1} 和 \textbf{w0} 寄存器中,接下来将 \textbf{w0} 和 \textbf{w1} 左移一位后的值做比较,若为真则跳转回 \textbf{<+52>} 位置;若为否则引爆炸弹,推测 \textbf{x19} 寄存器指向的一片地址空间(即 \textbf{数组})存放着我们输入的数据,使用\textbf{x /6d } 命令查看,发现结果和我们的推测一样(此处是补截的图,所以输入为密码),与此同时,我们也知道了之前 \textbf{w0 + 40} 指向的值与 \textbf{0x1} 做比较是为了检查数组的第一个元素是否为 \textbf{1}。
                        \begin{figure*}[htbp]
                            \hspace*{1.5cm}
                            \includegraphics*[width = 11cm]{s2_4.png} \\ 
                            \hspace*{2cm}
                            \includegraphics*[width = 8cm]{s2_4_1.png} 
                        \end{figure*}
                    \item 接下来我们先看到 \textbf{<+52>} 位置,将 \textbf{x19} 加上 \textbf{0x4} 的偏移量,此时 \textbf{x19} 指向的元素从数组的第一个元素变为数组的第二个元素,接下来将 \textbf{x20} 与 \textbf{x19} 做比较,如果相等,则跳转到 \textbf{<+88>} 位置,程序结束;否则程序继续运行,再次执行上述判断。                    
                        \begin{figure*}[htbp]
                            \centering
                            \includegraphics*[width = 12cm]{s2_5.png}
                        \end{figure*} 
                    \newpage
                    \item 回头看到 \textbf{<+36>} 位置,发现 \textbf{x20} 存放的是 \textbf{x19 + 0x14},现在,我们可以看出这是一个 \textbf{do while} 循环。
                        \begin{figure*}[htbp] 
                            \hspace*{2cm}
                            \includegraphics*[width = 10cm]{s2_6.png}
                        \end{figure*}
                \end{itemize}
            \item 至此,我们基本了解了这段汇编代码的功能:每次迭代判断数组第 \textbf{k} 个元素是否为下一个元素的 $\frac{1}{2}$。如果不等,则引爆炸弹;否则移动到 \textbf{k + 1 } 位置,直到 \textbf{x19} $=$ \textbf{x19 + 0x14} 时停止迭代,即当 \textbf{k} $=$ \textbf{5} 时停止循环。因此,所求密码是首项为 \textbf{1},公比为 \textbf{2},长度为 \textbf{6} 的等比数列。即 \textbf{1 \ 2 \ 4 \ 8 \ 16 \ 32}。
            \item 重新运行输入密码,成功拆除第二炸弹。
                \begin{figure*}[htbp]
                    \hspace*{1.2cm}
                    \includegraphics*[width = 10cm]{s2_7.png}
                \end{figure*}
        \end{enumerate}
    \newpage
    \subsection{第三炸弹:败者食尘}
        \begin{enumerate}
            \item 查看第三炸弹的汇编代码如下:
                \begin{figure*}[htbp]
                    \centering
                    \includegraphics*[width = 10cm]{s3_0.png}
                \end{figure*}
            \newpage
            \item 我们还是如法炮制地查看 \textbf{sscanf} 的输入,发现这段密码是由 \textbf{数字 \ 数字 \ } 组成的,并且在输入小于一个字符时引爆炸弹。
                \begin{figure*}[htbp]
                    \hspace*{1cm}
                    \includegraphics*[width = 12cm]{s3_1_0.png} \\ 
                    \hspace*{1.5cm}
                    \includegraphics*[width = 6cm]{s3_1.png}
                \end{figure*}
            \item 继续往后,程序将某一地址存放的值存入 \textbf{w1} 寄存器,并将其与 \textbf{0x3} 比较,推测 \textbf{[x29, \#28]} 位置存放的是我们传入的变量,使用 \textbf{print} 命令查看,发现是我们传入的第一个变量。
                \begin{figure*}[htbp]
                    \hspace*{1cm}
                    \includegraphics*[width = 12cm]{s3_2.png} \\
                    \hspace*{1.7cm}
                    \includegraphics*[width = 3cm]{s3_2_1.png}
                \end{figure*}
            \item 接下来是一系列判断语句,我们对其进行归纳并列出对应条件下的转跳位置。
            \begin{tabular}{cccccccc}
                0 & 1 & 2 & 3 & 4 & 5 & 6 & 7 \\
                <+116> & <+188> & <+196> & <+204> & <+212> & <+220> & <+228> & <+148> \\
            \end{tabular}
                \begin{figure*}[htbp]
                    \hspace*{1cm}
                    \includegraphics*[width = 14cm]{s3_3.png} \\
                    \hspace*{1cm}
                    \includegraphics*[width = 13cm]{s3_3_1.png}
                \end{figure*}
            \newpage
            \item 其中,当 \textbf{w1} $<$ \textbf{0} 时,程序首先会跳转至 \textbf{<+104>} 位置,并运行至 \textbf{<+120>} 位置,此时 \textbf{cbnz} 命令会判断 \textbf{w1} 是否为 \textbf{0}。如果不是,则引爆炸弹,故 \textbf{w1} 不能小于 \textbf{0};当 \textbf{w1} $>$ \textbf{7} 时,程序会一直运行至 \textbf{<+84>} 位置引爆炸弹。
                \begin{figure*}[htbp]
                    \hspace*{1cm}
                    \includegraphics*[width = 14cm]{s3_3_2.png}
                \end{figure*}
            \item 至此,我们发现程序会在 \textbf{w1} 为不同值时执行不同的命令,故此处应为一个 \textbf{switch} 语句。由于每个部分的功能都一样,所以在此只解释 \textbf{w1} $=$ \textbf{4} 对应的情况:
                \begin{itemize}
                    \item 通过上述转跳表,我们知道应跳转至 \textbf{<+212} 位置,然后程序将 \textbf{w0} 置 \textbf{0},并跳转至 \textbf{<+136>} 位置。
                        \begin{figure*}[htbp]
                            \hspace*{1cm}
                            \includegraphics*[width = 14cm]{s3_4.png}
                        \end{figure*}
                    \item 在 \textbf{<+136>} 位置,程序首先进行 \textbf{4} 次加减法,由于加上和减去的数一致,因此最终 \textbf{w0} 的值仍为 \textbf{0}。
                        \begin{figure*}[htbp]
                            \hspace*{1cm}
                            \includegraphics*[width = 14cm]{s3_5.png}
                        \end{figure*}
                    \item 接下来判断传入的第一个变量是否大于 \textbf{5},如果大于,则引爆炸弹。
                        \includegraphics*[width = 14cm]{s3_6.png}
                    \item 继续进行,我们发现其将 \textbf{[x29, \#24]} 地址存放的值存入 \textbf{w1} 中,使用 \textbf{print} 指令查看,发现是我们传入的第二个变量。
                        \begin{figure*}[htbp]
                            \hspace*{1cm}
                            \includegraphics*[width = 6cm]{s3_7.png}
                        \end{figure*}
                    \newpage
                    \item 然后比较 \textbf{w0} 和 \textbf{w1} 中的值是否相等,如果相等,就不会引爆炸弹。我们知道 \textbf{w0} 的值为 \textbf{0},\textbf{w1} 是传入的第二个变量,因此第一个变量为 \textbf{4},第二个变量为 \textbf{0}。
                        \begin{figure*}[htbp]
                            \hspace*{1cm}
                            \includegraphics*[width = 14cm]{s3_8.png}
                        \end{figure*}
                \end{itemize}
            \item 输入密码,解除第三炸弹。
                \begin{figure*}[htbp]
                    \hspace*{1.5cm}
                    \includegraphics*[width = 8cm]{s3_9.png}
                \end{figure*}
        \end{enumerate}
    \newpage
    \subsection{第四炸弹:递归炸弹}
        \par 我愿称其为除第一炸弹以外最简单的炸弹,因为真的很容易卡 \textbf{bug} 来 \textbf{skip}。(\sout{而且和低阶炸弹基本一模一样,嘻嘻})
        \begin{enumerate}
            \item 还是先查看其汇编代码。
                \begin{figure*}[htbp]
                    \hspace*{2cm}
                    \includegraphics*[width = 10cm]{s4_0.png}
                \end{figure*}
            \item 依然是先看 \textbf{sscanf} 传入参数用的几个寄存器,发现这次的密码是 \textbf{数字 \ 数字} \par 的形式。同时由上题可知,读入的两个数字分别被放在 \textbf{[x29, \#24]} 和 \textbf{[x29, \#28]} 中;在 \textbf{sscanf} 返回值不等于 \textbf{2} 时爆炸。
                \begin{figure*}[htbp]
                    \hspace*{2cm}
                    \includegraphics*[width = 6cm]{s4_1.png}
                \end{figure*}
            \item 接着往下看,在 \textbf{<+36>} 位置将第二个元素拷贝到 \textbf{w0} 寄存器中,减去 \textbf{0x2} 后与 \textbf{0x2} 做比较。如果大于 \textbf{0x2},则炸弹爆炸。所以我们输入的第二个数应该小于等于 \textbf{4}。然后将 \textbf{func4} 的参数立即数 \textbf{0x5} 和第二个元素传入 \textbf{w0}  和 \textbf{w1} 中,调用函数 \textbf{func4}。
                \begin{figure*}[htbp]
                    \hspace*{1.5cm}
                    \includegraphics*[width = 12cm]{s4_2.png} \\
                    \hspace*{2cm}
                    \includegraphics*[width = 4cm]{s4_2_1.png} 
                \end{figure*}
            \newpage
            \item 进入 \textbf{func4} 内部,查看其汇编代码,发现其是一个递归函数,我们逐步分析其功能。
                \begin{figure*}[htbp]
                    \hspace*{2cm}
                    \includegraphics*[width = 12cm]{s4_3.png}
                \end{figure*}
                \begin{itemize}
                    \item 首先是递归出口,当 \textbf{w0} 寄存器中的值小于等于 \textbf{0} 时直接返回。注意在函数的最后将 \textbf{0x0} 赋值给了 \textbf{w0},因此此处的返回值一定为 \textbf{0};接着先把 \textbf{w0} 复制到 \textbf{w19} 中,再将第二个元素复制到 \textbf{w0}中,当 \textbf{w19} 寄存器(即原来的 \textbf{w0})中的值等于 \textbf{1} 时,返回第二个元素。
                        \begin{figure*}[htbp]
                            \hspace*{2cm}
                            \includegraphics*[width = 12cm]{s4_4_0.png} \\
                            \hspace*{2cm}
                            \includegraphics*[width = 12cm]{s4_4_1.png}
                        \end{figure*}
                    \item 接着将 \textbf{w1} 复制到 \textbf{w20} 中,令 \textbf{w0} $=$ \textbf{w19 - 1}(\textbf{w19} 即原来的 \textbf{w0}),调用 \textbf{func4} 函数。然后将返回值 \textbf{w0} 加上 \textbf{w20}(即第二个元素)并存储在 \textbf{w21} 中。将 \textbf{w19}(也就是 \textbf{w0} 的初值)减 \textbf{2} 的值存放在 \textbf{w0} 中,再次调用 \textbf{func4},并将返回值加上之前的处理过的返回值 \textbf{w21}。
                        \begin{figure*}[htbp]
                            \hspace*{2cm}
                            \includegraphics*[width = 12cm]{s4_5.png}
                        \end{figure*}
                    \item 根据上面描述,我们可以将其翻译成 \textbf{C} 代码:
                    \lstinputlisting[style = C]{rec.c}
                \end{itemize}
            \item 分析完 \textbf{func4} 的功能,我们接着往下看。程序将第一个元素与函数返回值做比较。如果不相等,炸弹爆炸。
                \begin{figure*}[htbp]
                    \hspace*{2cm}
                    \includegraphics*[width = 12cm]{s4_6.png}
                \end{figure*}
            \item 综上,我们的输入必须满足 \textbf{elem1} $=$ \textbf{fun4(6, elem2)},同时第二个输入要小于等于 \textbf{4},至此第四炸弹解除。
                \begin{figure*}[htbp]
                    \hspace*{2cm}
                    \includegraphics*[width = 12cm]{s4_7.png}
                \end{figure*}
        \end{enumerate}
        \par \textbf{skip}方法:只要知道递归边界,我可以直接传边界进去,这样可以省去对递归函数的分析;或者随便试一组数据,单步执行,在 \textbf{func4} 结束后查看 \textbf{\%rax} 寄存器的值就可以找到对应的答案。
    \subsection{第五炸弹:真·二进制炸弹}
        \begin{enumerate}
            \item 老样子,先查汇编。(\sout{这题也和低阶的差不多,血赚})
                \begin{figure*}[htbp]
                    \hspace*{1.5cm}
                    \includegraphics*[width = 12cm]{s5_0.png}
                \end{figure*}
            \item 可以看到程序在 \textbf{<+16>} 位置调用了 \textbf{string\_length} 函数,并在之后将返回值与 \textbf{0x6} 比较。说明这次的输入是一个长度为 \textbf{6} 的字符串。
                \begin{figure*}[htbp]
                    \hspace*{1.5cm}
                    \includegraphics*[width = 12cm]{s5_1.png}
                \end{figure*}
            \item 继续往下看,发现在 \textbf{<+68>} 位置的跳转指令会跳转回 \textbf{<+48>} 位置,显然这也是一个 \textbf{do while} 循环,接下来分析循环的部分。
                \begin{figure*}[htbp]
                    \hspace*{1.5cm}
                    \includegraphics*[width = 12cm]{s5_2.png}
                \end{figure*}
            \newpage
            \item 在循环外,程序先将 \textbf{x19} \textbf{x19 + 6}拷贝到 \textbf{x2} 和 \textbf{x0}中,再将 \textbf{0x0} 赋值给了 \textbf{w3},然后我们先看循环部分的汇编代码。循环开始时,将 \textbf{[x2]} 所指向地址的数据的末尾 \textbf{8} 位做 \textbf{0} 拓展传给了 \textbf{w1} 寄存器,并让 \textbf{x2} 的值加一,然后将 \textbf{x1} 与 \textbf{0xf} 做按位与的操作。我们知道,\textbf{0xf} 的二进制表示为 \textbf{1111},因此这句话执行的结果为保留 \textbf{x1} 的最低 \textbf{4} 位。
                \begin{figure*}[htbp]
                    \hspace*{1.5cm}
                    \includegraphics*[width = 12cm]{s5_3.png}
                \end{figure*}
            \item 此外,我们注意到 \textbf{<+64>} 位置在 \textbf{x2} 的值等于 \textbf{x0}(即最初的 \textbf{x2 + 6}) 时结束循环。因此 \textbf{x2} 的作用为作为指针用来遍历我们输入的字符串。
                \begin{figure*}[htbp]
                    \hspace*{1.5cm}
                    \includegraphics*[width = 12cm]{s5_4.png}
                \end{figure*}
            \item 接下来,和以前一样,程序将之前取出的末尾 \textbf{4} 位作为索引去访问 \textbf{x4} 所指向的内存空间(因为输入为 \textbf{int} 类型,所以索引要乘上 \textbf{4},即 \textbf{lsl \#2})。我们回过头来看 \textbf{<+40>} 位置,发现 \textbf{x4} $=$ \textbf{0x402640},于是使用 \textbf{x /16d} 命令查看该地址存放的数据,发现是 \textbf{16} 个不重复打乱的数字。 
                \par 不知道为啥,\textbf{figure*} 环境只能插 \textbf{3} 张图,所以这里拿炸弹凑个数。
                \begin{figure*}[htbp]
                    \hspace*{1.5cm}
                    \includegraphics*[width = 12cm]{s5_5_0.png} \\
                    \hspace*{1.4cm}
                    \includegraphics*[width = 14cm]{s5_5_1.png} \\
                    \hspace*{2cm}
                    \includegraphics*[width = 14cm]{s5_5_2.png}
                \end{figure*}
            \newpage
            \item 然后程序将取得的结果存入 \textbf{w1} 寄存器中,并将其加到 \textbf{w3}。接着判断是否结束循环。经过上述分析,可以看出这段循环的作用是将输入字符串的每一字符的最低 \textbf{4} 位作为索引,取出一些数字并相加存放在 \textbf{w3} 中。
                \begin{figure*}[htbp]
                    \hspace*{1.5cm}
                    \includegraphics*[width = 12cm]{s5_6.png}
                \end{figure*}
            \item 接着在 \textbf{<+72>} 位置将之前得到的结果与 \textbf{0x3d} 做比较,如果相等则 \textbf{SAFE}。
                \begin{figure*}[htbp]
                    \hspace*{1.5cm}
                    \includegraphics*[width = 12cm]{s5_7.png}
                \end{figure*}
            \item 于是,这次的密码就很清晰了:通过查找 \textbf{acsii} 码确定究竟输入哪些字符可以从之前的数中依次取出某些数,使其和为 \textbf{0x3d},这里选择 \textbf{16 \ 15 \ 14 \ 13  \ 2 \ 1},对应下来为 \textbf{ONJEPS}。至此 \textbf{Dr. Evil} 的阴谋又双彻底破灭,可喜可贺。
        \end{enumerate}
