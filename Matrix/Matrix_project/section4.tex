\section{矩阵分解方法研究}
\fontsize{12pt}{14pt}\selectfont
\songti
    \subsection{矩阵的 LU 分解}
        \subsubsection{矩阵 LU 分解的步骤推导}
            \par 矩阵的 LU 分解与 $Gauss$ 消元过程密切相关,因此我们先推导 $Gauss$ 消元的过程:
            \par \textbf{Gauss 消元过程:}
            \par 设 $A^{(0)} = A$,其元素 $a_{ij}^{(0)} = a_{ij}(i, \ j = 1, \ 2, \ \dots, \ n)$. 记 $A$ 的 $k$ 阶顺序主子式为 $\Delta_k(k = 1, \ 2, \ \dots, \ n)$. 如果 $\Delta_1 = a_{11}^{(0)} \ne 0$,令 $c_{i1} = \frac{a_{i1}^{(0)}}{a_{11}^{0}}(i = 2, \ 3, \ \dots, \ n)$,并构造 Frobenius 矩阵
            \begin{equation*}
                L_1 = \begin{bmatrix}
                    1 & & & \\ c_{21} & 1 & & \\ \vdots & & \ddots & \\ c_{n1} & & & 1 
                \end{bmatrix}, \quad L_1^{-1}\begin{bmatrix}
                    1 & & & \\ -c_{21} & 1 & & \\ \vdots & & \ddots & \\ -c_{n1} & & & 1  
                \end{bmatrix}
            \end{equation*}
            计算 
            \begin{equation*}
                L_1^{-1}A^{(0)} = \begin{bmatrix}
                    a_{11}^{(0)} & a_{12}^{(0)} & \cdots & a_{1n}^{(0)} \\ & a_{22}^{(1)} & \cdots & a_{n}^{(1)} \\ & \vdots & & \vdots \\ & a_{n2}^{(1)} & \cdots & a_{nn}^{(1)}
                \end{bmatrix} = A^{(1)}
            \end{equation*}
            由此可见,$A^{0} = A$ 的第一列除主元 $a_{11}^{(0)}$ 外,其余元素全被化为零. 故有
            \begin{equation*}
                A^{(0)} = L_1A^{(1)}
            \end{equation*}
            因为倍加初等变换不改变矩阵的行列式的值,所以由 $A^{(0)}$ 得 $A$ 的二阶顺序主子式为
            \begin{equation*}
                \Delta_2 = a_{11}^{(0)}a_{22}^{(11)}
            \end{equation*}
            若 $\Delta_2 \ne 0$,则 $a_{22}^{(1)} \ne 0$. 令 $c_{i2} = \frac{a_{i2}^{(1)}}{a_{22}^{(1)}}(i = 3, \ 4, \ \dots, \ n)$,并构造矩阵
            \begin{equation*}
                L_2 = \begin{bmatrix}
                    1 & & & & \\ & 1 & & & \\ & c_{32} & 1 & & \\ & \vdots & & \ddots & \\ & c_{n2} & & & 1
                \end{bmatrix}, \quad L_2^{-1} = \begin{bmatrix}
                    1 & & & & \\ & 1 & & & \\ & -c_{32} & 1 & & \\ & \vdots & & \ddots & \\ & -c_{n2} & & & 1
                \end{bmatrix}
            \end{equation*}
            计算,得到
            \begin{equation*}
                L_2^{-1}A^{(1)} = \begin{bmatrix}
                    a_{11}^{(0)} & a_{12}^{(0)} & a_{13}^{(0)} & \cdots & a_{1n}^{(0)} \\ & a_{22}^{(1)} & a_{23}^{(1)} & \cdots & a_{n}^{(1)} \\ & & a_{33}^{(2)} & \cdots & a_{3n}^{(2)} \\ & & \vdots & & \vdots \\ & & a_{n3}^{(2)} & \cdots & a_{nn}^{(2)}
                \end{bmatrix} = A^{(2)}
            \end{equation*}
            同理,$A^{(2)}$ 的前两列中主元以下的元素全为零,有
            \begin{equation*}
                A^{(1)} = L_2A^{(2)}
            \end{equation*}
            因为倍加初等变换不改变矩阵的行列式的值,所以由 $A^{(2)}$ 得 $A$ 的三阶顺序主子式为
            \begin{equation*}
                \Delta_3 = a_{11}^{(0)}a_{22}^{(1)}a_{33}^{(2)}
            \end{equation*}
            如此迭代至第 $r-1$ 步,得到
            \begin{equation*}
                A^{(r-1)} = \begin{bmatrix}
                    a_{11}^{(0)} & \cdots & a_{1,r-1}^{(0)} & a_{1r}^{(0)} & \cdots & a_{1n}^{(0)} \\ & \ddots & \vdots & \vdots & & \vdots \\ & & a_{r-1,r-1}^{(r-2)} & a_{r-1,r}^{(r-2)} & \cdots & a_{r-1,n}^{(r-2)} \\ & & & a_{rr}^{(r-1)} & \cdots & a_{rn}^{(r-1)} \\ & & & \vdots & & \vdots \\ & & & a_{nr}^{(r-1)} & \cdots & a_{nn}^{(r-1)}
                \end{bmatrix}
            \end{equation*}
            若 $\Delta_r \ne 0$,则 $a_{rr}^{(r-1)} \ne 0$. 令 $C_{ir} = \frac{a_{ir}^{(r-1)}}{a_{rr}^{(r-1)}}(i = r + 1, \ r + 2, \ \dots, \ n)$,并构造 Frobenius 矩阵
            \begin{equation*}
                L_r = \begin{bmatrix}
                    1 & & & & & \\ & \ddots & & & & \\ & & 1 & & & \\ & & c_{r+1,r} & 1 & & \\ & & \vdots & & \ddots & \\ & & c_{nr} & & & 1
                \end{bmatrix}, \quad L_r^{-1} = \begin{bmatrix}
                    1 & & & & & \\ & \ddots & & & & \\ & & 1 & & & \\ & & -c_{r+1,r} & 1 & & \\ & & \vdots & & \ddots & \\ & & -c_{nr} & & & 1
                \end{bmatrix}
            \end{equation*}
            计算可得
            \begin{equation*}
                L_r^{-1}A^{(r-1)} = \begin{bmatrix}
                    a_{11}^{(0)} & \cdots & a_{1r}^{(0)} & a_{1,r+1}^{(0)} & \cdots & a_{1n}^{(0)} \\ & \ddots & \vdots & \vdots & & \vdots \\ & & a_{rr}^{(r-1)} & a_{r,r+1}^{(r-1)} & \cdots & a_{rn}^{(r-1)} \\ & & & a_{r+1,r+1}^{(r)} & \cdots & a_{r+1,n}^{(r)} \\ & & & \vdots & & \vdots \\ & & & a_{n,r+1}^{(r)} & \cdots & a_{nn}^{(r)}
                \end{bmatrix}
            \end{equation*}
            同理,有
            \begin{equation*}
                A^{(r-1)} = L_rA^{(r)}
            \end{equation*}
            故 $A$ 的 $r+1$ 阶顺序主子式为 
            \begin{equation*}
                \Delta_{r+1} = a_{11}^{(0)}a_{22}^{(1)} \dots a_{rr}^{(r-1)}a_{r+1,r+1}^{(r)}
            \end{equation*}
            迭代至第 $n-1$ 步,有
            \begin{equation*}
                A^{(n-1)} = \begin{bmatrix}
                    a_{11}^{0} & a_{12}^{0} & \cdots & a_{1,n-1}^{0} & a_{1n}^{0} \\ & a_{22}^{1} & \cdots & a_{2,n-1}^{1} & a_{2n}^{1} \\ & & \ddots & \vdots & \vdots \\ & & & a_{n-1,n-1}^{(n-2)} & a_{n-1,n}^{(n-2)} \\ & & & & a_{nn}^{(n-1)}
                \end{bmatrix}
            \end{equation*}
            \par 以上便是 \textbf{Gauss 消元} 的全过程. 根据前面所述,由于消元过程中未使用行、列的交换,因此其进行到底的条件是 \\ $a_{11}^{(0)}, \ a_{22}^{(1)}, \ \dots, \ a_{n-1, n-1}^{(n-2)}$ 都不为零,即
            \begin{equation*}
                \Delta_r \ne 0 \quad (r = 1, \ 2, \ \dots, \ n-1)
            \end{equation*}
            \\
            \par 从上述推导过程中不难发现,当 $\Delta_r \ne 0$ 时,有 
            \begin{equation*}
                A = A^{(0)} = L_1A^{(1)} = L_1L_2A^{(2)} = \dots = L_1L_2 \dots L_{n-1}A^{(n-1)}
            \end{equation*}
            令 $L = L_1L_2 \dots L_{n-1}$,有
            \begin{equation*}
                L = L_1L_2 \dots L_{n-1} = \begin{bmatrix}
                    1 & & & & \\ c_{21} & 1 & & & \\ \vdots & \vdots & \ddots & & \\ c_{n-1,1} & c_{n-1,2} & \cdots & 1 & \\ c_{n1} & c_{n2} & \cdots & c_{n,n-1} & 1
                \end{bmatrix}
            \end{equation*}
            $L$ 是一个对角元素都为 $1$ 的下三角矩阵,称为 \textbf{单位下三角矩阵}. 而由之前的推导,$A^{(n-1)}$ 是一个上三角矩阵,令 $U = A^{(n-1)}$,有
            \begin{equation*}
                A = LU
            \end{equation*}
            如此,矩阵 $A$ 就被分解成一个单位下三角矩阵与一个上三角矩阵的乘积. 现在,我们对 矩阵的 \textbf{LU 分解} 做如下定义:
            \par \textbf{LU 分解:}如果方阵 $A$ 可分解成一个下三角矩阵 $L$ 和一个上三角矩阵 $U$ 的乘积,则称 $A$ 可作 \textbf{三角分解 \ } 或 \textbf{LU 分解}. 如果方阵 $A$ 可分解成 $A = LDU$,其中 $L$ 是单位下三角矩阵,$D$ 是对角矩阵,$U$ 是单位上三角矩阵,则称 $A$ 可作 \textbf{LDU 分解}.
            \par 对于矩阵的 LU 分解有如下定理:
            \begin{enumerate}
                \item 设 $A = (a_{ij})_{n \times n}$ 是 $n$ 阶矩阵,则当且仅当 $A$ 的顺序主子式 $\Delta_k \ne 0 \ \ (k = 1, \ 2, \ \dots, \ n - 1)$ 时,$A$ 可唯一地分解为 $A = LDU$,其中 $L$ 是单位下三角矩阵,$U$ 是单位上三角矩阵,且
                    \begin{equation*}
                        D = diag(d_1, \ d_2, \ \dots, d_n)
                    \end{equation*}
                    其中 $d_k = \frac{\Delta_k}{\Delta_{k-1}}(k = 1, \ 2, \ \dots, \ n; \ \Delta_0 = 1)$.
                \item $n$ 阶可逆矩阵 $A$ 有三角分解 $A = LU$ 的充要条件是 $A$ 的顺序主子式 $\Delta_k \ne 0 \ \ (k = 1, \ 2, \ \dots, \ n — 1)$.
                \item 设 $A$ 是 $n$ 阶可逆矩阵,则存在置换矩阵 $P$ 使 $PA$ 的 $n$ 个顺序主子式非零,且有
                    \begin{equation*}
                        PA = L\hat{U} = LDU
                    \end{equation*}
            \end{enumerate}

            \par 由上述过程,我们可以得出矩阵的 $LDU$ 分解方法:
            \begin{enumerate}
                \item 判断矩阵 $A$ 的 $k$ 阶顺序主子式都满足 $\Delta_k \ne 0$,如果不,可以参考定理 $3$ 构造 $PAx = Pb$ 并求其解
                \item 根据 Gauss 消元过程递推地计算 $L_k, \ A^{(k)}$
                \item 计算 $L = L_{1}L_{2} \dots L_{n-1}$
                \item 根据 $A = L_{1}L_{2} \dots L_{n-1}A^{(n-1)} = LA^{(n-1)}$ 得到矩阵的 LDU 分解
            \end{enumerate}

            \par \textbf{Crout 分解:}设矩阵 $A$ 有唯一的 $LDU$ 分解. 若把 $A = LDU$ 中的 $L$ 与 $D$ 结合起来,并且用 $\hat{L}$ 来表示,就得到唯一的分解为
            \begin{equation*}
                A = (LD)U = \hat{L}U
            \end{equation*}
            \par 下面给出 Crout 分解的步骤推导. 设 
            \begin{equation*}
                \hat{L} = \begin{bmatrix}
                    l_{11} & & & \\ l_{21} & l_{22} & & \\ \vdots & \vdots & \ddots & \\ l_{n1} & l_{n2} & \cdots & l_{nn}
                \end{bmatrix}, \quad U = \begin{bmatrix}
                    1 & u_{12} & \cdots & u_{1n} \\ & 1 & \cdots & u_{2n} \\ & & \ddots & \vdots \\ & & & 1
                \end{bmatrix}
            \end{equation*}
            根据 $A = \hat{L}U$,可以得到
            \begin{align*}
                a_{i1} = l_{i1} \quad &or \quad l_{i1} = a_{i1} \quad (i = 1, \ 2, \ \dots, \ n)  \tag{4.1.1}\\
                a_{1j} = l_{11}u_{1j} \quad &or \quad u_{1j} = \frac{a_{1j}}{l_{11}} \quad (j = 2, \ 3, \ \dots, \ n) \tag{4.1.2}
            \end{align*}
            对于 $k = 2, \ 3, \ \dots, \ n$,当 $i \geq k$ 时,有
            \begin{align*}
                \because a_{ik} &= l_{i1}u_{1k} + \dots + l_{i,k-1}u_{k-1,k} + l_{ik} \\
                \therefore l_{ik} &= a_{ik} - (l_{i1}u_{1k} + \dots + l_{i,k-1}u_{k-1,k}) \tag{4.1.3}
            \end{align*}
            而当 $j > k$ 时,有
            \begin{align*}
                \because a_{kj} &= l_{k1}u_{1j} + \dots + l_{k,k-1}u_{k-1,j} + l_{kk}u_{kj} \\
                \therefore u_{kj} &= \frac{1}{l_{kk}}[a_{kj} - (l_{k1}u_{1j} + \dots + l_{k,k-1}u_{k-1,j})] \tag{4.1.4}
            \end{align*}
            \par 在实际计算时,顺序如下
            \begin{enumerate}
                \item 利用式 $(4.1.1)$ 计算 $\hat{L}$ 的第一列
                    \par 利用式 $(4.1.2)$ 计算 $U$ 的第一行
                \item 利用式 $(4.1.3)$ 计算 $\hat{L}$ 的第二列
                    \par 利用式 $(4.1.4)$ 计算 $U$ 的第二行 
                \item $\cdots \cdots$
                \item 利用式 $(4.1.3)$ 计算 $\hat{L}$ 的第 $n-1$ 列
                    \par 利用式 $(4.1.4)$ 计算 $U$ 的第 $n-1$ 行
                \item 利用式 $(4.1.3)$ 计算 $\hat{L}$ 的第 $n$ 列
            \end{enumerate} 
            \par \textbf{Doolittle 分解:}设矩阵 $A$ 有唯一的 $LDU$ 分解. 若把 $A = LDU$ 中的 $D$ 与 $U$ 结合起来,并且用 $\hat{U}$ 来表示,就得到唯一的分解为
            \begin{equation*}
                A = L(DU) = L \hat{U}
            \end{equation*}
            \par 与上述过程类似,我们可以得到 Doolittle 分解的计算公式为
            \begin{equation*}
                \begin{cases*}
                    u_{ik} = a_{ik} - \sum_{r=1}^{i-1}l_{ir}u_{rk} \quad (k = i, \ i + 1, \ \dots, n) \\
                    l_{ki} = \frac{1}{u_{ii}}(a_{ki} - \sum_{r=1}^{i-1}l_{kr}u_{ri}) \quad (k = i + 1, \ \dots, \ n)
                \end{cases*}
            \end{equation*}
            \par \textbf{实对称矩阵的 Cholesky 分解:}当 $A$ 为实对称正定矩阵时,$\Delta_k > 0 \ \ (k = 1, \ 2, \ \dots, \ n)$. 于是 $A$ 有唯一的 LDU 分解 $A = LDU$,其中 $D = diag(d_1, \ d_2, \ \dots, \ n)$, 且 $d_i > 0 \ \ (i = 1, \ 2, \ \dots, \ n)$. 令
            \begin{equation*}
                \widetilde{D} = diag(\sqrt{d_1}, \ \sqrt{d_2}, \ \dots, \ \sqrt{d_n})
            \end{equation*}
            则有 $A = L\widetilde{D}^2U$. 由 $A^T = A$ 得到 $L\widetilde{D}^2U = U^T\widetilde{D}^2L^T$,再由分解的唯一性有 $L = U^T, \ U = L^T$,故有
            \begin{equation*}
                A = L\widetilde{D}^2L^T = LDL^T 
            \end{equation*}
            或者
            \begin{equation*}
                A = L\widetilde{D}^2L^T = (L\widetilde{D})(L\widetilde{D})^T = GG^T
            \end{equation*}
            其中,$G = L\widetilde{D}$ 是下三角矩阵.
            \par 令 $G = (g_{ij})$,则由 $A = GG^T$ 两端相应元素相等可得
            \begin{equation*}
                a_{ij} = g_{i1}g_{j1} + g_{i2}g_{j2} + \dots + g_{ij}g_{ij} \quad (i > j), \quad a_{ii} = g_{i1}^2 + g_{i2}^2 + \dots + g_{ii}^2
            \end{equation*}
            从而得到计算 $g_{ij}$ 的递推关系式为
            \begin{equation*}
                g_{ij} = \begin{cases*}
                    (a_{ii} - \sum_{k=1}^{i-1}g_{ik}^2)^{1/2} \quad &(i = j) \\
                    \frac{1}{g_{jj}}(a_{ij} - \sum_{k=1}^{j-1}g_{ik}g_{jk}) \quad &(i > j) \\
                    0 \quad &(i < j)
                \end{cases*}
            \end{equation*}
            \par 实际求解过程是按照行优先原则对 $g_{ij}$ 进行求解.

        \subsubsection{举例展示求法}
            \begin{problem}
                \par 求矩阵 $A$ 的 LDU 分解和 Doolittle 分解,其中
                \begin{equation*}
                    A = \begin{bmatrix}
                        5 & 2 & -4 & 0 \\ 2 & 1 & -2 & 1 \\ -4 & -2 & 5 & 0 \\ 0 & 1 & 0 & 2
                    \end{bmatrix}
                \end{equation*}
             \end{problem}
             \begin{solution}
                \\ \textbf{Gauss 消元法:}
                \begin{enumerate}
                    \item 判断矩阵 $A$ 的 $k$ 阶顺序主子式都满足 $\Delta_k \ne 0$,如果不,可以参考定理 $3$ 构造 $PAx = Pb$ 并求其解
                        \par 因为 $\Delta_1 = 5, \ \Delta_2 = 1, \ \Delta_3 = 1$,所以 $A$ 有唯一的 LDU 分解.
                    \item 根据 Gauss 消元过程递推地计算 $L_k, \ A^{(k)}$
                        \par 对 $A$ 构造矩阵 
                        \begin{equation*}
                            L_1 = \begin{bmatrix}
                                1 & & & \\ \frac{2}{5} & 1 & & \\ -\frac{4}{5} & 0 & 1 & \\ 0 & 0 & 0 & 1
                            \end{bmatrix}, \quad L_1^{-1} = \begin{bmatrix}
                                1 & & & \\ -\frac{3}{5} & 1 &  & \\ \frac{4}{5} & 0 & 1 & \\ 0 & 0 & 0 & 1
                            \end{bmatrix}
                        \end{equation*}
                        计算,得 
                        \begin{equation*}
                            L_1^{-1}A = \begin{bmatrix}
                                5 & 2 & -4 & 0 \\ 0 & \frac{1}{5} & -\frac{2}{5} & 1 \\ 0 & -\frac{2}{5} & \frac{4}{5} & 0 \\ 0 & 1 & 0 & 2
                            \end{bmatrix} = A^{(1)}
                        \end{equation*}
                        对 $A^{(1)}$ 构造矩阵 
                        \begin{equation*}
                            L_2 = \begin{bmatrix}
                                1 & & & \\ 0 & 1 & & \\ 0 & -2 & 1 & \\ 0 & 5 & 0 & 1
                            \end{bmatrix}, \quad L_2^{-1} = \begin{bmatrix}
                                1 & & & \\ 0 & 1 & & \\ 0 & 2 & 1 & \\ 0 & -5 & 0 & 1
                            \end{bmatrix}
                        \end{equation*}
                        计算,得 
                        \begin{equation*}
                            L_2^{-1}A^{(1)} = \begin{bmatrix}
                                5 & 2 & -4 & 0 \\ 0 & \frac{1}{5} & - \frac{2}{5} & 1 \\ 0 & 0 & 1 & 2 \\ 0 & 0 & 2 & -3
                            \end{bmatrix} = A^{(2)}
                        \end{equation*}
                        对 $A^{(2)}$ 构造矩阵 
                        \begin{equation*}
                            L_3 = \begin{bmatrix}
                                1 & & & \\ 0 & 1 & & \\ 0 & 0 & 1 & \\ 0 & 0 & 2 & 1
                            \end{bmatrix}, \quad L_3^{-1} = \begin{bmatrix}
                                1 & & & \\ 0 & 1 & & \\ 0 & 0 & 1 & \\ 0 & 0 & -2 & 1
                            \end{bmatrix}
                        \end{equation*}
                        计算,得 
                        \begin{equation*}
                            L_3^{-1}A^{(2)} = \begin{bmatrix}
                                5 & 2 & -4 & 0 \\ & \frac{1}{5} & -\frac{2}{5} & 1 \\ & & 1 & 2 \\ & & & -7
                            \end{bmatrix} = A^{(3)}
                        \end{equation*}
                    \item 计算 $L = L_{1}L_{2} \dots L_{n-1}$
                        \par 令 $L = L_1L_2L_3$,有
                        \begin{equation*}
                            L = L_1L_2L_3 = \begin{bmatrix}
                                1 & & & \\ \frac{2}{5} & 1 & & \\ -\frac{4}{5} & -2 & 1 & \\ 0 & 5 & 2 & 1
                            \end{bmatrix}
                        \end{equation*}
                    \item 根据 $A = L_{1}L_{2} \dots L_{n-1}A^{(n-1)} = LA^{(n-1)}$ 得到矩阵的 LDU 分解
                        \par 于是,$A$ 的 LDU 分解为
                        \begin{equation*}
                            A = L_1L_2L_3A^{(3)} = \begin{bmatrix}
                                1 & & & \\ \frac{2}{5} & 1 & & \\ -\frac{4}{5} & -2 & 1 & \\ 0 & 5 & 2 & 1
                            \end{bmatrix}\begin{bmatrix}
                                5 & & & \\ & \frac{1}{5} & & \\ & & 1 & \\ & & & 7
                            \end{bmatrix}\begin{bmatrix}
                                1 & \frac{2}{5} & -\frac{4}{5} & 0 \\ & 1 & -2 & 5 \\ & & 1 & 2 \\ & & & 1
                            \end{bmatrix}
                        \end{equation*}
                \end{enumerate}
                \textbf{Doolittle 分解:}
                \par 这里我们使用 Crout 分解来求 Doolittle 分解.
                \begin{enumerate}
                    \item 利用式 $(4.1.1)$ 计算 $\hat{L}$ 的第一列
                        \par 利用式 $(4.1.2)$ 计算 $U$ 的第一行
                        \begin{align*}
                            l_{11} &= a_{11} = 5, \quad l_{21} = a_{21} = 2, \quad l_{31} = a_{31} = -4, \quad l_{41} = a_{41} = 0 \\
                            u_{12} &= \frac{a_{12}}{l_{11}} = \frac{2}{5}, \quad u_{13} = \frac{a_{13}}{l_{11}} = -\frac{4}{5}, \quad u_{14} = \frac{a_{14}}{l_{11}} = 0
                        \end{align*}
                    \item 利用式 $(4.1.3)$ 计算 $\hat{L}$ 的第二列
                        \par 利用式 $(4.1.4)$ 计算 $U$ 的第二行
                        \begin{align*}
                            l_{22} &= a_{22} - l_{21}u_{12} = \frac{1}{5}, \quad  l_{32} = a_{32} - l_{31}u_{12} = -\frac{2}{5}, \quad  l_{42} = a_{42} - l_{41}u_{12} = 1 \\
                            u_{23} &= \frac{1}{l_{22}}[a_{23} - l_{21}u_{13}] = -2, \quad u_{24} = \frac{1}{l_{24}}[a_{24} - l_{21}u_{14}] = 5        
                        \end{align*} 
                    \item 利用式 $(4.1.3)$ 计算 $\hat{L}$ 的第三列
                        \par 利用式 $(4.1.4)$ 计算 $U$ 的第三行
                        \begin{align*}
                            l_{33} &= a_{33} - (l_{31}u_{13} + l_{32}u_{23}) = 1, \quad  l_{43} = a_{43} - (l_{41}u_{13} + l_{42}u_{23}) = 2 \\
                            u_{34} &= \frac{1}{l_{33}}[a_{34} - (l_{31}u_{14} + l_{32}u_{24})] = 2
                        \end{align*}
                    \item 利用式 $(4.1.3)$ 计算 $\hat{L}$ 的第四列
                        \begin{equation*}
                            l_{44} = a_{44} - (l_{41}u_{14} + l_{42}u_{24} + l_{43}u_{34}) = 7
                        \end{equation*}
                \end{enumerate} 
                    \par 求得
                    \begin{equation*}
                        \widetilde{L} = \begin{bmatrix}
                            5 & & & \\ 2 & \frac{1}{5} & & \\ -4 & -\frac{2}{5} & 1 & \\ 0 & 1 & 2 & 7
                        \end{bmatrix}, \quad U = \begin{bmatrix}
                            1 & \frac{2}{5} & -\frac{4}{5} & 0 \\ & 1 & -2 & 5 \\ & & 1 & 2 \\ & & & 1
                        \end{bmatrix}
                    \end{equation*}
                    \par 故 Doolittle 分解为 
                    \begin{equation*}
                        A = \widetilde{L}U = LDU = L\widetilde{U} = \begin{bmatrix}
                            1 & & & \\ \frac{2}{5} & 1 & & \\ -\frac{4}{5} & -2 & 1 & \\ 0 & 5 & 2 & 1
                        \end{bmatrix}\begin{bmatrix}
                            5 & 2 & -4 & 0 \\ & \frac{1}{5} & -\frac{2}{5} & 1 \\ & & 1 & 2 \\ & & & -7
                        \end{bmatrix}
                    \end{equation*}
            \end{solution}
            \begin{problem}
                求对称正定矩阵 $A$ 的 Cholesky 分解,其中
                \begin{equation*}
                    A = \begin{bmatrix}
                        5 & 2 & -4 \\ 2 & 1 & -2 \\ -4 & -2 & 5
                    \end{bmatrix}
                \end{equation*}
            \end{problem} 
            \begin{solution}
                \\ \textbf{Cholesky 分解:}
                \par 容易验证 $A$ 式对称正定矩阵.
                \begin{align*}
                    g_{11} &= \sqrt{a_{11}} = \sqrt{5} \\
                    g_{21} &= \frac{a_{21}}{g_{11}} = \frac{2}{\sqrt{5}}, \quad g_{22} = (a_{22} - g_{21}^2)^{1/2} = \frac{1}{\sqrt{5}} \\
                    g_{31} &= \frac{a_{31}}{g_{11}} = -\frac{4}{\sqrt{5}}, \quad g_{32} = \frac{a_{32} - g_{31}g_{21}}{g_{22}} = -\frac{2}{\sqrt{5}}, \quad g_{33} = (a_{33} - g_{31}^2 - g_{32}^2)^{1/2} = 1
                \end{align*}
                从而,得
                \begin{equation*}
                    A = \begin{bmatrix}
                        \sqrt{5} & & \\ \frac{2}{\sqrt{5}} & \frac{1}{\sqrt{5}} \\ -\frac{4}{\sqrt{5}} & -\frac{2}{\sqrt{5}} & 1
                    \end{bmatrix}\begin{bmatrix}
                        \sqrt{5} & \frac{2}{\sqrt{5}} & -\frac{4}{\sqrt{5}} \\  & \frac{1}{\sqrt{5}} & -\frac{2}{\sqrt{5}} \\  &  & 1
                    \end{bmatrix}
                \end{equation*}
            \end{solution}

    \subsection{矩阵的 QR 分解}
        \subsubsection{矩阵 QR 分解的步骤推导}
            \par \textbf{Givens 矩阵:}设实数 $c$ 和 $s$ 满足 $c^2 + s^2 = 1$,对应的 $Givens$ 矩阵为 $T_{ij}(c, \ s)(i \ne j)$,则
            \begin{enumerate}
                \item $T_{ij}(c, \ s)$ 是正交矩阵;
                \item $[T_{ij}(c, \ s)]^{-1} = T_{ij}(c, \ -s)$;
                \item $det[T_{ij}(c, \ s)] = 1$;
                \item 设实向量 $\textbf{textit{x}} = (\xi_1, \ \xi_2, \ \dots, \ \xi_n)^T, \ y = T_{ij}(c, \ s)\textbf{\textit{x}} = (\eta_1, \ \eta_2, \ \dots, \ eta_n)^T$,则
                    \begin{align*}
                        \eta_i &= c\xi_i + s\xi_j, \quad \eta_j = -s\xi_i + c\xi_j \\
                        \eta_k &= \xi_k \quad (k \ne i, \ j; \ k = 1, \ 2, \ \dots, \ n)
                    \end{align*}
                \item 设 $\textbf{\textit{x}} = (\xi_1, \ \xi_2, \ \dots, \ \xi_n)^T \ne 0$,则存在有限个 Givens 矩阵的乘积,记作 $T$,使得 $T\textbf{\textit{x}} = |\textbf{\textit{x}}|e_1$
            \end{enumerate}
            \par \textbf{Householder 矩阵:}设 $u \in R^n$ 是单位列向量,对应 Householder 矩阵为 $H_u = I_n - 2uu^T$,则
            \begin{enumerate}
                \item $H_u$ 是对称$(H^T=H)$、正交$(H^TH=I)$、自逆$(H^2=I)$、对合矩阵$H^{-1}=H$
                \item $detH = -1$
                \item 任意给定非零列向量 $\textbf{\textit{z}} \in R^n(n > 1)$ 及单位列向量 $\textbf{\textit{z}} \in R^n$,则存在 Householder 矩阵 $H$,使得 $H\textbf{\textit{x}} = |\textbf{\textit{x}}|\textbf{\textit{z}}$
                \item 一个 Givens 矩阵可以表示为两个 Householder 矩阵的乘积
            \end{enumerate}
            \par \textbf{矩阵的 QR 分解:}如果实(复)可逆矩阵 $A$ 能够化成正交(酉)矩阵 $Q$ 与实(复)可逆上三角矩阵 $R$ 的乘积,即
            \begin{equation*}
                A = QR
            \end{equation*}
            则称 $A = QR$ 为 $A$ 的 \textbf{QR 分解}.
            \\
            \textbf{Schimidt 正交化求 QR 分解:}
            \par 记矩阵 $A$ 的 $n$ 个列向量依次为 $a_1, \ a_2, \ \dots, \ a_n$. 因为 $A$ 可逆,所以这 $n$ 个列向量线性无关. 将它们按 Schmidt 正交化方法正交化之,可得到 $n$ 个标准正交列向量 $q_1, \ q_2, \ \dots, \ q_n$.
            \par 对 $a_1, \ a_2, \ \dots, \ a_n$ 正交化,可得
            \begin{equation*}
                \begin{cases*}
                    \begin{aligned}
                        b_1 &= a_1 \\ b_2 &= a_2 - k_{21}b_1 \\ &\cdots \cdots \\ b_n &= a_n - k_{n,n-1}b_{n-1} - \dots - k_{n1}b_1
                    \end{aligned}
                \end{cases*}
            \end{equation*}
            其中 $k_{ij} = \frac{a_i, \ b_j}{(b_j, \ b_j)} \ \ (j < i)$. 将上式改写为
            \begin{equation*}
                \begin{cases*}
                    \begin{aligned}
                        a_1 &= b_1 \\ a_2 &= k_{21}b_1 + b_2 \\ &\cdots \cdots \\ a_n &= k_{n1}b_1 + k_{n2}b_2 + \dots + k{n,n-1}b_{n-1} + b_n
                    \end{aligned}
                \end{cases*}
            \end{equation*}
            用矩阵形式表示为 
            \begin{equation*}
                (a_1, \ a_2, \ \dots, \ a_n) = (b_1, \ b_2, \ \dots, \ b_n)C
            \end{equation*}
            其中
            \begin{equation*}
                C = \begin{bmatrix}
                    1 & k_{21} & \cdots & k_{n1} \\ & 1 & \cdots & k_{n2} \\ & & \ddots & \vdots \\ & & & 1
                \end{bmatrix}
            \end{equation*}
            对 $b_1, \ b_2, \ \dots, \ b_n$ 单位化,可以得到
            \begin{equation*}
                q_i = \frac{1}{|b_i|}b_i \quad (i = 1, \ 2, \ \dots, \ n)
            \end{equation*}
            于是有 
            \begin{equation*}
                (a_1, \ a_2, \ \dots, \ a_n) = (b_1, \ b_2, \ \dots, b_n)C = (q_1, \ q_2, \ \dots, \ q_n)\begin{bmatrix}
                    |b_1| & & & \\ & |b_2| & & \\ & & \ddots & \\ & & & |b_n|
                \end{bmatrix}C
            \end{equation*}
            令 
            \begin{equation*}
                \begin{cases*}
                    Q = (q_1, \ q_2, \ \dots, \ q_n) \\ R = diag(|b_1|, \ |b_2|, \ \dots, \ |b_n|) \cdot C
                \end{cases*}
            \end{equation*}
            则 $Q$ 是正交(酉)矩阵,$R$ 是可逆上三角矩阵,且有 $A = QR$.
            \par 总结求解流程如下:
            \begin{enumerate}
                \item 对 $A$ 的列向量组 $a_1, \ a_2, \ \dots, a_n$ 正交化得正交向量组 $b_1, \ b_2, \ \dots, \ b_n$
                \item 构造正交矩阵 $Q = (q_1, \ q_2, \ \dots, \ q_n)$,其中 $q_j = \frac{b_j}{|b_j|}$
                \item 构造上三角矩阵 $R = diag(|b_1|, \ |b_2|, \ \dots, \ |b_n|)\cdot C$,那么 $A = QR$
            \end{enumerate}

            \par \textbf{Givens 变换求 QR 分解:}
            \par 由 $det A \ne 0$ 知,$A$ 的第 $1$ 列 $b^{(1)} = (a_{11}, \ a_{21}, \ \dots, \ a_{n1})^T \ne 0$. 存在有限个 Gievns 矩阵的乘积,记作 $T_1$, 使得
            \begin{equation*}
                T_1b^{(1)} = |b^{(1)}|e_1 \quad (e_1 \in R^n)
            \end{equation*}
            令 $a_{11}^{(1)} = |b^{(1)}|$,则有
            \begin{equation*}
                T_1A = \begin{bmatrix}
                    a_{11}^{(1)} &  a_{12}^{(1)} & \cdots &  a_{1n}^{(1)} \\ 0 & & & \\ \vdots & & A^{(1)} & \\ 0 & & & 
                \end{bmatrix}
            \end{equation*}
            由 $detA^{(1)} \ne 0$ 知,$A^{(1)}$ 的第 $1$ 列 $b^{(2)} = (a_{22}^{(1)}, \ a_{23}^{(1)}, \ \dots, \ a_{n2}^{(1)})$. 存在有限个 Givens 矩阵的乘积,记作 $T_2$,使得
            \begin{equation*}
                T_2b^{(2)} = |b^{(2)}|e_1 \quad (e_1 \in R^{(n-1)})
            \end{equation*}
            令 $a_{22}^{(2)} = |b^{(2)}|$,则有
            \begin{equation*}
                T_2A^{(1)} = \begin{bmatrix}
                    a_{22}^{(2)} &  a_{23}^{(2)} & \cdots &  a_{24}^{(2)} \\ 0 & & & \\ \vdots & & A^{(2)} & \\ 0 & & & 
                \end{bmatrix}
            \end{equation*} 
            以此类推,最后有 
            \begin{equation*}
                T_{n-1}b^{(n-1)} = |b^{(n-1)}|e_1 \quad (e_1 \in R^2)
            \end{equation*}
            令 $a_{n-1,n-1}^{n-1} = |b^{(n-1)}|$,则有
            \begin{equation*}
                T_{n-1}A^{(n-2)} = \begin{bmatrix}
                    a_{n-1,n-1}^{(n-1)} & a_{n-1,n}^{(n-1)} \\ 0 & a_{nn}^{(n-1)}
                \end{bmatrix}
            \end{equation*}
            最后,令
            \begin{equation*}
                T = \begin{bmatrix}
                    I_{n-2} & O \\ O & T_{n-1} 
                \end{bmatrix} \dots \begin{bmatrix}
                    I_2 & O \\ O & T_3
                \end{bmatrix} \begin{bmatrix}
                    1 & O \\ O & T_2
                \end{bmatrix} T_1
            \end{equation*}
            则 $T$ 是有限个 Givens 矩阵的乘积,使得
            \begin{equation*}
                TA = \begin{bmatrix}
                    a_{11}^{(1)} & a_{12}^{(1)} & \cdots & a_{1,n-1}^{(1)} & a_{1n}^{(1)} \\ & a_{22}^{(2)} & \cdots &a_{2,n-1}^{(2)} & a_{2n}^{(2)} \\ & & \ddots & \vdots & \vdots \\ & & & a_{n-1,n-1}^{(n-1)} & a_{n-1,n}^{(n-1)} \\ & & & & a_{nn}^{(n-1)}
                \end{bmatrix}
            \end{equation*}
            记为 $R$,有 $A = QR$,其中 $Q = T^{-1}$. 因为 $T$ 是有限个 Givens 矩阵的乘积,而 Givens 矩阵都是正交矩阵,所以 $T$ 是正交矩阵,于是 $Q = T^{-1} = T^T$ 也是正交矩阵.
            \par 总结求解流程如下:
            \begin{enumerate}
                \item 对 $A$ 的第 $1$ 列 $b^{(1)}$ 构造有限个 Givens 矩阵的乘积 $T_1$,使得
                    \begin{align*}
                        T_1b^{(1)} = |b^{(1)}|e_1 \quad (e_1 \in R^n)
                        T_1A = \begin{bmatrix}
                            a_{11}^{(1)} &  a_{12}^{(1)} & \cdots &  a_{1n}^{(1)} \\ 0 & & & \\ \vdots & & A^{(1)} & \\ 0 & & & 
                        \end{bmatrix}
                    \end{align*}
                \item 对 $A^{(1)}$ 的第 $1$ 列 $b^{(2)}$ 构造 $T_2$
                \item $\cdots \cdots$
                \item 对 $A^{(n-2)}$ 的第 $1$ 列 $b^{(n-1)}$ 构造 Givens 矩阵 $T_{n-1}$
                \item 构造上三角矩阵 $R$,计算正交矩阵 $Q$,那么 $A = QR$.
            \end{enumerate}
            \par \textbf{Householder 变换求 QR 分解:}
            \par Householder 变化求解过程与 Givens 类似,这里不再赘述重复部分.
            \par 最后得出 
            \begin{equation*}
                S = \begin{bmatrix}
                    I_{n-2} & O \\ O & H_{n-1} 
                \end{bmatrix} \dots \begin{bmatrix}
                    I_2 & O \\ O & H_3
                \end{bmatrix} \begin{bmatrix}
                    1 & O \\ O & H_2
                \end{bmatrix} H_1
            \end{equation*}
            并注意到,若 $H_n$ 是 $n-l$ 阶 Householder 矩阵,即
            \begin{equation*}
                H_u = I_{n-1} - 2uu^T \quad (u \in R^{n-l}, \ u^Tu = 1)
            \end{equation*}
            令 $v = \begin{bmatrix} 0 \\ u \end{bmatrix} \in R^n$,则 $v^Tv = u^Tu = 1$,且
            \begin{equation*}
                \begin{bmatrix}
                    I_l & O \\ O & H_u 
                \end{bmatrix} = \begin{bmatrix}
                    I_l & O \\ O & I_{n-l}
                \end{bmatrix} - 2\begin{bmatrix}
                    O & O \\ O & uu^T
                \end{bmatrix} = I_n - 2\begin{bmatrix}
                    0 \\ u
                \end{bmatrix}\begin{bmatrix}
                    0^T | u^T
                \end{bmatrix} = I_n - 2vv^T
            \end{equation*}
            是 $n$ 阶 Householder 矩阵. 因此,$S$ 是有限个 Householder 矩阵的乘积,且使得 $SA$ 等于上述上三角矩阵,记为 $R$,则 $Q = S^{-1} = S^T$,有 $A = QR$.
        \subsubsection{举例展示求法}
            \begin{problem}
                \par 用 Schmidt 正交化方法求矩阵
                \begin{equation*}
                    A = \begin{bmatrix}
                        0 & 1 & 1 \\ 1 & 1 & 0 \\ 1 & 0 & 1
                    \end{bmatrix}
                \end{equation*}
                的 QR 分解
            \end{problem}
            \begin{solution}
                \begin{enumerate}
                    \item 对 $A$ 的列向量组 $a_1, \ a_2, \ \dots, a_n$ 正交化得正交向量组 $b_1, \ b_2, \ \dots, \ b_n$
                        \par 令 $a_1 = (0, \ 1, \ 1)^T, \ a_2 = (1, \ 1, \ 0)^T, \ a_3 = (1, \ 0, \ 1)^T$,正交化可得
                        \begin{align*}
                            b_1 &= a_1 = (0, \ 1, \ 1)^T \\ b_2 &= a_2 - \frac{1}{2}b_1 = (1, \ \frac{1}{2}, \ -\frac{1}{2})^T \\ b_3 &= a_3 - \frac{1}{3}b_2 - \frac{1}{2}b_1 = (\frac{2}{3}, \ -\frac{2}{3}, \ \frac{2}{3})^T
                        \end{align*}
                    \item 构造正交矩阵 $Q = (q_1, \ q_2, \ \dots, \ q_n)$,其中 $q_j = \frac{b_j}{|b_j|}$
                        \par 根据给定方法构造矩阵
                        \begin{equation*}
                            Q = \begin{bmatrix}
                                0 & \frac{2}{\sqrt{6}} & \frac{1}{\sqrt{3}} \\ \frac{1}{\sqrt{2}} & \frac{1}{\sqrt{6}} & -\frac{1}{\sqrt{3}} \\ \frac{1}{\sqrt{2}} & -\frac{1}{\sqrt{6}} & \frac{1}{\sqrt{3}}
                            \end{bmatrix}
                        \end{equation*}
                    \item 构造上三角矩阵 $R = diag(|b_1|, \ |b_2|, \ \dots, \ |b_n|)\cdot C$,那么 $A = QR$
                        \begin{equation*}
                            R = \begin{bmatrix}
                                \sqrt{2} & \frac{1}{\sqrt{2}} & \frac{1}{\sqrt{2}} \\ & \frac{3}{\sqrt{6}} & \frac{1}{\sqrt{6}} \\ & & \frac{2}{\sqrt{3}}
                            \end{bmatrix}
                        \end{equation*}
                        则有 $A = QR$
                \end{enumerate}
            \end{solution}
            \begin{problem}
                \par 用 Givens 变换求矩阵 
                \begin{equation*}
                    A = \begin{bmatrix}
                        2 & 2 & 1 \\ 0 & 2 & 2 \\ 2 & 1 & 2
                    \end{bmatrix}
                \end{equation*}
                的 QR 分解
            \end{problem}
            \begin{solution}
                \begin{enumerate}
                    \item 对 $A$ 的第 $1$ 列 $b^{(1)}$ 构造有限个 Givens 矩阵的乘积 $T_1$
                       \par 对 $A$ 的第 $1$ 列 $b^{(1)} = (2, \ 0, \ 2)^T$,构造 $T_1$ 使得 $T_1b^{(1)} = |b^{(1)}|e_1$ 
                       \begin{align*}
                            T_{12} = \begin{bmatrix}
                                \frac{1}{\sqrt{2}} & 0 & \frac{1}{\sqrt{2}} \\ 0 & 1 & 0 \\ -\frac{1}{\sqrt{2}} & 0 & \frac{1}{\sqrt{2}}
                            \end{bmatrix}, \quad T_{13}b^{(1)} = \begin{bmatrix}
                                2\sqrt{2} \\ 0 \\ 0
                            \end{bmatrix} \\
                            T_1 = T_{13}, \quad T_1A = \begin{bmatrix}
                                2\sqrt{2} & \frac{3}{\sqrt{2}} & \frac{3}{\sqrt{2}} \\ 0 & 2 & 2 \\ 0 & -\frac{1}{\sqrt{2}} & \frac{1}{\sqrt{2}}
                            \end{bmatrix}
                       \end{align*}
                    \item 对 $A^{(1)}$ 的第 $1$ 列 $b^{(2)}$ 构造 $T_2$
                        \par 对 $A^{(1)} = \begin{bmatrix}
                            2 & 2 \\ -\frac{1}{\sqrt{2}} & \frac{1}{\sqrt{2}}
                        \end{bmatrix}$ 的第 $1$ 列 $b^{(2)} = \begin{bmatrix}
                            2 \\ -\frac{1}{\sqrt{2}}
                        \end{bmatrix}$,构造 $T_2$,使得 $T_2b^{(2)} = |b^{(2)}|e_1$
                        \begin{align*}
                            T_{12} = \begin{bmatrix}
                                \frac{2\sqrt{2}}{3} & -\frac{1}{3} \\ \frac{1}{3} & \frac{2\sqrt{2}}{3}
                            \end{bmatrix}, \quad T_{12}b^{(2)} = \begin{bmatrix}
                                \frac{3}{\sqrt{2}} \\ 0
                            \end{bmatrix} \\
                            T_2 = T_{12}, \quad T_2A^{(1)} = \begin{bmatrix}
                                \frac{3}{\sqrt{2}} & \frac{7}{3\sqrt{2}} \\ 0 & \frac{4}{3}
                            \end{bmatrix}
                        \end{align*}
                    \item 构造上三角矩阵 $R$,计算正交矩阵 $Q$,那么 $A = QR$.
                        \par 最后,令 $T = \begin{bmatrix}
                            1 & \\ & T_2
                        \end{bmatrix}T_1$,则有
                        \begin{equation*}
                            Q = T^T = \begin{bmatrix}
                                \frac{1}{\sqrt{2}} & \frac{1}{3\sqrt{2}} & -\frac{2}{3} \\ 0 & \frac{4}{3\sqrt{2}} & \frac{1}{3} \\ \frac{1}{\sqrt{2}} & -\frac{1}{3\sqrt{2}} & \frac{2}{3}
                            \end{bmatrix}, \quad R = \begin{bmatrix}
                                2\sqrt{2} & \frac{3}{\sqrt{2}} & \frac{3}{\sqrt{2}} \\ & \frac{3}{\sqrt{2}} & \frac{7}{3\sqrt{2}} \\ & & \frac{4}{3}
                            \end{bmatrix}
                        \end{equation*}
                        则有 $A = QR$
                \end{enumerate}
            \end{solution}

    \subsection{矩阵的满秩分解}
        \subsubsection{矩阵满秩分解的步骤推导}
            \par \textbf{满秩分解:}设 $A \in C_r^{m \times n}(r > 0)$,如果存在矩阵 $F \in C_r^{m \times r}$ 和 $G \in C_r^{r \times n}$,使得
            \begin{equation*}
                A = FG
            \end{equation*}
            则称其为矩阵 $A$ 的 \textbf{满秩分解}.
            \par 当 $A$ 是满秩(列满秩或行满秩)矩阵时,$A$ 可分解为一个因子是单位矩阵,另一个因子是 $A$ 本身,称此满秩分解为 \textbf{平凡分解}.
            \par 设 $A \in C_R^{m \times n}(r > 0)$. $rankA = r$ 时,根据矩阵的初等变换理论,对 $A$ 进行初等行变换,可将 $A$ 化为阶梯形矩阵 $B$,即
            \begin{equation*}
                A \stackrel{\text{行}}{\longrightarrow} B = \begin{bmatrix}
                    G \\ O
                \end{bmatrix}, \quad G \in C_r^{r \times n}
            \end{equation*}
            于是存在有限个 $m$ 阶初等矩阵的乘积,记作 $P$,使得 $PA = B$,或者 $A = P^{-1}B$. 将 $P^{-1}$ 分块为
            \begin{equation*}
                P^{-1} = \left[\begin{array}{c:c} F & S \end{array}\right] \quad (F \in C_r^{m \times r}, \ S \in C_{m-r}^{m \times (m-r)})
            \end{equation*}
            则有
            \begin{equation*}
                A = P^{-1}B = \left[\begin{array}{c:c} F & S \end{array}\right]\begin{bmatrix}
                    G \\ O
                \end{bmatrix} = FG
            \end{equation*}
            其中 $F$ 是列满秩矩阵,$G$ 是行满秩矩阵.
            \par 故可以使用矩阵的初等行变换方法求矩阵的满秩分解:
            \begin{enumerate}
                \item $\left[\begin{array}{c:c} A & I \end{array}\right] \stackrel{\text{行}}{\longrightarrow} \left[\begin{array}{c:c} B & P \end{array}\right]$,其中 $B$ 为阶梯形矩阵
                \item 计算 $P^{-1}$(或者 $P^{-1}$ 的前 $r$ 列)
                \item 取 $F$ 为 $P^{-1}$ 的前 $r$ 列构成的列满秩矩阵,$G$ 为 $B$ 的前 $r$ 行构成的行满秩矩阵,那么 $A = FG$
            \end{enumerate}
            \par 除此之外,我们还可以通过 \textbf{Hermite 标准形 \ } 求解满秩分解.
            \\
            \par \textbf{Hermite 标准形:}设 $B \in C_r^{m \times n} \ (r > 0)$,且满足:
            \begin{enumerate}
                \item $B$ 的前 $r$ 行中每一行至少含一个非零元素,且第一个非零元素是 $1$,而后 $m-1$ 行元素均为零;
                \item 若 $B$ 中第 $i$ 行的第一个非零元素 $1$ 在第 $j_i$ 列 $(i = 1, \ 2, \ \dots, \ r)$,则 $j_1 < j_2 < \dots < j_r$;
                \item $B$ 中的 $j_1, \ j_2, \ \dots, \ j_r$ 列为单位矩阵 $I_m$ 的前 $r$ 列;
            \end{enumerate}
            那么就称 $B$ 为 \textbf{Hermite 标准形},即为初等变换意义下的行最简形.
            \par \textbf{拟 Hermite 标准形:}设 $B \in C_r^{m \times n} \ (r > 0)$,且满足:
            \begin{enumerate}
                \item $B$ 的后 $m-r$ 行元素均为零;
                \item $B$ 中的 $j_1, \ j_2, \ \dots, \ j_r$ 列为单位矩阵 $I_m$ 的前 $r$ 列;
            \end{enumerate}
            那么就称 $B$ 为 \textbf{拟 Hermite 标准形}.
            \par \textbf{Hermite 标准形求解满秩分解:}设 $A \in C_r^{m \times n} \ (r > 0)$ 的 (拟)Hermite 标准形 为$B$,那么,在 $A$ 的满秩分解式中,可取 $F$ 为 $A$ 的 $j_1, \ j_2, \ \dots , \ j_r$ 列构成的 $m \times r$ 矩阵,$G$ 为 $B$ 的前 $r$ 行构成的 $r \times n$ 矩阵.
            \par 由 $A \stackrel{\text{行}}{\longrightarrow} B$ 知,存在 $m$ 阶可逆矩阵 $P$,使得 $PA = B$,或者 $A = P^{-1}B$,可将 $P^{-1}$ 分块为
            \begin{equation*}
                P^{-1} = \left[\begin{array}{c:c}
                    F & S
                \end{array}\right], \quad (F \in C_r^{m \times r}, \quad S \in C_{m-r}^{m \times (m-r)})
            \end{equation*}
            可得满秩分解 $A = FG$,其中 $G$ 为 $B$ 的前 $r$ 行构成的 $r \times n$ 矩阵.
            \par 下面确定列满秩矩阵 $F$. 参照 $A$ 的 (拟)Hermite 标准形 $B$,构造 $n \times r$ 矩阵,有
            \begin{equation*}
                P_1 = (e_{j_1}, \ \dots, \ e_{j_r})
            \end{equation*}
            其中 $e_j$ 表示单位矩阵 $I_n$ 的第 $j$ 个列向量,则有
            \begin{equation*}
                GP_1 = I_r, \quad AP_1 = (FG)P_1 = F(GP_1) = F
            \end{equation*}
            即 $F$ 为 $A$ 的 $j_1, \ j_2, \ \dots, \ j_r$ 列构成的矩阵.
            \par 具体步骤如下所示:
            \begin{enumerate}
                \item $A \stackrel{\text{行}}{\longrightarrow} B$,其中 $B$ 为 Hermite 标准形矩阵,且 $B$ 的 $j_1, \ j_2, \ \dots, \ j_r$ 列为单位矩阵 $I_m$ 的前 $r$ 列;
                \item 取 $F$ 为 $A$ 的 $j_1, \ j_2, \ \dots, \ j_r$ 列构成的列满秩矩阵,$G$ 为 $B$ 的前 $r$ 行构成的行满秩矩阵,那么 $A = FG$.
            \end{enumerate}

        \subsubsection{举例展示求法}
            \begin{problem}
                \par 求下列各矩阵的满秩分解:
                \begin{equation*}
                    (1) \quad \begin{bmatrix}
                        1 & 2 & 3 & 0 \\ 0 & 2 & 1 & -1 \\ 1 & 0 & 2 & 1 \\
                    \end{bmatrix}; \quad (2) \quad \begin{bmatrix}
                        1 & -1 & 1 & 1 \\ -1 & 1 & -1 & -1 \\ -1 & -1 & 1 & 1 \\ 1 & 1 & -1 & -1
                    \end{bmatrix}
                \end{equation*}
            \end{problem}
            \begin{solution}
                \\ \textbf{逆矩阵方法:}
                \begin{enumerate}
                    \item $\left[\begin{array}{c:c} A & I \end{array}\right] \stackrel{\text{行}}{\longrightarrow} \left[\begin{array}{c:c} B & P \end{array}\right]$,其中 $B$ 为阶梯形矩阵
                        \begin{equation*}
                            \left[\begin{array}{c:c} A & I \end{array}\right] = \left[\begin{array}{cccc:ccc}
                                1 & 2 & 3 & 0 & 1 & 0 & 0 \\ 0 & 2 & 1 & -1 & 0 & 1 & 0 \\ 1 & 0 & 2 & 1 & 0 & 0 & 1
                            \end{array}\right] \rightarrow \left[\begin{array}{cccc:ccc}
                                1 & 0 & 2 & 1 & 1 & -1 & 0 \\ 0 & 1 & \frac{1}{2} & -\frac{1}{2} & 0 & \frac{1}{2} & 0 \\ 0 & 0 & 0 & 0 & -1 & 1 & 1
                            \end{array}\right]
                        \end{equation*}
                        \par 则有 
                        \begin{equation*}
                            B = \begin{bmatrix}
                                1 & 0 & 2 & 1 \\ 0 & 1 & \frac{1}{2} & -\frac{1}{2} \\ 0 & 0 & 0 & 0
                            \end{bmatrix}, \quad P = \begin{bmatrix}
                                1 & -1 & 0 \\ 0 & \frac{1}{2} & 0 \\ -1 & 1 & 1
                            \end{bmatrix}
                        \end{equation*}
                    \item 计算 $P^{-1}$(或者 $P^{-1}$ 的前 $r$ 列)
                        \begin{equation*}
                            P^{-1} = \begin{bmatrix}
                                1 & 2 & 0 \\ 0 & 2 & 0 \\ 1 & 0 & 1
                            \end{bmatrix}
                        \end{equation*}
                    \item 取 $F$ 为 $P^{-1}$ 的前 $r$ 列构成的列满秩矩阵,$G$ 为 $B$ 的前 $r$ 行构成的行满秩矩阵,那么 $A = FG$
                        \begin{equation*}
                            F = \begin{bmatrix}
                                1 & 2 \\ 0 & 2 \\ 1 & 0
                            \end{bmatrix}, \quad G = \begin{bmatrix}
                                1 & 0 & 2 & 1 \\ 0 & 1 & \frac{1}{2} & -\frac{1}{2}
                            \end{bmatrix}
                        \end{equation*}
                        故有
                        \begin{equation*}
                            A = FG = \begin{bmatrix}
                                1 & 2 \\ 0 & 2 \\ 1 & 0
                            \end{bmatrix}\begin{bmatrix}
                                1 & 0 & 2 & 1 \\ 0 & 1 & \frac{1}{2} & -\frac{1}{2}
                            \end{bmatrix}
                        \end{equation*}
                \end{enumerate}
                \textbf{Hermite 标准形方法:}
                \begin{enumerate}
                    \item $A \stackrel{\text{行}}{\longrightarrow} B$,其中 $B$ 为 Hermite 标准形矩阵,且 $B$ 的 $j_1, \ j_2, \ \dots, \ j_r$ 列为单位矩阵 $I_m$ 的前 $r$ 列;
                        \begin{equation*}
                            A \stackrel{\text{行}}{\longrightarrow} \begin{bmatrix}
                                1 & 0 & 0 & 0 \\ 0 & 1 & -1 & -1 \\ 0 & 0 & 0 & 0 \\ 0 & 0 & 0 & 0
                            \end{bmatrix} = B
                        \end{equation*}
                    \item 取 $F$ 为 $A$ 的 $j_1, \ j_2, \ \dots, \ j_r$ 列构成的列满秩矩阵,$G$ 为 $B$ 的前 $r$ 行构成的行满秩矩阵,那么 $A = FG$.
                        \par $rankB = 2$ 且 $B$ 中的第 $1$ 列和第 $2$ 列为单位矩阵的前两列,故
                        \begin{equation*}
                            A = \begin{bmatrix}
                                1 & -1 \\ -1 & 1 \\ -1 & -1 \\ 1 & 1
                            \end{bmatrix}\begin{bmatrix}
                                1 & 0 & 0 & 0 \\ 0 & 1 & -1 & -1
                            \end{bmatrix}
                        \end{equation*}
                \end{enumerate}
            \end{solution}


    \subsection{矩阵的奇异值分解}
        \subsubsection{矩阵奇异值分解的步骤推导}
            \textbf{奇异值:}设 $A \in C_r^{m \times n}(r > 0)$,$A^HA$ 的特征值为 
            \begin{equation*}
                \lambda_1 \geq \lambda_2 \geq \dots \geq \lambda_r \> \lambda_{r+1} = \dots = \lambda_n = 0
            \end{equation*}
            则称 $\sigma_i = \sqrt{\lambda_i}(i = 1, \ 2, \ \dots, \ n)$ 为 $A$ 的 \textbf{奇异值}. 当 $A$ 为零矩阵时,它的奇异值都是 $0$.
            \par 设 $A \in C_r^{m \times n}(r > 0)$,则存在 $m$ 阶酉矩阵 $U$ 和 $n$ 阶酉矩阵 $V$,使得
            \begin{equation*}
                U^HAV = \begin{bmatrix}
                    \varSigma & O \\ O & O
                \end{bmatrix}
            \end{equation*}
            其中 $\varSigma = diag(\sigma_1, \ \sigma_2, \ \dots, \ \sigma_r)$,而 $\sigma_i(i = 1, \ 2, \ \dots, \ r)$ 为矩阵 $A$ 的全部非零奇异值.
            \par 记 Hermite 矩阵 $A^HA$ 的特征值为
            \begin{equation*}
                \lambda_1 \geq \lambda_2 \geq \dots \geq \lambda_r \> \lambda_{r+1} = \dots = \lambda_n = 0
            \end{equation*}
            存在 $n$ 阶酉矩阵 $V$,使得
            \begin{equation*}
                V^H(A^HA)V = \begin{bmatrix}
                    \lambda_1 & & \\ & \ddots & \\ & & \lambda_n
                \end{bmatrix} = \begin{bmatrix}
                    \varSigma^2 & O \\ O & O
                \end{bmatrix}
            \end{equation*}
            将 $V$ 分块为 
            \begin{equation*}
                V = [V_1 | V_2], \quad V_1 \in C_r^{n \times r}, \ V_2 \in C_{n-r}^{n \times (n-r)}
            \end{equation*}
            这样,可以将上式改写为
            \begin{equation*}
                A^HAV = V \begin{bmatrix}
                    \varSigma & O \\ O & O
                \end{bmatrix}
            \end{equation*}
            则有
            \begin{equation*}
                A^HAV_1 = V_1\varSigma^2, \quad A^HAV_2 = O 
            \end{equation*}
            故有
            \begin{align*}
                V_1^HA^HAV_1 = \varSigma^2 \quad &or \quad (AV_1\varSigma^{-1})(AV_1\varSigma^{-1}) = I_r \\
                (AV_2)^H(AV_2) = O \quad &or \quad AV_2 = O
            \end{align*}
            令 $U_1 = AV_1\varSigma^{-1}$,则 $U_1^HU_1 = I_r$,即 $U_1$ 的 $r$ 个列是两两正交的单位向量,记作 $U_1 = (u_1, \ u_2, \ \dots, \ u_r)$. 将其扩充为 $C^m$ 的标准正交基,记增添的向量为 $u_{r+1}, \ \dots, \ u_m$,并构造矩阵 $U_2 = (u_{r+1}, \ \dots, \ u_m)$,则
            \begin{equation*}
                U = \left[\begin{array}{c:c} U_1 & U_2 \end{array}\right] = (u_1, \ u_2, \ \dots, \ u_r, \ u_{r+1}, \ \dots, \ u_m)
            \end{equation*}
            是 $m$ 阶酉矩阵,且有
            \begin{equation*}
                U_1^HU_1 = I_r, \quad U_2^HU_1 = O
            \end{equation*}
            于是可得
            \begin{equation*}
                U^HAV = U^H \left[\begin{array}{c:c}
                     AV_1 & AV_2 \end{array}\right] = \begin{bmatrix}
                    U_1^H \\ U_2^H
                \end{bmatrix}\left[\begin{array}{c:c}
                    U_1\varSigma & O
                \end{array}\right] = \begin{bmatrix}
                    U_1^HU_1\varSigma & O \\ U_2^HU_1\varSigma & O
                \end{bmatrix} = \begin{bmatrix}
                    \varSigma & O \\ O & O
                \end{bmatrix}
            \end{equation*}
            故有
            \begin{equation*}
                A = U\begin{bmatrix}
                    \varSigma & O \\ O & O 
                \end{bmatrix}V^H
            \end{equation*}
            称上式为矩阵 $A$ 的 \textbf{奇异值分解}

            \par 总结求解过程如下:
            \begin{enumerate}
                \item 求酉矩阵 $V_{n \times n}$,使得
                    \begin{equation*}
                        V^H(A^HA)V = diag(\lambda_1, \ \dots, \ \lambda_r, \ 0, \ \dots, 0)
                    \end{equation*}
                \item 计算 $U_1 = AV_1\varSigma^{-1}$,其中 $V_1$ 为 $V$ 的前 $r$ 列构成的矩阵
                \item 扩充 $U_1$ 的 $r$ 个列向量为 $C^m$ 的标准正交基,并记由增加的 $m-r$ 个列向量构成的矩阵为 $U_2$,那么 $U = \left[\begin{array}{c:c}
                    U_1 & U_2
                \end{array}\right]$ 是酉矩阵
                \item 写出 $A$ 的奇异值分解 $A = U \begin{bmatrix}
                    \varSigma & O \\ O & O 
                \end{bmatrix}V^H$ 
            \end{enumerate}

        \subsubsection{举例展示求法}
            \begin{problem}
                \par 求 $A = \begin{bmatrix}
                    1 & 0 \\ 0 & 1 \\ 1 & 1
                \end{bmatrix}$ 的奇异值分解.
            \end{problem}
            \begin{solution}
                \begin{enumerate}
                    \item 求酉矩阵 $V_{n \times n}$,使得 $V^H(A^HA)V = diag(\lambda_1, \ \dots, \ \lambda_r, \ 0, \ \dots, 0)$
                    \par 计算 
                        \begin{equation*}
                            B = A^TA = \begin{bmatrix}
                                2 & 1 \\ 1 & 2
                            \end{bmatrix}
                        \end{equation*}
                        
                    求得 $B$ 的特征值为 $\lambda_1 = 3, \ \lambda_2 = 1$,对应的特征向量依次为 
                    \begin{equation*}
                        \xi_1 = \begin{bmatrix}
                            1 \\ 1
                        \end{bmatrix}, \quad \xi_2 = \begin{bmatrix}
                            -1 \\ 1
                        \end{bmatrix}
                    \end{equation*}
                    于是可得 
                    \begin{equation*}
                        rankA = 2, \quad \varSigma = \begin{bmatrix}
                            \sqrt{3} & 0 \\ 0 & 1
                        \end{bmatrix}, \quad V = \begin{bmatrix}
                            \frac{1}{\sqrt{2}} & -\frac{1}{\sqrt{2}} \\ \frac{1}{\sqrt{2}} & \frac{1}{\sqrt{2}}
                        \end{bmatrix}
                    \end{equation*}
                    \item 计算 $U_1 = AV_1\varSigma^{-1}$,其中 $V_1$ 为 $V$ 的前 $r$ 列构成的矩阵
                        \par 有 $V_1 = V$,计算
                        \begin{equation*}
                            U_1 = AV_1\varSigma^{-1} = \begin{bmatrix}
                                \frac{1}{\sqrt{6}} & -\frac{1}{\sqrt{2}} \\ \frac{1}{\sqrt{6}} & \frac{1}{\sqrt{2}} \\ \frac{2}{\sqrt{6}} & 0
                            \end{bmatrix}
                        \end{equation*}
                    \item 扩充 $U_1$ 的 $r$ 个列向量为 $C^m$ 的标准正交基,并记由增加的 $m-r$ 个列向量构成的矩阵为 $U_2$,那么 $U = \left[\begin{array}{c:c} U_1 & U_2 \end{array}\right]$ 是酉矩阵
                        \par 取 $U_2 = \begin{bmatrix}
                            -\frac{1}{\sqrt{3}} \\ -\frac{1}{\sqrt{3}} & \frac{1}{\sqrt{3}}
                        \end{bmatrix}$
                        构造正交矩阵
                        \begin{equation*}
                            U = \left[\begin{array}{c:c}
                                U_1 & U_2
                            \end{array}\right] = \begin{bmatrix}
                                \frac{1}{\sqrt{6}} & -\frac{1}{\sqrt{2}} & -\frac{1}{\sqrt{3}} \\ \frac{1}{\sqrt{6}} & \frac{1}{\sqrt{2}} & -\frac{1}{\sqrt{3}} \\ \frac{2}{\sqrt{6}} & 0 & \frac{1}{\sqrt{3}}
                            \end{bmatrix}
                        \end{equation*}
                    \item 写出 $A$ 的奇异值分解 $A = U \begin{bmatrix}
                        \varSigma & O \\ O & O 
                    \end{bmatrix}V^H$ 
                    \begin{equation*}
                        A = U\begin{bmatrix}
                            \sqrt{3} & 0 \\ 0 & 1 \\ 0 & 0
                        \end{bmatrix}V^T
                    \end{equation*}
                \end{enumerate} 
            \end{solution}

    \subsection{利用矩阵分解求矩阵广义逆}
        \subsubsection{矩阵广义逆介绍}
            \par 逆矩阵的概念只是对可逆矩阵才有意义. 但是在实际问题中,遇到的矩阵不一 定是方阵,即便是方阵也不一定可逆,这就需要考虑,可否将逆矩阵概念进一步推 广,为此,引进下列条件:
            \begin{enumerate}
                \item 该矩阵对于不可逆矩阵甚至长方矩阵都存在;
                \item 它具有通常逆矩阵的一些性质;
                \item 当矩阵可逆时,它还原到通常的逆矩阵.
            \end{enumerate}
            称满足以上三个条件的矩阵为 \textbf{广义逆矩阵}.
            \\
            \par \textbf{Penrose 的广义逆矩阵:} 设矩阵 $A \in C^{m \times n}$,若矩阵 $X \in C^{n \times m}$ 满足以下 $4$ 个 Penrose 方程
            \begin{table}[htbp]
                \centering
                \begin{tabular}{ll}
                    (1)$AXA = A$; & (2)$XAX = X$; \\
                    (3)$(AX)^H = AX$; & (4)$(XA)^H = XA$ \\
                \end{tabular}
            \end{table}
            则称 $X$ 为 $A$ 的 \textbf{Moore-Penrose 逆},记为 $A^{+}$.
            \par 易知一下特例:
            \begin{enumerate}
                \item 若 $A$ 是可逆矩阵,则 $A^{+} = A^{-1}$;
                \item 若 $A = O_{m \times n}$,则 $A^{+} = O_{n \times m}$;
                \item 若 $A = \begin{bmatrix}
                    1 \\ 1
                \end{bmatrix}$,则 $A^{+} = \begin{bmatrix}
                    \frac{1}{2} & \frac{1}{2}
                \end{bmatrix}$
            \end{enumerate}
            \par \textbf{矩阵 Moore-Penrose 逆的存在性:}
            \begin{enumerate}
                \item 对于任意 $A \in C^{m \times n}$,$A^{+}$ 存在并且唯一;
                \item 设 $A \in C_r^{m \times n}$ 的不可逆值分解为 
                    \begin{equation*}
                        A = U \begin{bmatrix}
                            \varSigma_r & O \\ O & O 
                        \end{bmatrix}_{m \times n} V^H
                    \end{equation*}
                    那么 
                    \begin{equation*}
                        A^{+} = V \begin{bmatrix}
                            \varSigma_r^{-1} & O \\ O & O
                        \end{bmatrix}_{n \times m} U^H
                    \end{equation*}
            \end{enumerate}

        \subsubsection{利用矩阵满秩分解求矩阵广义逆}
            \par 通过证明上述存在性定理一,我们可以完成利用矩阵满秩分解求矩阵广义逆的推导.
            \par 设 $rankA = r$. 若 $r = 0$,则 $A = O_{m \times n}$,则 $A^{+} = O_{n \times m}$;若 $r > 0$,由矩阵满秩分解的存在性定理(设 $A \in C_r^{m \times n}(r > 0)$,则 $A$ 有满秩分解式),$A$ 可进行满秩分解:
            \begin{equation*}
                A = FG \quad (F \in C_r^{m \times r}, \ G \in C_r^{r \times n})
            \end{equation*}
            令 $X = G^{+}F^{+}$,则有 
            \begin{align*}
                AXA &= FG \cdot G^{+}F^{+} \cdot FG = FG = A \\ XAX &= G^{+}F^{+} \cdot FG \cdot G^{+}F^{+} = G^{+}F^{+} = X \\ (AX)^H &= (FG \cdot G^{+}F^{+})^H = (FF^{+})^H = FF^{+} = F \cdot GG^{+} \cdot F^{+} = AX \\ (XA)^H &= (G^{+}F^{+} \cdot FG)^H = (G^{+}G)^H = G^{+}G = G^{+} \cdot F^{+}F \cdot G = XA
            \end{align*}
            故
            \begin{equation*}
                A^{+} = G^{+}F^{+} = G^H(F^HAG^H)^{-1}F^H
            \end{equation*}
            \\
            \par 由上式我们可以总结出求矩阵的 Moore-Penrose 逆的满秩分解方法:
            \begin{enumerate}
                \item 求 $A$ 的满秩分解 $A = FG$
                \item 计算 $F^{+} = (F^HF)^{-1}F^H$ 和 $G^{+} = G^H(GG^H)^{-1}$
                \item 计算 $A^{+} = G^{+}F^{+}$
            \end{enumerate}

        \subsubsection{利用矩阵奇异值分解求矩阵广义逆}
            \par 由上述存在性定理二,我们可以得出利用矩阵奇异值分解求矩阵广义逆的推导:
            \par 令 $\varSigma = \begin{bmatrix}
                \varSigma_r & O \\ O & O
            \end{bmatrix}$,则 $\varSigma^{+} = \begin{bmatrix}
                \varSigma_r^{-1} & O \\ O & O
            \end{bmatrix}$
            \begin{align*}
                A^{+} &= A^H(AA^H)^{+} = (U\varSigma V^H)^H(U\varSigma V^HV\varSigma^HU^H) \\ &= V\varSigma^HU^H[(U\varSigma\varSigma^HU^H)^{+} \\ &= V[(U\varSigma)^H)(U\varSigma(U\varSigma)^H)^{+}] \\ &= V(U\varSigma)^{+} = V\varSigma^{+}U^H
            \end{align*}
            \par 由此可以总结出求矩阵的 Moore-Penrose 逆的奇异值分解方法:
            \begin{enumerate}
                \item 求 $A$ 的奇异值分解 $A = U\begin{bmatrix}
                    \varSigma_r & O \\ O & O
                \end{bmatrix}_{m \times n}V^H$
                \item 计算 $A^{+} = V\begin{bmatrix}
                    \varSigma_r^{-1} & O \\ O & O
                \end{bmatrix}_{n \times m}U^H$
            \end{enumerate}
            \par 矩阵的 Moore-Penrose 逆是一种广义逆矩阵,它满足 $4$ 个 Penrose 方程.下面介绍满足一个或几个 Penrose 方程的广义逆矩阵:
            \par 设矩阵 $A \in C^{m \times n}$,矩阵 $X \in C^{n \times m}$
            \begin{enumerate}
                \item 若 $X$ 满足 Penrose 方程中的第 $(i)$ 个方程,则称 $X$ 为 $A$ 的 $\{i\}$-逆,记作 $A^{(i)}$,全体 $\{i\}$-逆 的集合记作 $A\{i\}$. 这种广义逆矩阵共有 $4$ 类;
                \item 若 $X$ 满足 Penrose 方程中的第 $(i), \ (j)$ 个方程 $(i \ne j)$,称 $X$ 为 $A$ 的 $\{i,j\}$-逆,记作 $A^{(i,j)}$,全体 $\{i, j\}$-逆 的集合记作 $A\{i,j\}$. 这种广义逆矩阵共有 $6$ 类;
                \item 若 $X$ 满足 Penrose 方程中的第 $(i), \ (j),(k)$ 个方程 $(i, \ j, \ k \ \textbf{互异})$, 称 $X$ 为 $A$ 的 $\{i,j,k\}$-逆,记作 $A^{(i,j,k)}$,全体 $\{i,j,k\}$-逆 的集合记作 $A\{i,j,k\}$. 这种广义逆矩阵共有 $4$ 类;
            \end{enumerate}
            \par 其中,应用较为广泛的广义逆矩阵有以下 $5$ 种:
            \begin{equation*}
                A\{1\}, \quad A\{1,2\}, \quad A\{1,3\}, \quad A\{1,4\}, \quad A^{+}
            \end{equation*}
            \par 由于任意矩阵 $A \in C_r^{m \times n}(r > 0)$ 都可通过初等行变换化为(拟) Hermite 标准形 $B$,即存在有限个初等矩阵的乘积,记作 $Q$,使得 $QA = B$. 根据矩阵 $B$,构造置换矩阵(交换单位矩阵的列向量构成的矩阵) $P$,使得
            \begin{equation*}
                QAP = \begin{bmatrix}
                    I_r & K \\ O & O
                \end{bmatrix}
            \end{equation*}
            其中 $K$ 是 $r \times (n-r)$ 子矩阵.
            \par 利用矩阵的(拟) Hermite 标准形,容易求得矩阵的 $\{1\}$-逆 和 $\{1,2\}$-逆.
            \\
            \par 由上式可知
            \begin{equation*}
                QAP = \begin{bmatrix}
                    I_r & K \\ O & O
                \end{bmatrix}, \quad A = Q^{-1}\begin{bmatrix}
                    I_r & K \\ O & O
                \end{bmatrix}P^{-1}
            \end{equation*}
            则有 
            \begin{equation*}
                X = P\begin{bmatrix}
                    I_r & O \\ O & L
                \end{bmatrix}Q, \quad X = P\begin{bmatrix}
                    I_r & O \\ O & O
                \end{bmatrix}Q
            \end{equation*}
            分别为 $A$ 的 $\{1\}$-逆 和 $\{1,2\}$-逆.
            \\
            \par 总结方法如下:
            \begin{enumerate}
                \item $\left[\begin{array}{c:c}
                    A & I
                \end{array}\right] \stackrel{\text{行}}{\longrightarrow} \left[\begin{array}{c:c}
                    B & Q
                \end{array}\right]$,其中 $B$ 为 Hermite 标准形矩阵
                \item 构造置换矩阵 $P$,使得 $BP = \begin{bmatrix}
                    I_r & K \\ O & O
                \end{bmatrix}_{m \times n}$
                \item 计算 $X = P\begin{bmatrix}
                    I_r & O \\ O & L
                \end{bmatrix}_{n \times m}Q$ 或 $X_0 = P\begin{bmatrix}
                    I_r & O \\ O & O
                \end{bmatrix}_{n \times m}Q$
                那么 $X \in A\{1\}, \ X_0 \in A\{1,2\}$.
            \end{enumerate}
        \subsubsection{举例展示求法}
            \begin{problem}
                \par 设 $A = \begin{bmatrix}
                    2 & 1 & 0 & 2 \\ 0 & 0 & 1 & 2 \\ 2 & 1 & 1 & 4
                \end{bmatrix}$,分别求 $A^{(1)}, A^{(1,2)}$ 及 $A^{+}$.
            \end{problem}
            \begin{solution}
                \begin{enumerate}
                    \item $\left[\begin{array}{c:c}
                        A & I
                    \end{array}\right] \stackrel{\text{行}}{\longrightarrow} \left[\begin{array}{c:c}
                        B & Q
                    \end{array}\right]$ 
                    \begin{equation*}
                        \left[\begin{array}{c:c}
                            A & I
                        \end{array}\right] \stackrel{\text{行}}{\longrightarrow} \begin{array}{cccc:ccc}
                            2 & 1 & 0 & 2 & 1 & 0 & 0 \\ 0 & 0 & 1 & 2 & 0 & 1 & 0 \\ 0 & 0 & 0 & 0 & -1 & -1 & -1
                        \end{array}
                    \end{equation*} 故 $A$ 的拟 Hermite 标准形为 
                    \begin{equation*}
                        B = \begin{bmatrix}
                            2 & 1 & 0 & 2 \\ 0 & 0 & 1 & 2 \\ 0 & 0 & 0 & 0
                        \end{bmatrix}
                    \end{equation*}
                    \item 构造置换矩阵 $P$,使得 $BP = \begin{bmatrix}
                        I_r & K \\ O & O
                    \end{bmatrix}_{m \times n}$
                    \par 构造可逆矩阵 
                    \begin{equation*}
                        Q = \begin{bmatrix}
                            1 & 0 & 0 \\ 0 & 1 & 0 \\ -1 & -1 & 1
                        \end{bmatrix}, \quad P = (e_2, \ e_3, \ e_1, \ e_4) = \begin{bmatrix}
                            0 & 0 & 1 & 0 \\ 1 & 0 & 0 & 0 \\ 0 & 1 & 0 & 0 \\ 0 & 0 & 0 & 1
                        \end{bmatrix}
                    \end{equation*}
                    \item 计算 $X = P\begin{bmatrix}
                        I_r & O \\ O & L
                    \end{bmatrix}_{n \times m}Q$ 或 $X_0 = P\begin{bmatrix}
                        I_r & O \\ O & O
                    \end{bmatrix}_{n \times m}Q$
                    \begin{equation*}
                        A^{(1)} = P\left[\begin{array}{cc:c}
                            1 & 0 & 0 \\ 0 & 1 & 0 \\ \hdashline 0 & 0 & a \\ 0 & 0 & b
                        \end{array}\right]Q = \begin{bmatrix}
                            -a & -a & a \\ 1 & 0 & 0 \\ 0 & 1 & 0 \\ -b & -b & b
                        \end{bmatrix}, \quad A^{(1,2)} = \begin{bmatrix}
                            0 & 0 & 0 \\ 1 & 0 & 0 \\ 0 & 1 & 0 \\ 0 & 0 & 0
                        \end{bmatrix}
                    \end{equation*}
                \end{enumerate}
                \par 下面求 $A^{+}$
                \begin{enumerate}
                    \item 求 $A$ 的满秩分解 $A = FG$
                        \par 因为拟 Hermite 标准形 $B$ 的第 $2$ 列和第 $3$ 列构成 $I_3$ 的前两列,所以 $F$ 为 $A$ 的第 $2$ 列和第 $3$ 列构成的 $3 \times 2$ 矩阵,从而有 
                        \begin{equation*}
                            A = \begin{bmatrix}
                                1 & 0 \\ 0 & 1 \\ 1 & 1
                            \end{bmatrix}\begin{bmatrix}
                                2 & 1 & 0 & 2 \\ 0 & 0 & 1 & 2
                            \end{bmatrix} = FG
                        \end{equation*}
                    \item 计算 $F^{+} = (F^HF)^{-1}F^H$ 和 $G^{+} = G^H(GG^H)^{-1}$
                        \begin{align*}
                            F^{+} &= (F^HF)^{-1}F^H = \frac{1}{3}\begin{bmatrix}
                                2 & -1 & 1 \\ -1 & 2 & 1
                            \end{bmatrix} \\ G^{+} &= G^H(GG^H)^{-1} = \frac{1}{29}\begin{bmatrix}
                                10 & -8 \\ 5 & -4 \\ -4 & 9 \\ 2 & 10
                            \end{bmatrix}
                        \end{align*} 
                    \item 计算 $A^{+} = G^{+}F^{+}$
                        \begin{equation*}
                            A^{+} = G^{+}F^{+} = \frac{1}{87}\begin{bmatrix}
                                28 & -26 & 2 \\ 14 & -13 & 1 \\ -17 & 22 & 5 \\ -6 & 18 & 12
                            \end{bmatrix}
                        \end{equation*}
                \end{enumerate}
            \end{solution}