\documentclass[UTF8, 12pt, a4paper, oneside]{ctexart}
\usepackage{listing}

\usepackage{amsmath}
\usepackage{amsfonts}
\usepackage{geometry}
\usepackage{ulem}
\usepackage[most]{tcolorbox}
\usepackage[hidelinks]{hyperref}
\usepackage{multicol}
\usepackage{color}
\usepackage{xcolor}
\usepackage{framed}
\usepackage{mdframed}
\usepackage{mathtools,amssymb}
\usepackage{bm}
\usepackage{enumitem}
\usepackage{titlesec}
\usepackage{caption}
\usepackage{fancyhdr}
\usepackage{arydshln}
%\usepackage{draftwatermark}         % 所有页加水印
%\usepackage[firstpage]{draftwatermark} % 只有第一页加水印
%\SetWatermarkText{傻逼优诺我星野爱和你爆了}
%\SetWatermarkText{\includegraphics{fig/texlion.png}}         % 设置水印logo
%\SetWatermarkLightness{0.8}             % 设置水印透明度 0-1
%\SetWatermarkScale{0.5}                   % 设置水印大小 0-1 

\definecolor{backcolour}{rgb}{0.18,0.18,0.18}
\definecolor{codegray}{rgb}{0.93,0.93,0.93}
\definecolor{codegreen}{rgb}{0.34,0.76,0.34}
\definecolor{codeblue}{rgb}{0.27,0.35,0.69}
\definecolor{codepurple}{rgb}{0.69,0.25,0.82}
\definecolor{codeorange}{rgb}{0.8,0.6,0.2}
\definecolor{codered}{rgb}{0.8,0.25,0.33}

\titleformat{\section}
    {\fontsize{16pt}{16pt}\bfseries\centering}
    {第 \thesection 章} % 标签部分为空
    {0.5em} % 标题与正文的距离
    {} % 标题为空
    
\titleformat{\subsection}
    {\fontsize{14pt}{16.8pt}\bfseries}
    {\thesubsection} % 标签部分为空
    {0.5em} % 标题与正文的距离
    {} % 标题为空 

\titleformat{\subsubsection}
    {\fontsize{12pt}{14pt}\bfseries}
    {\thesubsubsection} % 标签部分为空
    {0.5em} % 标题与正文的距离
    {} % 标题为空 

\titlespacing{\subsubsection}
  {\parindent}{3.25ex plus 1ex minus .2ex}{1em}

\pagestyle{fancy}
\fancyhf{} % 清除默认页眉页脚
\fancyfoot[C]{\thepage} % 页脚居中显示页码

\lstset{
    backgroundcolor     =   \color{codegray},
    basicstyle          =   \sffamily,          % 基本代码风格
    keywordstyle        =   \bfseries,          % 关键字风格
    commentstyle        =   \rmfamily\itshape,  % 注释的风格,斜体
    stringstyle         =   \ttfamily,  % 字符串风格
    flexiblecolumns,                % 别问为什么,加上这个
    numbers             =   left,   % 行号的位置在左边
    showspaces          =   false,  % 是否显示空格,显示了有点乱,所以不现实了
    numberstyle         =   \zihao{-5}\ttfamily,    % 行号的样式,小五号,tt等宽字体
    showstringspaces    =   false,
    captionpos          =   t,      % 这段代码的名字所呈现的位置,t指的是top上面
    aboveskip           =   1em,
    belowskip           =   1em,
    frame               =   lrtb,   % 显示边框
}

\lstdefinestyle{C++}{
    language        =   C++, % 语言选Python
    basicstyle      =   \zihao{-5}\ttfamily,
    numberstyle     =   \zihao{-5}\ttfamily,
    keywordstyle    =   \color{codepurple},
    keywordstyle    =   [2] \color{codegray},
    stringstyle     =   \color{magenta},
    commentstyle    =   \color{codegreen}\ttfamily,
    breaklines      =   true,   % 自动换行,建议不要写太长的行
    columns         =   fixed,  % 如果不加这一句,字间距就不固定,很丑,必须加
    basewidth       =   0.5em,
}

\lstdefinestyle{C}{
    language        =   C, % 语言选Python
    basicstyle      =   \zihao{-5}\ttfamily,
    numberstyle     =   \zihao{-5}\ttfamily,
    keywordstyle    =   \color{codepurple},
    keywordstyle    =   [2] \color{codegray},
    stringstyle     =   \color{magenta},
    commentstyle    =   \color{codegreen}\ttfamily,
    breaklines      =   true,   % 自动换行,建议不要写太长的行
    columns         =   fixed,  % 如果不加这一句,字间距就不固定,很丑,必须加
    basewidth       =   0.5em,
}

\newlist{choices}{enumerate}{2}
\setlist[choices,1]{
  label = \Alph*.,       % 设置标签格式为大写字母后跟括号,例如 A)、B) 等
  leftmargin = *,        % 设置左边距,使其与上层列表对齐
  align = left,          % 设置标签对齐方式为左对齐
  labelwidth = 1.5em,    % 设置标签宽度
  itemsep = 0.5em        % 设置列表项之间的距离
}

\newmdenv[
  leftmargin=0cm,
  rightmargin=0cm,
  skipabove=1em,
  skipbelow=1em,
  backgroundcolor=gray!20,
  linewidth=0pt
]{mquote}

\usepackage{paralist}
\let\itemize\compactitem
\let\enditemize\endcompactitem
\let\enumerate\compactenum
\let\endenumerate\endcompactenum
\let\description\compactdesc
\let\enddescription\endcompactdesc

\definecolor{shadecolor}{RGB}{241, 241, 255}
\geometry{left=2.54cm,right=2.54cm,top=3.18cm,bottom=3.18cm}
\linespread{1.5}

\newcounter{problemname}[subsubsection]
%\newenvironment{problem}{\begin{shaded}\stepcounter{problemname}\par\noindent\textbf{题目\arabic{problemname}. }}{\end{shaded}\par}
\newenvironment{problem}{\stepcounter{problemname}\par\noindent\textbf{题目. }\par}

%\newenvironment{problem}{\begin{shaded}\par\noindent\textbf{题目. }}{\end{shaded}\par}
\newenvironment{solution}{\par\noindent\textbf{解答. }}{\par}
\newenvironment{note}{\par\noindent\textbf{题目\arabic{problemname}的注记. }}{\par}

\begin{document}
	\sloppy
	\thispagestyle{empty}
    \begin{figure}[t]
		\centering
		\includegraphics[width=13cm]{logo1.jpg}
	\end{figure}

    \begin{center}
        \Huge\textbf{期末课程论文}
    \end{center}
	\vspace*{1em}
    \begin{figure}[htbp]
		\centering
		\includegraphics[width=3cm]{logo2.jpg}
	\end{figure}
		\begin{center}
			\Large\textbf{题目:} \underline{\textbf{矩阵函数的求法和矩阵分解方法研究}} 
		\end{center}
    \vspace*{3em}
	\begin{table}[htbp]
		\hspace*{8em}
		\large
		\begin{tabular}{ll}
		\textbf{姓 \hspace*{1.3em}  名:} & \underline{胡宇杭} \\
		\textbf{学 \hspace*{1.3em}  院:} & \underline{计算机学院(国家示范性软件学院)} \\
        \textbf{专 \hspace*{1.3em}  业:} & \underline{计算机类} \\
        \textbf{班 \hspace*{1.3em}  级:} & \underline{2022211320} \\
        \textbf{学 \hspace*{1.3em}  号:} & \underline{2022212408} \\
        \textbf{指导教师:} & \underline{李昊辰} \\
		\end{tabular}
	\end{table}
    \vspace*{2em}
    \begin{center}
        \Large\textbf{\today}
    \end{center}
    \setcounter{page}{1}
	\section{实验内容和实验环境描述}
    \subsection{实验任务内容和目的}
        \begin{itemize}
            \item 利用所学数据链路层原理,自己设计一个滑动窗口协议,在仿真环境下编程实现有噪音信道环境下两站点之间无差错双工通信
            \item 实现有噪音信道环境下的无差错传输,充分利用传输信道的带宽
            \item 实现搭载 ACK 技术的 GBN 和 选择重传协议
            \item 实现 ACK 计时器、捎带确认、NAK 等补充功能
            \item 根据信道实际情况合理地为协议配置工作参数,包括滑动窗口的大小和重传定时器时限以及ACK搭载定时器的时限
            \item 进一步巩固和深刻理解数据链路层误码检测的 CRC 校验技术,以及滑动窗口的工作机理。
        \end{itemize}

    \subsection{实验环境}
        \par 本次实验在 WINDOWS 11 下进行,使用 CLion 作为 IDE,gcc 作为编译工具,Ninja 作为生成器
        \begin{itemize}
            \item 系统版本:WINDOWS 11
            \item 编译器:gcc 8.1.0
            \item 生成器:Ninja
            \item IDE:CLion
        \end{itemize}


    \thispagestyle{empty}
\section*{Research on Matrix Function Calculation and Matrix Decomposition Methods}
\fontsize{12pt}{14pt}\selectfont
    \begin{center}
        \textbf{\large Abstract}
    \end{center}
    \par The main content of this paper is a comprehensive summary of existing matrix function calculation and matrix decomposition techniques, proposing an integrated solution. It focuses on the theoretical foundations and practical applications of different matrix function solutions and decomposition methods, while comparing their advantages and limitations. Additionally, the paper delves into the implementation and optimization of these methods in computer programs, demonstrating how to use them to solve real-world problems. These research findings offer valuable references for matrix theory and its application in scientific computing.
    \\
    \\
    \textbf{KEYWORDS} \quad Matrix Function Calculation \quad Matrix Decomposition Methods \quad Program implementation
    \pagenumbering{Roman}
	\tableofcontents
    \newpage
    \pagenumbering{arabic}
    \section{相关技术}
    \fontsize{10.5pt}{12.6pt}\selectfont
    \songti
    \par 基数估计定义:给定 \textbf{SQL} 查询语句,数据库 $D$,其表示在数据库 $D$ 中执行查询 $q$ 返回的结果行数,记为 $C(q|D)=|D'|$,其中 $D'$表示查询结果。
    \par 数据库优化查询系统中,基数估计发挥了重要作用。基数估计技术根据是否使用机器学习算法分为了传统的基数估计技术和基于机器学习的基数估计技术两类。传统的基数估计一般采用统计的方法,其核心是使用某种数据结构(例如直方图,数据画像)来拟合表上的数据分布。基于直方图的算法\cite{ref3}根据边界 $[b_1, b_2, \dots, b_n]$ 将列数据划分成若干份,并且统计如下信息,一个是该属性位于 $b_{i-1}$ 和 $b_i$ 之间的元素行数,另一个是位于此范围不同的元素行数,这种直方图间隔也被称作是分桶。其中J.Xu\cite{ref4} 等人利用直方图的方式对表中的数据分布进行查询,此方法会导致基数估计偏低,任意数据分布可以通过哈希函数得到一个均匀分布。
    \par 基于数据画像的统计方法,它的核心思想是使用位图来记录元素的出现情况,从而在降低计算成本的同时提供对不同元素数量的估计。这种方法适用于大规模数据集,其中直接计算精确的基数可能变得非常昂贵。与传统的基数估计算法相比,基于数据画像的方法通常更节省内存,因为它们不需要存储实际的元素,而是使用位图记录元素的出现情况。然而这种方法是一种概率性估计,结果可能受到哈希冲突等因素的影响\cite{ref2}。
    \par 基于统计的基数估计技术适用于拟合单列的数据分布,而在处理任意多列组合上数据之间的复杂关系,其能力较弱\cite{ref1}。
    \par 还有一类基于线性映射的基数估计方法,包括线性计数\cite{ref6}和布隆过滤\cite{ref7}.基本思想是使用线性哈希函数将数据均匀映射到位图上,根据位图上每个位置被访问到的次数,利用极大似然对基数进行估计。后者为了在有限的空间中减少哈希结果碰撞,使用了多个独立的哈希函数,每个元素可以被映射到固定数量的位置上,布隆过滤的方法已经被广泛使用\cite{ref2}。
    \par 采样是一种能够替代基于统计法的基数估计方法.它不依赖于特定假设,能够发现一些偶然的关联从而获得更加准确的估计\cite{ref2}。基于采样的基数估计技术在大规模数据表的复杂查询场景中需要消耗大量空间存储采样的元组,同时有效采样元组会随着多表复杂的连接而减少,损失估计的性能。
    \par 综上所述,传统的基数估计技术计算通常需要存储直方图,位图,采样的元组等信息,这会占用较大的存储空间,并且可能难以适应数据的动态变化。而基于机器学习的基数估计技术则是利用机器学习或深度学习的方法来学习数据的分布和查询的特征,从而预测基数,需要存储学习映射函数 $f(\cdot)$ 的模型\cite{ref1}。相较于传统基数估计基数,其模型占用空间小,这种技术的优点是可以更好地拟合数据的复杂分布和查询的复杂关系,从而提高基数估计的准确度。本文采取的基于机器学习的数据库基数估计,成功引入了多层感知机模型,可进一步提升基数估计精度,减少空间占用,加强拟合复杂数据关系能力。
    \vfill
    \section{实验内容}
    \subsection{实验内容一:结构体及文件基本操作}
        \textbf{要求:}
        \begin{itemize}
            \item \textbf{基本要求:}定义一个结构数组,用于保存学生信息,数据项包括:学号、姓名、年龄;
            \item \textbf{功能项1:}从键盘输入学生信息存入结构数组,可按照学号依次输入从本人开始的 \textbf{\textit{5}} 名同学信息;
            \item \textbf{功能项2:}输入命令可将结构数组中的数据将保存在 \textbf{\textit{input.dat}} 文件中(在文件中以二进制或文本存储信息均可);
            \item \textbf{功能项3:} 输入命令可将信息从 \textbf{\textit{input.dat}} 文件中读入到结构数字,然后反向顺序输出到 \textbf{\textit{output.dat}} 文件中(在文件中以二进制或文本存储信息均可);
            \item \textbf{附加要求:}设置条件断点,在执行到处理第 \textbf{\textit{5}} 位学生信息时中断。
        \end{itemize}
    \subsection{实验内容二:算法执行效率测量与分析}
        \textbf{要求:}
        \begin{itemize}
            \item 编写程序分别调用下述两个函数 \textbf{\textit{copyij}} 和 \textbf{\textit{copyji}};
            \item 统计两个子程序的运行时间(绝对时间);
            \item 比较两个子程序运行绝对时间上的差异,试分析成因;
            \item 给出两个算法的时间复杂度,说明差异;
            \item \textbf{附加要求:}编译调试时采用优化和非优化方式进行分别进行编译测试,观察启用编译优化后是否能带来性能变化。
        \end{itemize}
        \begin{figure*}[htbp]
            \centering
            \includegraphics*[width = 14cm]{program1.jpg}
        \end{figure*}
    \section{实验步骤}
    \textbf{操作步骤+运行截图}
    \subsection{代码解释}
        \par 由于本次实验要求实现的功能繁多,因此采用分文件编写的方法,下面对每一个模块进行解释说明。
        \subsubsection{LINKSTRING}  
            \par \textbf{LinkString} 用于存储每一行的字符串,由于代码文本编辑器需要频繁的进行插入和删除的功能,同时为了节省空间,这里采用不带头结点的链串进行实现。
            \begin{figure*}[htbp]
                \includegraphics*[width = 12cm]{ls_1.png}
            \end{figure*}
            \newpage
            \par 下面对每一个函数进行说明:
            \begin{enumerate}
                \item \textbf{LinkString* init\_linkstring(void)}
                    \par 在初始化时,我们在堆区分配一块内存,并将头结点设为 \textbf{NULL} (因为是无头结点的链串),并初始化 \textbf{length} 为 \textbf{0}。
                    \par 由于所有操作都是在常数时间内完成的,因此时间复杂度为 \textbf{O(1)}
                    \begin{figure*}[htbp]
                        \includegraphics*[width = 12cm]{ls_2.png}
                    \end{figure*}
                \item \textbf{void destory\_linkstring(LinkString* ls)}
                    \par 该函数实现了销毁链串的功能,我们一次遍历每一个结点,将其释放掉在移动至下一个结点,最后将链串本身释放掉。
                    \par 函数需要遍历一次链串,因此时间复杂度为 \textbf{O(n)}
                    \begin{figure*}[htbp]
                        \includegraphics*[width = 8cm]{ls_3.png}
                    \end{figure*}
                \item \textbf{bool insert\_substring(LinkString* ls, int pos, const char* substr)}
                    \par 该函数实现了在特定位置插入子串的功能,同时做了特判,在输入的 \textbf{pos} 大于链串长度时,默认从串尾进行插入操作。在函数中,我们首先迭代至起始位置,同时由于链串不带头结点,因此需要利用 \textbf{prev} 记录当前结点的前驱结点。在插入操作时,我们需要查看前驱结点 \textbf{prev} 是否为空指针。如果为空指针,说明我们需要将结点插入至队头位置。
                    \par 函数首先迭代至指定插入的结点,这部分的时间复杂度为 \textbf{O(n)},然后进行子串的插入操作,该过程需要遍历子串,故时间复杂度为 \textbf{O(m)}。因此,总的时间复杂度为 \textbf{O(n + m)}。
                    \begin{figure*}[htbp]
                        \includegraphics*[width = 10cm]{ls_4.png}
                    \end{figure*}
                \newpage
                \item \textbf{bool delete\_substring(LinkString* ls, int pos, int len)}
                    \par 该函数实现了在特定位置删除指定长度子串的功能,与插入类似,我们对 \textbf{len} 越界的情况做了特殊处理,并且利用 \textbf{prev} 储存前驱结点,防止因当前结点为队头而错误地将队头指针变为野指针。
                    \par 函数的基本实现与插入函数类似,都需要迭代至需要删除的位置并删除指定长度的子串,因此时间复杂度也为 \textbf{O(n + m)}
                    \begin{figure*}[htbp]
                        \includegraphics*[width = 10cm]{ls_5.png}
                    \end{figure*}
                \newpage
                \item \textbf{char* get\_linkstring(LinkString* ls)}
                    \par 该函数实现了遍历整个链串,并将其中元素提取出来,最后返回一个 \textbf{C} 的字符串的功能,同时使用了 \textbf{malloc} 开辟内存,防止链串长度超出 \textbf{res} 的最大长度导致截断。
                    \par 函数需要遍历整个链串,因此时间复杂度为 \textbf{O(n)}。
                    \begin{figure*}[htbp]
                        \includegraphics*[width = 9cm]{ls_6.png}
                    \end{figure*}
                \item \textbf{int find\_substring(LinkString* ls, const char* pattern)}
                    \par 该函数实现了使用 \textbf{KMP} 算法(\textbf{next} 数组初始化 \textbf{0} 的版本)完成了模式串与主串匹配的过程。
                    \par 我们知道 \textbf{KMP} 算法只需遍历一次主串就能实现查找功能,其中,\textbf{next} 数组的求解还需要遍历一次模式串,因此总的时间复杂度为 \textbf{O(n + m)}。
                    \begin{figure*}[htbp]
                        \includegraphics*[width = 9cm]{ls_7.png}
                    \end{figure*}
            \end{enumerate}
        \subsubsection{LINKLIST}
            \par 将\textbf{LinkString} 封装进 \textbf{LinkList},这样我们就实现了最基本的文本存储访问功能:通过 \textbf{LinkList} 访问指定的行,然后通过 \textbf{LinkString} 访问指定的列。
            \begin{figure*}[htbp]
                \includegraphics*[width = 12cm]{ll_1.png}
            \end{figure*}
            \par 下面对每一个函数进行说明:
            \begin{enumerate}
                \item \textbf{LinkList* init\_linklist(void)}
                    \par 该函数的实现逻辑与链串的初始化一致,再次不再赘述。
                    \par 时间复杂度 \textbf{O(1)}
                    \begin{figure*}[htbp]
                        \includegraphics*[width = 11cm]{ll_2.png}
                    \end{figure*}
                \item \textbf{LNode* get\_row(LinkList* ll, int row\_pos)}
                    \par 该函数用于通过指定的 \textbf{pos} 找到相应的结点。
                    \par 函数需要迭代至指定位置,因此时间复杂度为 \textbf{O(n)}
                    \begin{figure*}[htbp]
                        \includegraphics*[width = 11cm]{ll_4.png}
                    \end{figure*}
                \item \textbf{void destory\_linklist(LinkList* ll)}
                    \par 与 \textbf{LinkString} 类似,该函数用于销毁 \textbf{LinkList},在此不再赘述。
                    \par 时间复杂度 \textbf{O(n)}
                    \begin{figure*}[htbp]
                        \includegraphics*[width = 12cm]{ll_3.png}
                    \end{figure*}
                \item \textbf{bool insert\_linklist(LinkList* ll, int row\_pos, int col\_pos, const char* s)}
                    \par 该函数实现了在指定位置插入行的功能,同时在函数中我们调用了之前的 \textbf{insert\_substring()} 函数完成插入指定字符串的功能,防止了代码冗余的现象。
                    \par 函数需要迭代至指定位置,这部分的时间复杂度为 \textbf{O(n)},根据之前的分析,在链串中插入字符串的时间复杂度为 \textbf{O(n + m)(n = 0)} $=$ \textbf{O(m)}。因此总的时间复杂度为 \textbf{O(n + m)}。
                    \begin{figure*}[htbp]
                        \includegraphics*[width = 12cm]{ll_5.png}
                    \end{figure*}
                \newpage
                \item \textbf{bool delete\_linklist(LinkList* ll, int pos, int len)}
                    \par 该函数与插入类似,都调用了 \textbf{LinkString} 中的函数,在此不再赘述。
                    \par 时间复杂度 \textbf{O(n + m)}
                    \begin{figure*}[htbp]
                        \includegraphics*[width = 10cm]{ll_6.png}
                    \end{figure*}
            \end{enumerate}
       \subsubsection{OPRTSTACK}
            \par \textbf{OprtStack} 用于实现附加要求中的运算符匹配算法,由于之前没写过数组版本的栈,所以在这就不再写链栈了(\sout{绝对不是因为懒})。
            \begin{figure*}[htbp]
                \includegraphics*[width = 8cm]{os_1.png}
            \end{figure*}
            \begin{enumerate}
                \item \textbf{OpStack* init\_stack(void)}
                    \par 初始化时为栈开辟指定的内存空间,由于使用动态数组实现,因此初始化 \textbf{top} $=$ \textbf{-1} 代表一开始栈是空的。
                    \begin{figure*}[htbp]
                        \includegraphics*[width = 9cm]{os_2.png}
                    \end{figure*}
                \item \textbf{bool extend\_stack(OpStack* s)}
                    \par 该函数用于实现在栈满时对栈容量的扩展,保证不会因为栈的容量问题导致判断运算符失败。为了防止 \textbf{realloc} 函数开辟空间失败返回空指针导致原空间无法访问造成内存泄漏,这里创建临时变量接受新的内存地址。
                    \par 由于 \textbf{realloc} 函数并不是在原有内存空间上进行扩展,而是开辟一个新的空间,并将原有数据拷贝过去,因此时间复杂度为 \textbf{O(n)}
                    \begin{figure*}[htbp]
                        \includegraphics*[width = 12cm]{os_3.png}
                    \end{figure*}
                \newpage
                \item 其他函数的实现比较简单,这里不再赘述。
            \end{enumerate}
        \subsubsection{TEXTEDITOR}
            \par 在完成了 \textbf{LinkList} 的编写后,我们将其封装进 \textbf{TextEditor} 中,方便调用。
            \begin{figure*}[htbp]
                \includegraphics*[width = 12cm]{te_1.png}
            \end{figure*}
            \par 下面对每一个函数进行说明:
            \begin{enumerate}
                \item \textbf{TextEditor* init\_editor} 与 \textbf{void destory\_editor(TextEditor* editor, int oprt)}
                    \par 在 \textbf{init\_editor} 中,我们为其在堆区开辟了一片空间,并使用之前编写的 \textbf{init\_linklist} 初始化封装的链表。时间复杂度 \textbf{O(1)}
                    \par 在 \textbf{destory\_editor} 中,我们调用之前编写的 \textbf{destory\_linklist} 函数销毁链表,但由于在某些操作中,我们可能并不希望将 \textbf{editor} 也释放掉,所以增加了变量 \textbf{oprt},当 \textbf{oprt} $=$ \textbf{0} 时,才会执行 \textbf{free(editor)} 操作,彻底销毁整个 \textbf{TextEditor}。时间复杂度 \textbf{O(nm)}
                    \begin{figure*}[htbp]
                        \includegraphics*[width = 12cm]{te_2.png}
                    \end{figure*}
                \item \textbf{void show\_menu(void)}
                    \par 该函数用于打印可执行操作的选项。
                \item \textbf{bool read\_file(TextEditor** editor, const char* file\_name)}
                    \par 该函数用于实现读取文件的功能。由于我们可能在程序运行时从不同的文件中读入数据,因此每次读入前都需要将当前的数据释放掉,这里使用了之前编写的 \textbf{destory\_editor} 函数,同时我们希望能继续使用 \textbf{editor} 存储新的数据,因此 \textbf{oprt} $=$ \textbf{1}。
                    \par 需要注意的是,由于此处涉及到对指针分配内存,因此函数的参数应该是一个二重指针。
                    \par 由于需要将整个文本读入,因此时间复杂度为 \textbf{O(nm)}
                    \begin{figure*}[htbp]
                        \includegraphics*[width = 18cm]{te_5.png}
                    \end{figure*}
                \newpage
                \item \textbf{bool save\_file(TextEditor* editor, const char* file\_name)}
                    \par 该函数用于实现将数据写入文件的功能,我们遍历链表的每一行,同时调用 \textbf{get\_linkstring} 函数,将每一行存储的链串转换成字符串并存储到文件中。
                    \par 在函数中我们需要遍历链表的每一个结点,所以外层循环的迭代次数为 \textbf{n},根据之前的分析,每次调用 \textbf{get\_linkstring} 函数的时间复杂度为 \textbf{O(m)},因此总的时间复杂度为 \textbf{O(nm)}
                    \begin{figure*}[htbp]
                        \includegraphics*[width = 12cm]{te_6.png}
                    \end{figure*}
                \newpage
                \item \textbf{void print\_text(TextEditor* editor, int start, int end)}
                    \par 该函数用于打印指定范围的行,在打印时,会显示当前行数,方面对其进行操作。与之前类似,当输入的数据大于行数是默认打印至末尾。我们先使用 \textbf{get\_row} 函数获取起始结点,依次遍历剩下的结点,每次调用 \textbf{get\_linkstring} 函数获得字符串并打印。
                    \par 根据之前的分析,\textbf{get\_row} 函数和外层循环的时间复杂度为 \textbf{O(n)},调用 \textbf{get\_linkstring} 函数的时间复杂度为 \textbf{O(m)},总的时间复杂度为 \textbf{O(nm)}
                    \begin{figure*}[htbp]
                        \includegraphics*[width = 13cm]{te_4.png}
                    \end{figure*}
                \item \textbf{bool check\_oprt(TextEditor* editor)}
                    \par 该函数用于实现检测运算符是否匹配。具体实现原理是通过利用栈的后进先出性质,按行优先的顺序访问整个文本,将所有的 \textbf{\{, (} 运算符压入栈中,并查看之后出现的 \textbf{\}, )} 运算符是否与栈顶元素匹配。在实现基本功能的同时,通过一定的算法将运算符是否在头文件中、\textbf{comment} 中、 \textbf{`'}中的情况过滤。
                    \par 由于需要遍历一遍文本,因此时间复杂度为 \textbf{O(nm)}
                    \begin{figure*}[htbp]
                        \includegraphics*[width = 12cm]{te_7.png}
                    \end{figure*}
            \end{enumerate}
    \newpage
    \subsection{运行截图}
        \begin{enumerate}
            \item \textbf{运行程序}
                \par 运行程序,会先给出可用操作,并提示操作。
                \begin{figure*}[htbp]
                    \centering
                    \includegraphics*[width = 10cm]{op_1.png}
                \end{figure*}
            \item \textbf{操作一:读入文件}
                \par 输入 \textbf{1},程序提示输入文件名,如果文件不存在,则会提示失败;相反,会提示成功
                \begin{figure*}[htbp]
                    \centering
                    \includegraphics*[width = 7cm]{op_2.png}
                    \includegraphics*[width = 7.5cm]{op_3.png}
                \end{figure*}
            \item \textbf{操作二:输出指定范围的行}
                \par 输入 \textbf{2},程序提示输入范围,在越界时会默认输出至末尾
                \begin{figure*}[htbp]
                    \centering
                    \includegraphics*[width = 7cm]{op_4.png}
                    \includegraphics*[width = 7cm]{op_5.png}
                \end{figure*}
            \item \textbf{操作三:行插入}
                \par 输入 \textbf{3},程序提示输入指定行,在越界时默认在末尾插入
                \begin{figure*}[htbp]
                    \centering
                    \includegraphics*[width = 10cm]{op_6.png}
                \end{figure*}
                \newpage
                \par 查看插入后的效果
                \begin{figure*}[htbp]
                    \centering
                    \includegraphics*[width = 10cm]{op_7.png}
                \end{figure*}
            \item \textbf{操作四:行删除}
                \par 输入 \textbf{4},程序提示输入指定行和长度,在越界时默认从起始位置开始全部删除
                \begin{figure*}[htbp]
                    \centering
                    \includegraphics*[width = 12cm]{op_8.png}
                \end{figure*}
                \newpage
                \par 查看删除后的效果
                \begin{figure*}[htbp]
                    \centering
                    \includegraphics*[width = 12cm]{op_9.png}
                \end{figure*}
            \item \textbf{操作五:行内文本插入}
                \par 输入 \textbf{5},程序提示输入指定行和列,在越界时默认在末尾插入
                \begin{figure*}[htbp]
                    \centering
                    \includegraphics*[width = 12cm]{op_10.png}
                \end{figure*}
                \par 查看插入后的效果
                \begin{figure*}[htbp]
                    \centering
                    \includegraphics*[width = 11cm]{op_11.png}
                \end{figure*}
            \newpage
            \item \textbf{操作六:行内文本删除}
                \par 输入 \textbf{6},程序提示输入指定行和列,以及长度,在越界时默认从起始位置开始全部删除
                \begin{figure*}[htbp]
                    \centering
                    \includegraphics*[width = 12cm]{op_12.png}
                \end{figure*}
                \par 查看插入后的效果
                \begin{figure*}[htbp]
                    \centering
                    \includegraphics*[width = 10cm]{op_13.png}
                \end{figure*}
            \item \textbf{操作七:文本查找}
                \par 输入 \textbf{7},程序提示输入指定的行和想查找的字符串,如果成功,则返回字符串的起始位置(下标从 \textbf{0} 开始)
                \begin{figure*}[htbp]
                    \centering
                    \includegraphics*[width = 11cm]{op_14.png}
                \end{figure*}
            \newpage
            \item \textbf{操作八:检查运算符匹配}
                \par 输入 \textbf{8},程序自动检查运算符是否匹配并输出
                \begin{figure*}[htbp]
                    \centering
                    \includegraphics*[width = 7cm]{op_15.png}
                    \includegraphics*[width = 7cm]{op_17.png}
                \end{figure*}
                \begin{figure*}[htbp]
                    \centering
                    \includegraphics*[width = 6cm]{op_13.png}
                    \includegraphics*[width = 8cm]{op_16.png}
                \end{figure*}
                \par 包含所有特殊情况的例子如下
                \begin{figure*}[htbp]
                    \centering
                    \includegraphics*[width = 8cm]{op_19_1.png}
                \end{figure*}
            \item \textbf{操作九:保存文件}
                \par 输入 \textbf{9},程序保存文件,结束运行
                \begin{figure*}[htbp]
                    \centering
                    \includegraphics*[width = 12cm]{op_18.png}
                \end{figure*}
                \par 重新运行程序,将之前保存的文件读入并打印
                \begin{figure*}[htbp]
                    \centering
                    \includegraphics*[width = 10cm]{op_19.png}
                \end{figure*}
        \end{enumerate}
         
    \section{总结体会}
    \begin{itemize}
        \item 收获:
            \par 在一开始阅读汇编代码(尤其是第二个炸弹时),只知道从头到尾线性地去阅读,结果效率十分低下。在经过这次实验后,我知道了读汇编代码应该从整体入手,掌握大体框架后再去细致地看每一部分的内容;
            \par 同时,我也掌握了实际运用课程所学知识的能力,比如各个控制语句的汇编结构,函数参数传递的底层原理和过程等。
        \item 建议:
            \par 建议下次炸弹部署时间是中国时间,不是北美时间;
            \par 虽然可以使用文件避免重复输入,但好像 \textbf{gdb} 调试时并不支持,希望下次能把这个加上。

    \end{itemize} 
    \section{程序源代码}
    \subsection{实验内容一}
        \lstinputlisting[
            style = C,
            title = {\bf Array of Student Structures}
        ]{work1.c}
    \subsection{实验内容二}
        \lstinputlisting[
            style = C,
            title = {\bf Algorithm Execution Efficiency Measurement and Analysis}
        ]{work2.c}
\end{document}