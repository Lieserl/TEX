\section{引言}
\fontsize{12pt}{14pt}\selectfont
\songti
    \subsection{背景介绍}
        \subsubsection{矩阵理论与方法介绍}
            \par 作为数学的一个重要分支,矩阵理论具有极为丰富的内容. 其由矩阵运算、特征值和特征向量、线性方程组、矩阵分解、矩阵微积分等方面组成.
            \par 矩阵理论的历史可以追溯到 $18$ 世纪和 $19$ 世纪的数学发展,但它的真正崭露头角是在 $20$ 世纪早期。矩阵在矩阵代数、线性代数和线性空间中的研究变得非常重要。著名的数学家如 Sylvester、Cayley 和 Frobenius 等都为矩阵理论的发展作出了重要贡献。矩阵的应用范围迅速扩展,尤其是在工程、物理学、计算机科学和统计学等领域。
            \par 作为一种基本工具,矩阵理论在数学学科以及其他科学技术领域都有非常广泛的应用,包括但不限于:
            \begin{enumerate}
                \item \textbf{线性代数和线性方程组求解:}矩阵用于表示和求解线性方程组,如高斯消元法和矩阵的逆运算等;
                \item \textbf{统计学:}协方差矩阵和相关矩阵用于描述和分析随机变量之间的关系;
                \item \textbf{计算机图形学:}矩阵用于实现图像变换、旋转和投影,以及三维图形的转换;
                \item \textbf{机器学习和数据分析:}矩阵用于数据降维、特征选择和各种机器学习算法的实现。
            \end{enumerate}

        \subsubsection{函数矩阵和矩阵函数介绍}
            \par \textbf{函数矩阵:} 函数矩阵是一个矩阵,其元素不是常数,而是函数。通常,一个函数矩阵会包含一个矩阵中的每个元素都是一个函数,这些函数可能依赖于一个或多个变量。函数矩阵可以表示为:
            \begin{equation*}
                A(t) = \begin{bmatrix}
                    a_{11}(t) & a_{12}(t) & \ldots & a_{1n}(t) \\
                    a_{21}(t) & a_{22}(t) & \ldots & a_{2n}(t) \\
                    \vdots & \vdots & \ddots & \vdots \\
                    a_{m1}(t) & a_{m2}(t) & \ldots & a_{mn}(t)
                \end{bmatrix}
            \end{equation*}
            如果其中每一个元素都是变量 $t$ 的可导函数,则称 $A(t)$ 可导,其导数定义为
            \begin{equation*}
                A'(t) = \frac{d}{dt}A(t) = (\frac{d}{dt}a_{ij}(t))_{m \times n}
            \end{equation*}
            函数矩阵也有与普通函数相似的性质:
            \begin{align*}
                &\frac{d}{dt}(A(t) + B(t)) = \frac{d}{dt}A(t) + \frac{d}{dt}B(t) \\
                &\frac{d}{dt}(A(t)B(t)) = \frac{d}{dt}A(t) \cdot B(t) + A(t) \cdot \frac{d}{dt}B(t) \\ 
                &\frac{d}{dt}(aA(t)) = \frac{da}{dt}A(t) + a\frac{d}{dt}A(t) \\
            \end{align*}
            这里,$a = a(t)$ 为 $t$ 的可导函数.
            \par 总结一下,函数矩阵是一个矩阵,其元素是函数,而矩阵函数是一个函数,其输入和输出都是矩阵。这两个概念都在数学和工程领域中有广泛的应用,用于描述和处理复杂的多维数据和变量.
            \\
            \par \textbf{矩阵函数:} \par 矩阵函数是以矩阵为自变量且取值为矩阵的一类函数,它是对一元函数概念的推广. 起先,矩阵函数是由一个收敛的矩阵幂级数的和来定义,之后根据计算矩阵函数值的 Jordan 标准形方法又对矩阵函数的概念进行了拓宽.
            \par 设一元函数 $f(z)$ 能够展开为 $z$ 的幂级数 
            \begin{align*}
                f(z) = \sum_{k=0}^{\infty}c_kz^k \quad (|z| < r)
            \end{align*}
            其中 $r > 0$ 表示该幂级数的收敛半径。当 $n$ 阶矩阵 $A$ 的谱半径 $\rho(A) < r$ 时,把收敛的矩阵幂级数 $\sum_{k=0}^{\infty}c_kA^k$ 的和称为矩阵函数,记为 $f(A)$,即
            \begin{align*}
                f(A) = \sum_{k=0}^{\infty}c_kA^k \tag{3.3.2}
            \end{align*}
            \par 矩阵函数在多个领域中具有广泛的应用,包括但不限于数值分析、微分方程求解、量子力学、机器学习和数据分析、信号处理等方面.
            \par 总之,矩阵函数是数学和工程领域中的一个关键工具,用于处理和分析多维数据和系统. 它们在各种应用中发挥着重要作用,从数值计算到科学研究和工程设计等领域都有广泛的应用. 理解矩阵函数有助于解决复杂的多维问题,并优化系统的设计和性能.

        \subsubsection{线性代数方程组求解}
            \par 线性代数方程组求解的历史可以追溯到古希腊时期,但真正的突破发生在 $17$ 世纪. 其中最早的方法之一是高斯消元法,由卡尔·弗里德里希·高斯于 $1799$ 年提出。该方法通过行变换将线性方程组转化为上三角形式,从而容易求解. 后来,利普希茨和克拉默分别独立提出了克拉默法则和利普希茨条件,这些方法也用于线性方程组的求解.
            \par 随着计算机科学的发展,各种数值方法和算法被开发出来,用于求解大规模和高维度的线性方程组。这些方法包括迭代法(如雅可比迭代、高斯-赛德尔迭代和共轭梯度法)、直接法(如 LU 分解和 QR 分解)以及特征值分解等.
            \par 线性代数方程组求解的方法可以大致分为以下几类:
            \begin{enumerate}
                \item \textbf{直接法:}这些方法直接找到方程组的解,无需迭代。常见的直接法包括:
                    \begin{itemize}
                        \item \textbf{高斯消元法:}通过行变换将方程组转化为上三角形式,然后回代求解;
                        \item \textbf{LU 分解:}将系数矩阵分解为下三角矩阵和上三角矩阵的乘积,然后通过前代和回代求解;
                        \item \textbf{QR 分解:}将系数矩阵分解为正交矩阵和上三角矩阵的乘积,然后通过回代求解.
                    \end{itemize}
                \item \textbf{迭代法:}这些方法通过迭代逐步逼近方程组的解。常见的迭代法包括:
                    \begin{itemize}
                        \item \textbf{雅可比迭代法:}通过逐个更新解的各个分量来逼近精确解;
                        \item \textbf{高斯-赛德尔迭代法:}与雅可比迭代法类似,但每次更新使用最新计算的分量;
                        \item \textbf{共轭梯度法:}用于解决对称正定系数矩阵的线性方程组,通常用于优化问题.
                    \end{itemize}
                \item \textbf{特征值分解:}特征值和特征向量的方法可以用于求解对称或正定系数矩阵的线性方程组. 这包括使用 Cholesky 分解和对称幂法等方法;
                \item \textbf{数值软件:}许多数学库和软件包(如 NumPy、SciPy、MATLAB 等)提供了现成的线性代数方程组求解工具,使求解更加方便和高效.
            \end{enumerate}
            \par 线性代数方程组求解是数学和工程领域的一个重要问题,有着悠久的历史和多种方法. 选择合适的方法通常取决于问题的特点,例如系数矩阵的性质、方程组的规模和精度要求. 不同的方法都有各自的优点和局限性,熟练地选择和应用这些方法可以帮助有效地解决线性方程组问题.
        
    \subsection{问题介绍}
        \subsubsection{矩阵函数的求法问题介绍}
            \par 求解矩阵函数的方法取决于函数的类型和矩阵的特性. 常见的矩阵函数包括指数函数、对数函数、幂函数等.
            \par 求解矩阵函数主要的问题包括:
            \begin{enumerate}
                \item \textbf{定义和存在性:}不是所有的数学函数都可以直接应用于矩阵. 首先需要确定函数对于给定的矩阵是否有意义,以及如何适当地定义这个函数;
                \item \textbf{计算方法:}矩阵函数的计算通常比简单的标量函数复杂得多. 需要选择合适的计算方法,如泰勒级数展开、特征分解、Jordan 形式或者积分表示等,这取决于矩阵的性质和函数的类型;
                \item \textbf{数值稳定性:}在计算矩阵函数时,保持数值稳定性是一个重要的问题. 某些方法可能在数值上不稳定,尤其是当矩阵的条件数较大时.
            \end{enumerate}
            \par 下面是几种常见的矩阵函数求解方法:
            \begin{enumerate}
                \item \textbf{泰勒级数:}对于一些函数,比如指数函数和对数函数,可以通过将函数展开为泰勒级数来计算其在矩阵上的值;
                \item \textbf{特征值分解:}如果矩阵可对角化,即 $A = PDP^{-1}$,其中 $D$ 是对角矩阵,那么 $f(A)$ 可以通过先对 $D$ 中的每个特征值应用函数 $f$,然后重组矩阵来求得;
                \item \textbf{Jordan 形式:}对于不能对角化的矩阵,可以使用 Jordan 分解方法。这种方法涉及将矩阵转换为 Jordan 标准形式,然后在这种形式上应用函数;
                \item \textbf{积分表示法:}对于某些函数,可以通过积分的方式来定义矩阵函数.
            \end{enumerate}

        \subsubsection{矩阵分解的方法问题介绍}
            \par 矩阵分解,也称为矩阵因式分解,是将一个矩阵分解为几个具有特定特性的矩阵乘积的过程。这种分解在数值分析、信号处理、统计学、机器学习等领域中有广泛的应用,因为它有助于简化复杂的矩阵运算,揭示数据的潜在结构,以及提高算法的效率和稳定性.
            \par 矩阵分解的主要问题和挑战通常包括:
            \begin{enumerate}
                \item \textbf{计算复杂度:}矩阵分解通常涉及大量的计算,尤其是当处理大规模数据时. 因此,计算效率和优化是核心考虑因素;
                \item \textbf{数值稳定性:}在进行矩阵分解时,数值稳定性是一个重要问题. 某些分解方法可能在数值上不稳定,尤其是在处理条件数不佳的矩阵时;
                \item \textbf{适用性和限制:}不同的矩阵分解方法适用于不同类型的矩阵和问题. 例如,特征分解只适用于方阵,而奇异值分解(SVD)适用于任何形状的矩阵;
            \end{enumerate}
            \par 下面是几种常见的矩阵分解方法:
            \begin{enumerate}
                \item \textbf{LU 分解:}将矩阵分解为一个下三角矩阵(L)和一个上三角矩阵(U)。这种分解常用于解线性方程组、计算行列式和矩阵的逆;
                \item \textbf{QR 分解:}将矩阵分解为一个正交矩阵(Q)和一个上三角矩阵(R)。QR 分解在求解线性最小二乘问题和计算特征值等问题中非常有用;
                \item \textbf{奇异值分解(SVD):}将矩阵分解为三个矩阵的乘积,形式为 $U\varSigma V^H$. SVD 在信号处理、统计学和机器学习中非常有用,特别是在主成分分析(PCA)和低秩近似问题中;
                \item \textbf{特征分解:}也称为谱分解,是将矩阵分解为其特征向量和特征值的乘积. 特征分解在理解线性变换的本质方面很有帮助,但只适用于方阵.
            \end{enumerate}

    \subsection{上述问题国内外研究成果介绍}
        \subsubsection{矩阵函数的求法研究现状}
            \par 矩阵函数求解的主要聚焦方向包括数值方法改进、并行计算和分布式计算、高性能计算、自动化工具和软件的开发等方向. 具体包括:
            \begin{enumerate}
                \item \textbf{矩阵指数函数:}矩阵指数函数是一种常见的矩阵函数,它经常出现在线性常微分方程、控制理论和量子力学等领域中. 目前有多种数值方法用于计算矩阵指数函数,包括 Pade 逼近、泰勒级数展开、Krylov 子空间方法等。
                \item \textbf{矩阵对数函数:}矩阵对数函数在矩阵计算中也是重要的. 它的计算通常涉及到特征值分解、Schur分解或极化恒等式等技术;
                \item \textbf{矩阵函数的插值方法:}近年来,研究人员提出了基于插值的方法,用于计算矩阵函数的近似值. 这些方法利用一系列已知的矩阵值来逼近矩阵函数,然后进行插值以获得更精确的近似;
                \item \textbf{高性能计算和并行计算:}随着计算机性能的提高,矩阵函数的求解也得到了显著的改进。并行计算和分布式计算技术被广泛应用于加速矩阵函数的计算.
            \end{enumerate}

        \subsubsection{矩阵分解方法研究现状}
            \par 矩阵分解的主要聚焦方向包括提高经典分解方法的效率和稳健性,探索高阶矩阵和张量分解方法,以及将深度学习技术与矩阵分解方法相结合等,具体如下:
            \begin{enumerate}
                \item \textbf{奇异值分解(SVD):}奇异值分解是一种经典的矩阵分解方法,用于将矩阵分解为三个矩阵的乘积,通常应用于数据降维、主成分分析(PCA)和推荐系统等领域. 研究人员一直在寻找更高效和稀疏的 SVD 算法,以处理大规模数据;
                \item \textbf{非负矩阵分解(NMF):}非负矩阵分解是用于数据聚类和特征提取的方法,它约束了分解出的矩阵因子为非负值; 近年来,研究者提出了各种改进的 NMF 算法,包括多任务 NMF、稀疏 NMF 等;
                \item \textbf{Tucker分解和高阶矩阵分解:}对于高阶张量和多维数据,Tucker 分解和高阶矩阵分解方法变得越来越重要。这些方法有助于提取数据中的高级结构和模式;
                \item \textbf{稀疏矩阵分解:}稀疏矩阵分解方法用于处理大规模和稀疏数据。研究者关注如何设计更加高效的稀疏分解算法;
                \item \textbf{深度学习方法:}深度学习方法如神经网络和自动编码器也可以看作是矩阵分解方法的一种,研究人员不断探索如何将深度学习与传统的矩阵分解方法相结合,以获得更好的性能.
            \end{enumerate}

    \subsection{本论文工作简介}
        \subsubsection{本论文对上述问题的研究简述}
            \par 本文首先对矩阵理论与方法进行了概述,强调了矩阵在数学和工程领域中的重要性. 随后,我们提出了线性代数方程组的求解作为研究的第一个基础,并强调了其在解决实际问题中的关键作用. 接着,我们介绍了矩阵函数、函数矩阵和矩阵分解,这些概念为进一步的矩阵分析奠定了基础. 并在各部分给出了详细的代码实现.

        \subsubsection{本论文创新点或特点简述}
            \par 本论文由浅入深、由面及点地提供了一套详尽的矩阵函数的求解方法和矩阵分解方法的推导过程和使用方法,同时提供了部分代码实现,契合了当今时代主题.

        \subsubsection{本论文撰写结构简述}
            \par 本文围绕矩阵理论中的矩阵函数和矩阵分解的求解为核心展开:
            \begin{enumerate}
                \item \textbf{第一章:}详细介绍了欧氏空间、线性变换、向量范数、矩阵范数、矩阵函数等概念,为下一步对各种求解方法提供理论基础;
                \item \textbf{第二章:}从待定系数法、数项级数求和法、对角型法、若尔当标准型法开始介绍了现有的矩阵函数的求法及其推导过程和使用示例;
                \item \textbf{第三章:}深入探讨矩阵分解的方法,提供了详细地推导过程和代码实现.
            \end{enumerate}

